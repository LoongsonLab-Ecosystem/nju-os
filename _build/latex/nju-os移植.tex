%% Generated by Sphinx.
\def\sphinxdocclass{report}
\documentclass[a4paper,10pt,english,openany,fleqn]{sphinxmanual}
\ifdefined\pdfpxdimen
   \let\sphinxpxdimen\pdfpxdimen\else\newdimen\sphinxpxdimen
\fi \sphinxpxdimen=.75bp\relax
\ifdefined\pdfimageresolution
    \pdfimageresolution= \numexpr \dimexpr1in\relax/\sphinxpxdimen\relax
\fi
%% let collapsible pdf bookmarks panel have high depth per default
\PassOptionsToPackage{bookmarksdepth=5}{hyperref}
%% turn off hyperref patch of \index as sphinx.xdy xindy module takes care of
%% suitable \hyperpage mark-up, working around hyperref-xindy incompatibility
\PassOptionsToPackage{hyperindex=false}{hyperref}
%% memoir class requires extra handling
\makeatletter\@ifclassloaded{memoir}
{\ifdefined\memhyperindexfalse\memhyperindexfalse\fi}{}\makeatother


\PassOptionsToPackage{warn}{textcomp}

\catcode`^^^^00a0\active\protected\def^^^^00a0{\leavevmode\nobreak\ }

\usepackage[T1]{fontenc}
\usepackage{amsmath,amssymb,amstext}
\usepackage{babel}


\usepackage[UTF8, fontset=none]{ctex}
\usepackage{fontspec}

\setmainfont{DejaVuSans.ttf}[BoldFont=DejaVuSans-Bold.ttf, ItalicFont=DejaVuSans-Oblique.ttf]
\setsansfont{DejaVuSans.ttf}[BoldFont=DejaVuSans-Bold.ttf, ItalicFont=DejaVuSans-Oblique.ttf]
\setmonofont{DejaVuSans.ttf}[BoldFont=DejaVuSans-Bold.ttf, ItalicFont=DejaVuSans-Oblique.ttf]

\setCJKmainfont{SourceHanSansSC-Regular.otf}[AutoFakeBold,AutoFakeSlant]
\setCJKsansfont{SourceHanSansSC-Regular.otf}[AutoFakeBold,AutoFakeSlant]
\setCJKmonofont{SourceHanSansSC-Regular.otf}[AutoFakeBold,AutoFakeSlant]



\usepackage[Sonny]{fncychap}
\usepackage{sphinx}

\fvset{fontsize=\small,formatcom=\xeCJKVerbAddon}
\usepackage[left=2cm, right=2cm, top=2.54cm, bottom=2.54cm]{geometry}


% Include hyperref last.
\usepackage{hyperref}
% Fix anchor placement for figures with captions.
\usepackage{hypcap}% it must be loaded after hyperref.
% Set up styles of URL: it should be placed after hyperref.
\urlstyle{same}


\usepackage{sphinxmessages}
\setcounter{tocdepth}{2}

%%
%%
%% Start of EVAS pdf style preamble configuration (preamble.tex)
%%
%%
\usepackage{docinfo}

\setcounter{secnumdepth}{4}
\setcounter{tocdepth}{2}

\usepackage{amsmath,amsfonts,amssymb,amsthm}
\usepackage{graphicx}
%%% reduce spaces for Table of contents, figures and tables
%%% it is used "\addtocontents{toc}{\vskip -1.2cm}" etc. in the document
\usepackage[notlot,nottoc,notlof]{}

% Optimize the display effect
\usepackage[verbose=silent]{microtype}

\usepackage{silence}
\WarningsOff[everypage]

% Set all the text contents align to left
\usepackage{ragged2e}  
\setlength{\RaggedRightRightskip}{0pt plus 3em}
\RaggedRight

\usepackage{color}
\usepackage{eso-pic}

%% spacing between line
\usepackage{setspace}
\setstretch{1.2}
% \singlespacing

%% table line space
\renewcommand{\arraystretch}{1.5}

\definecolor{myblue}{RGB}{47, 85, 151}
\definecolor{mygray}{RGB}{127, 127, 127}

%%%%%%%%%%% datetime
\usepackage{datetime}

\newdateformat{MonthYearFormat}{%
    \monthname[\THEMONTH], \THEYEAR}


%% RO, LE will not work for 'oneside' layout.
%% Change oneside to twoside in document class
\usepackage{fancyhdr}
\pagestyle{fancy}
\fancyhf{}

% Set chapter text with number
\renewcommand{\chaptermark}[1]{\markboth{#1}{}}

% Header and footer
\makeatletter
\def\@dotsep{2}
\fancypagestyle{normal}{
    \fancyhf{}
    \fancyhead[L]{\@title}
    \fancyhead[R]{\ifnum\thechapter=0 \leftmark \else \thechapter.\leftmark \fi}
    \fancyfoot[L]{\@author}
    \fancyfoot[C]{\thepage}
    \fancyfoot[R]{\version}
    \renewcommand{\headrulewidth}{0.4pt}
    \renewcommand{\footrulewidth}{0.4pt}
}
\makeatother

% remove visible space and other symbols in line break
\makeatletter
\def\spx@verbatim@space {%
     \nobreak\hskip\z@skip
     \discretionary{}{}{\kern\fontdimen2\font}%
}%
\makeatother

% Set the header and footer divider thickness
\renewcommand{\headrulewidth}{0.5pt}
\renewcommand{\footrulewidth}{0.5pt}

% Set Margins
\setlength{\headheight}{12.12004pt}
\setlength{\headsep}{10mm}
\addtolength{\topmargin}{-0.12004pt}

% Define a spacing for section, subsection and subsubsection
% http://tex.stackexchange.com/questions/108684/spacing-before-and-after-section-titles

\titlespacing*{\chapter}{0pt}{-30pt plus 0pt minus 0pt}{6pt plus 0pt minus 0pt}
\titlespacing*{\section}{0pt}{6pt plus 0pt minus 0pt}{6pt plus 0pt minus 0pt}
\titlespacing*{\subsection}{0pt}{6pt plus 0pt minus 0pt}{6pt plus 0pt minus 0pt}
\titlespacing*{\subsubsection}{0pt}{6pt plus 0pt minus 0pt}{6pt plus 0pt minus 0pt}
\titlespacing*{\paragraph}{0pt}{0pt plus 0pt minus 0pt}{0pt plus 0pt minus 0pt}
\titlespacing*{\subparagraph}{0pt}{0pt plus 0pt minus 0pt}{0pt plus 0pt minus 0pt}

% Define the colors of Hyperlinks
\hypersetup{
	colorlinks=true,
	linkcolor=black,
	filecolor=blue,
	urlcolor=blue,
	citecolor=black,
}

% Define some content name to chinese
\addto\captionsenglish{
    \renewcommand{\contentsname}{目录}
    \renewcommand{\listfigurename}{插图目录}
    \renewcommand{\listtablename}{表格目录}
    \renewcommand{\refname}{参考文献}
    \renewcommand{\abstractname}{摘要}
    \renewcommand{\indexname}{索引}
    \renewcommand{\tablename}{表}
    \renewcommand{\figurename}{图}
}


%%%
%%% Watermark setting
%%%

\usepackage{background}
\SetBgContents{
    \begin{tabular}{c}
        \hspace{-5cm}	\watermarktext \\
        \vspace{6cm}                  \\
        \hspace{10cm}	\watermarktext   \\
        \vspace{6cm}                  \\
        \hspace{-10cm}	\watermarktext
        \vspace{6cm}                  \\
        \hspace{15cm}	\watermarktext
    \end{tabular}\hfill}
\SetBgScale{1}
\SetBgAngle{45}

% reduce spacing for itemize
\usepackage{enumitem}
\setlist{nosep}

% Quote Styles
\AtBeginEnvironment{quote}{\slshape\par\singlespacing}

% space between digit and title
\usepackage{titlesec}
\titleformat{\chapter}
{\normalcolor\LARGE\bfseries}{\thechapter.}{0.2em}{}

\titleformat{\section}
{\normalcolor\LARGE\bfseries}{\thesection.}{0.2em}{}

\titleformat{\subsection}
{\normalcolor\Large\bfseries}{\thesubsection.}{0.2em}{}

\titleformat{\subsubsection}
{\normalcolor\large\bfseries}{\thesubsubsection.}{0.2em}{}

\titleformat{\paragraph}
{\normalcolor\large\bfseries}{\theparagraph}{1em}{}

\titleformat{\subparagraph}
{\normalcolor\normalsize\bfseries}{\thesubparagraph}{1em}{}

\titleformat{\subsubparagraph}
{\normalcolor\normalsize\bfseries}{\thesubsubparagraph}{1em}{}

\assignpagestyle{\chapter}{normal}

%%
%%
%% End of EVAS pdf style preamble configuration  (preamble.tex)
%%
%%

\title{NJU-OS移植}
\date{2026 年 01 月 23 日}
\release{V1.0}
\author{白子尧}
\newcommand{\sphinxlogo}{\sphinxincludegraphics{logo.pdf}\par}
\renewcommand{\releasename}{发行版本}
\makeindex
\begin{document}

\ifdefined\shorthandoff
  \ifnum\catcode`\=\string=\active\shorthandoff{=}\fi
  \ifnum\catcode`\"=\active\shorthandoff{"}\fi
\fi

\pagestyle{empty}
%%
%%
%% Start of EVAS pdf title page style configuration (titlepage.tex)
%%
%%

\makeatletter
\newgeometry{left=0cm,right=0cm,top=0cm,bottom=0cm}

\renewcommand{\headrulewidth}{0pt}

\begin{flushright}
	\begin{minipage}[b]{5cm}
		\vspace{20pt}
		\sphinxlogo
	\end{minipage}
	\qquad\qquad\qquad\qquad
\end{flushright}

\vspace{4cm}

{\color{myblue}\rule{30pt}{6cm}
    \hspace{0.3cm}
    \begin{minipage}[b]{19cm}
	    {\fontsize{36pt}{1pt}\textbf{\color{mygray}\@title}\vspace{0.4cm}}\\
	    {\fontsize{28pt}{1pt}\textbf{\color{mygray}\subtitle}\vspace{0.4cm}}
    \end{minipage}
}
\vspace{10cm}

\begin{flushright}
    \setlength\parindent{10cm}
    \hspace{0.2cm}
    \begin{minipage}[b]{8cm}
        {\LARGE{\draftorrelease\hspace{0.5em}\version}\vspace{0.3em}}\smallskip\newline
        {\LARGE{\@author}\vspace{0.2em}}\smallskip\newline
        {\LARGE{\@date}\vspace{1em}}\smallskip
    \end{minipage}
    {\color{myblue}\rule{30pt}{4cm}}
\end{flushright}

\begin{minipage}[b]{20cm}
	\vspace{4em}\hspace{1cm}
	\large{$\copyright$\hspace{0.3em}\the\year\hspace{0.3em} EVAS Intelligence, all rights reserved}\smallskip\newline
\end{minipage}

\restoregeometry
\makeatother

%%
%%
%% End of EVAS pdf title page style configuration (titlepage.tex)
%%
%%
\pagestyle{plain}
\pagestyle{normal}
    \sphinxtableofcontents
\pagestyle{normal}
\phantomsection\label{\detokenize{index::doc}}


\sphinxstepscope


\chapter{项目介绍}
\label{\detokenize{README:id1}}\label{\detokenize{README::doc}}
\sphinxAtStartPar
NJU\sphinxhyphen{}OS是南京大学的一个教学操作系统,现有x86版本。 LoongArch是由我国龙芯中科研发的自主指令系统(龙芯架构)。 本项目基于xv6\sphinxhyphen{}loongarch将NJU\sphinxhyphen{}OS移植到LoongArch平台上,在Ubuntu 20.04中通过QEMU模拟器(在PC上模拟LoongArch硬件)编译NJU\sphinxhyphen{}OS并运行,完成NJU\sphinxhyphen{}OS的三个实验(跳过实验一)。


\section{环境准备}
\label{\detokenize{README:id2}}
\sphinxAtStartPar
本实验所用的交叉编译工具链为:\\
\sphinxhref{https://github.com/loongson/build-tools/releases/download/2025.08.08/x86\_64-cross-tools-loongarch64-binutils\_2.45-gcc\_15.1.0-glibc\_2.42.tar.xz}{交叉编译工具链下载地址}\\
下载后通过共享文件夹到Ubuntu\\
设置路径:

\begin{sphinxVerbatim}[commandchars=\\\{\}]
\PYG{n+nb}{export}\PYG{+w}{ }\PYG{n+nv}{PATH}\PYG{o}{=}\PYG{l+s+s2}{\PYGZdq{}}\PYG{l+s+s2}{\PYGZob{}/path/to/cross\PYGZhy{}tools\PYGZcb{}/bin:}\PYG{n+nv}{\PYGZdl{}PATH}\PYG{l+s+s2}{\PYGZdq{}}
\end{sphinxVerbatim}

\sphinxAtStartPar
上述的命令只是临时设置环境变量,如需永久设置,可通过修改/etc/profile实现;

\begin{sphinxVerbatim}[commandchars=\\\{\}]
loongarch64\PYGZhy{}unknown\PYGZhy{}linux\PYGZhy{}gnu\PYGZhy{}gcc\PYG{+w}{ }\PYGZhy{}\PYGZhy{}version
\end{sphinxVerbatim}

\sphinxAtStartPar
看到如下提示则说明已正确设置\\
\sphinxincludegraphics{{01}.png}


\section{编译NJU\sphinxhyphen{}OS内核}
\label{\detokenize{README:nju-os}}
\sphinxAtStartPar
下载NJU\sphinxhyphen{}OS\sphinxhyphen{}loongarch

\begin{sphinxVerbatim}[commandchars=\\\{\}]
\PYG{n}{git} \PYG{n}{clone} \PYG{n}{git}\PYG{n+nd}{@github}\PYG{o}{.}\PYG{n}{com}\PYG{p}{:}\PYG{n}{bzy666}\PYG{o}{\PYGZhy{}}\PYG{l+m+mi}{666}\PYG{o}{/}\PYG{n}{NJU}\PYG{o}{\PYGZhy{}}\PYG{n}{OS}\PYG{o}{\PYGZhy{}}\PYG{n}{loongarch}\PYG{o}{.}\PYG{n}{git}
\PYG{n}{cd} \PYG{p}{\PYGZob{}}\PYG{n}{choose} \PYG{n}{a} \PYG{n}{lab}\PYG{p}{\PYGZcb{}}
\end{sphinxVerbatim}

\sphinxAtStartPar
开始编译

\begin{sphinxVerbatim}[commandchars=\\\{\}]
\PYG{n}{make} \PYG{n+nb}{all}
\end{sphinxVerbatim}

\sphinxAtStartPar
终端会输出包括如下部分的编译信息\\
\sphinxincludegraphics{{02}.png}

\sphinxAtStartPar
当前路径下会生成\sphinxcode{\sphinxupquote{fs.img}}文件,\sphinxcode{\sphinxupquote{\textbackslash{}kernel}}下也会生成所有链接时需要的*.o, *.d等文件,以及最终的\sphinxcode{\sphinxupquote{kernel}}二进制文件,编译成功!

\sphinxstepscope


\chapter{lab1}
\label{\detokenize{nju:lab1}}\label{\detokenize{nju::doc}}

\section{实验目的}
\label{\detokenize{nju:id1}}\begin{enumerate}
\sphinxsetlistlabels{\arabic}{enumi}{enumii}{}{.}%
\item {} 
\sphinxAtStartPar
实现一个简单的应用程序,并在其中调用两个自定义实现的系统调用。

\item {} 
\sphinxAtStartPar
了解基于中断实现系统调用的全过程。

\end{enumerate}


\section{实验内容}
\label{\detokenize{nju:id2}}\begin{enumerate}
\sphinxsetlistlabels{\arabic}{enumi}{enumii}{}{.}%
\item {} 
\sphinxAtStartPar
内核初始化。

\item {} 
\sphinxAtStartPar
内核态向用户态的跳转(原实验为:Bootloader 从实模式进入保护模式,加载内核至内存,并跳转执行,内核加载用户程序至内存,对内核堆栈进行设置,通过 iret 切换至用户空间,执行用户程序。)

\item {} 
\sphinxAtStartPar
用户程序调用自定义实现的库函数 scanf 完成格式化输入和 printf 完成格式化输出。

\item {} 
\sphinxAtStartPar
scanf 基于中断陷入内核,内核扫描按键状态获取输入完成格式化输入(现阶段不需要考虑键盘中断)。

\item {} 
\sphinxAtStartPar
printf 基于中断陷入内核,由内核完成在视频映射的显存地址中写入内容,完成字符串的打印。

\end{enumerate}


\section{实验过程}
\label{\detokenize{nju:id3}}

\subsection{内核态初始化}
\label{\detokenize{nju:id4}}

\subsubsection{kinit()}
\label{\detokenize{nju:kinit}}\begin{enumerate}
\sphinxsetlistlabels{\arabic}{enumi}{enumii}{}{.}%
\item {} 
\sphinxAtStartPar
初始化锁。

\item {} 
\sphinxAtStartPar
初始化可用内存范围,将物理内存范围 {[}RAMBASE, RAMSTOP{]}标记为空闲可用。

\end{enumerate}


\subsubsection{vminit()}
\label{\detokenize{nju:vminit}}\begin{enumerate}
\sphinxsetlistlabels{\arabic}{enumi}{enumii}{}{.}%
\item {} 
\sphinxAtStartPar
分配一页物理内存作为内核页表,并清空页表内容(0)。

\item {} 
\sphinxAtStartPar
为进程设置内核栈的虚拟地址映射。

\item {} 
\sphinxAtStartPar
将内核页的物理地址写入pgdl寄存器。

\item {} 
\sphinxAtStartPar
初始化tlb。

\item {} 
\sphinxAtStartPar
设置了遍历页表的参数。

\end{enumerate}


\subsubsection{trapinit()}
\label{\detokenize{nju:trapinit}}\begin{enumerate}
\sphinxsetlistlabels{\arabic}{enumi}{enumii}{}{.}%
\item {} 
\sphinxAtStartPar
初始化锁。

\item {} 
\sphinxAtStartPar
配置异常配置寄存器ECFG,当中断发生时,CPU跳转到不同的处理函数,而不是统一的异常入口。

\item {} 
\sphinxAtStartPar
配置定时器配置寄存器TCFG,定时器每计数0x1000000个CPU周期就触发一次中断。

\item {} 
\sphinxAtStartPar
设置异常入口点,发生异常时,CPU跳转到kernelvec开始执行。

\item {} 
\sphinxAtStartPar
设置TLB重填入口,当页表缺页时调用。

\item {} 
\sphinxAtStartPar
设置机器错误异常处理入口,硬件错误时调用。

\item {} 
\sphinxAtStartPar
开启中断。

\end{enumerate}


\subsubsection{apic\_init()}
\label{\detokenize{nju:apic-init}}\begin{enumerate}
\sphinxsetlistlabels{\arabic}{enumi}{enumii}{}{.}%
\item {} 
\sphinxAtStartPar
配置uart0串口中断,只允许uart0中断,屏蔽其他所有硬件中断。

\end{enumerate}


\subsubsection{extioi\_init()}
\label{\detokenize{nju:extioi-init}}\begin{enumerate}
\sphinxsetlistlabels{\arabic}{enumi}{enumii}{}{.}%
\item {} 
\sphinxAtStartPar
使能中断。

\end{enumerate}


\subsubsection{binit()}
\label{\detokenize{nju:binit}}
\sphinxAtStartPar
构建双向循环链表。\\
双向循环链表:每个节点都有两个指针,一个指向前一个节点,一个指向后一个节点,因此可以从任意一个节点开始,双向遍历整个链表。\\
\sphinxincludegraphics{{img1}.png}


\subsubsection{userinit()}
\label{\detokenize{nju:userinit}}\begin{enumerate}
\sphinxsetlistlabels{\arabic}{enumi}{enumii}{}{.}%
\item {} 
\sphinxAtStartPar
分配用户进程。

\item {} 
\sphinxAtStartPar
将init程序的代码和数据映射到用户地址空间,设置用户态执行上下文,为第一次内核返回用户态做准备。

\item {} 
\sphinxAtStartPar
设置进程名称,工作目录,进程状态,准备开启调度。

\end{enumerate}


\subsection{内核态向用户态跳转}
\label{\detokenize{nju:id5}}
\sphinxAtStartPar
调度器遍历所有进程,找到状态为runnable的进程,设置状态为running,进行上下文的切换,跳转到p\sphinxhyphen{}>context.ra所设置的forkret函数,先进行文件系统初始化,读取超级块进行魔数判断,成功后进入usertrapret(),设置下次trap的入口,保存内核状态到陷阱帧,设置返回地址,获取用户页表物理地址,进入userret(),通过ertn跳转到用户态。


\subsection{用户态}
\label{\detokenize{nju:id6}}
\sphinxAtStartPar
在init.c中调用了printf和scanf函数,下面分别阐明调用过程和实现机制。


\subsubsection{printf}
\label{\detokenize{nju:printf}}
\begin{sphinxVerbatim}[commandchars=\\\{\}]
\PYG{k+kt}{void}
\PYG{n+nf}{printf}\PYG{p}{(}\PYG{k}{const}\PYG{+w}{ }\PYG{k+kt}{char}\PYG{+w}{ }\PYG{o}{*}\PYG{n}{fmt}\PYG{p}{,}\PYG{+w}{ }\PYG{p}{.}\PYG{p}{.}\PYG{p}{.}\PYG{p}{)}
\PYG{p}{\PYGZob{}}
\PYG{+w}{  }\PYG{k+kt}{va\PYGZus{}list}\PYG{+w}{ }\PYG{n}{ap}\PYG{p}{;}\PYG{+w}{ }
\PYG{+w}{  }\PYG{n}{va\PYGZus{}start}\PYG{p}{(}\PYG{n}{ap}\PYG{p}{,}\PYG{+w}{ }\PYG{n}{fmt}\PYG{p}{)}\PYG{p}{;}\PYG{+w}{ }
\PYG{+w}{  }\PYG{n}{vprintf}\PYG{p}{(}\PYG{l+m+mi}{1}\PYG{p}{,}\PYG{+w}{ }\PYG{n}{fmt}\PYG{p}{,}\PYG{+w}{ }\PYG{n}{ap}\PYG{p}{)}\PYG{p}{;}
\PYG{p}{\PYGZcb{}}
\end{sphinxVerbatim}

\sphinxAtStartPar
通过va\_start(ap, fmt)指向第一个可变参数,进入vprintf。

\begin{sphinxVerbatim}[commandchars=\\\{\}]
\PYG{k+kt}{void}
\PYG{n+nf}{vprintf}\PYG{p}{(}\PYG{k+kt}{int}\PYG{+w}{ }\PYG{n}{fd}\PYG{p}{,}\PYG{+w}{ }\PYG{k}{const}\PYG{+w}{ }\PYG{k+kt}{char}\PYG{+w}{ }\PYG{o}{*}\PYG{n}{fmt}\PYG{p}{,}\PYG{+w}{ }\PYG{k+kt}{va\PYGZus{}list}\PYG{+w}{ }\PYG{n}{ap}\PYG{p}{)}
\PYG{p}{\PYGZob{}}
\PYG{+w}{  }\PYG{k+kt}{char}\PYG{+w}{ }\PYG{o}{*}\PYG{n}{s}\PYG{p}{;}
\PYG{+w}{  }\PYG{k+kt}{int}\PYG{+w}{ }\PYG{n}{c}\PYG{p}{,}\PYG{+w}{ }\PYG{n}{i}\PYG{p}{,}\PYG{+w}{ }\PYG{n}{state}\PYG{p}{;}

\PYG{+w}{  }\PYG{n}{state}\PYG{+w}{ }\PYG{o}{=}\PYG{+w}{ }\PYG{l+m+mi}{0}\PYG{p}{;}
\PYG{+w}{  }\PYG{k}{for}\PYG{p}{(}\PYG{n}{i}\PYG{+w}{ }\PYG{o}{=}\PYG{+w}{ }\PYG{l+m+mi}{0}\PYG{p}{;}\PYG{+w}{ }\PYG{n}{fmt}\PYG{p}{[}\PYG{n}{i}\PYG{p}{]}\PYG{p}{;}\PYG{+w}{ }\PYG{n}{i}\PYG{o}{+}\PYG{o}{+}\PYG{p}{)}\PYG{p}{\PYGZob{}}
\PYG{+w}{    }\PYG{n}{c}\PYG{+w}{ }\PYG{o}{=}\PYG{+w}{ }\PYG{n}{fmt}\PYG{p}{[}\PYG{n}{i}\PYG{p}{]}\PYG{+w}{ }\PYG{o}{\PYGZam{}}\PYG{+w}{ }\PYG{l+m+mh}{0xff}\PYG{p}{;}
\PYG{+w}{    }\PYG{k}{if}\PYG{p}{(}\PYG{n}{state}\PYG{+w}{ }\PYG{o}{=}\PYG{o}{=}\PYG{+w}{ }\PYG{l+m+mi}{0}\PYG{p}{)}\PYG{p}{\PYGZob{}}
\PYG{+w}{      }\PYG{k}{if}\PYG{p}{(}\PYG{n}{c}\PYG{+w}{ }\PYG{o}{=}\PYG{o}{=}\PYG{+w}{ }\PYG{l+s+sc}{\PYGZsq{}}\PYG{l+s+sc}{\PYGZpc{}}\PYG{l+s+sc}{\PYGZsq{}}\PYG{p}{)}\PYG{p}{\PYGZob{}}
\PYG{+w}{        }\PYG{n}{state}\PYG{+w}{ }\PYG{o}{=}\PYG{+w}{ }\PYG{l+s+sc}{\PYGZsq{}}\PYG{l+s+sc}{\PYGZpc{}}\PYG{l+s+sc}{\PYGZsq{}}\PYG{p}{;}
\PYG{+w}{      }\PYG{p}{\PYGZcb{}}\PYG{+w}{ }\PYG{k}{else}\PYG{+w}{ }\PYG{p}{\PYGZob{}}
\PYG{+w}{        }\PYG{n}{putc}\PYG{p}{(}\PYG{n}{fd}\PYG{p}{,}\PYG{+w}{ }\PYG{n}{c}\PYG{p}{)}\PYG{p}{;}
\PYG{+w}{      }\PYG{p}{\PYGZcb{}}
\PYG{+w}{    }\PYG{p}{\PYGZcb{}}\PYG{+w}{ }\PYG{k}{else}\PYG{+w}{ }\PYG{k}{if}\PYG{p}{(}\PYG{n}{state}\PYG{+w}{ }\PYG{o}{=}\PYG{o}{=}\PYG{+w}{ }\PYG{l+s+sc}{\PYGZsq{}}\PYG{l+s+sc}{\PYGZpc{}}\PYG{l+s+sc}{\PYGZsq{}}\PYG{p}{)}\PYG{p}{\PYGZob{}}
\PYG{+w}{      }\PYG{k}{if}\PYG{p}{(}\PYG{n}{c}\PYG{+w}{ }\PYG{o}{=}\PYG{o}{=}\PYG{+w}{ }\PYG{l+s+sc}{\PYGZsq{}}\PYG{l+s+sc}{d}\PYG{l+s+sc}{\PYGZsq{}}\PYG{p}{)}\PYG{p}{\PYGZob{}}
\PYG{+w}{        }\PYG{n}{printint}\PYG{p}{(}\PYG{n}{fd}\PYG{p}{,}\PYG{+w}{ }\PYG{n}{va\PYGZus{}arg}\PYG{p}{(}\PYG{n}{ap}\PYG{p}{,}\PYG{+w}{ }\PYG{k+kt}{int}\PYG{p}{)}\PYG{p}{,}\PYG{+w}{ }\PYG{l+m+mi}{10}\PYG{p}{,}\PYG{+w}{ }\PYG{l+m+mi}{1}\PYG{p}{)}\PYG{p}{;}
\PYG{+w}{      }\PYG{p}{\PYGZcb{}}\PYG{+w}{ }\PYG{k}{else}\PYG{+w}{ }\PYG{k}{if}\PYG{p}{(}\PYG{n}{c}\PYG{+w}{ }\PYG{o}{=}\PYG{o}{=}\PYG{+w}{ }\PYG{l+s+sc}{\PYGZsq{}}\PYG{l+s+sc}{l}\PYG{l+s+sc}{\PYGZsq{}}\PYG{p}{)}\PYG{+w}{ }\PYG{p}{\PYGZob{}}
\PYG{+w}{        }\PYG{n}{printint}\PYG{p}{(}\PYG{n}{fd}\PYG{p}{,}\PYG{+w}{ }\PYG{n}{va\PYGZus{}arg}\PYG{p}{(}\PYG{n}{ap}\PYG{p}{,}\PYG{+w}{ }\PYG{n}{uint64}\PYG{p}{)}\PYG{p}{,}\PYG{+w}{ }\PYG{l+m+mi}{10}\PYG{p}{,}\PYG{+w}{ }\PYG{l+m+mi}{0}\PYG{p}{)}\PYG{p}{;}
\PYG{+w}{      }\PYG{p}{\PYGZcb{}}\PYG{+w}{ }\PYG{k}{else}\PYG{+w}{ }\PYG{k}{if}\PYG{p}{(}\PYG{n}{c}\PYG{+w}{ }\PYG{o}{=}\PYG{o}{=}\PYG{+w}{ }\PYG{l+s+sc}{\PYGZsq{}}\PYG{l+s+sc}{x}\PYG{l+s+sc}{\PYGZsq{}}\PYG{p}{)}\PYG{+w}{ }\PYG{p}{\PYGZob{}}
\PYG{+w}{        }\PYG{n}{printint}\PYG{p}{(}\PYG{n}{fd}\PYG{p}{,}\PYG{+w}{ }\PYG{n}{va\PYGZus{}arg}\PYG{p}{(}\PYG{n}{ap}\PYG{p}{,}\PYG{+w}{ }\PYG{k+kt}{int}\PYG{p}{)}\PYG{p}{,}\PYG{+w}{ }\PYG{l+m+mi}{16}\PYG{p}{,}\PYG{+w}{ }\PYG{l+m+mi}{0}\PYG{p}{)}\PYG{p}{;}
\PYG{+w}{      }\PYG{p}{\PYGZcb{}}\PYG{+w}{ }\PYG{k}{else}\PYG{+w}{ }\PYG{k}{if}\PYG{p}{(}\PYG{n}{c}\PYG{+w}{ }\PYG{o}{=}\PYG{o}{=}\PYG{+w}{ }\PYG{l+s+sc}{\PYGZsq{}}\PYG{l+s+sc}{p}\PYG{l+s+sc}{\PYGZsq{}}\PYG{p}{)}\PYG{+w}{ }\PYG{p}{\PYGZob{}}
\PYG{+w}{        }\PYG{n}{printptr}\PYG{p}{(}\PYG{n}{fd}\PYG{p}{,}\PYG{+w}{ }\PYG{n}{va\PYGZus{}arg}\PYG{p}{(}\PYG{n}{ap}\PYG{p}{,}\PYG{+w}{ }\PYG{n}{uint64}\PYG{p}{)}\PYG{p}{)}\PYG{p}{;}
\PYG{+w}{      }\PYG{p}{\PYGZcb{}}\PYG{+w}{ }\PYG{k}{else}\PYG{+w}{ }\PYG{k}{if}\PYG{p}{(}\PYG{n}{c}\PYG{+w}{ }\PYG{o}{=}\PYG{o}{=}\PYG{+w}{ }\PYG{l+s+sc}{\PYGZsq{}}\PYG{l+s+sc}{s}\PYG{l+s+sc}{\PYGZsq{}}\PYG{p}{)}\PYG{p}{\PYGZob{}}
\PYG{+w}{        }\PYG{n}{s}\PYG{+w}{ }\PYG{o}{=}\PYG{+w}{ }\PYG{n}{va\PYGZus{}arg}\PYG{p}{(}\PYG{n}{ap}\PYG{p}{,}\PYG{+w}{ }\PYG{k+kt}{char}\PYG{o}{*}\PYG{p}{)}\PYG{p}{;}
\PYG{+w}{        }\PYG{k}{if}\PYG{p}{(}\PYG{n}{s}\PYG{+w}{ }\PYG{o}{=}\PYG{o}{=}\PYG{+w}{ }\PYG{l+m+mi}{0}\PYG{p}{)}
\PYG{+w}{          }\PYG{n}{s}\PYG{+w}{ }\PYG{o}{=}\PYG{+w}{ }\PYG{l+s}{\PYGZdq{}}\PYG{l+s}{(null)}\PYG{l+s}{\PYGZdq{}}\PYG{p}{;}
\PYG{+w}{        }\PYG{k}{while}\PYG{p}{(}\PYG{o}{*}\PYG{n}{s}\PYG{+w}{ }\PYG{o}{!}\PYG{o}{=}\PYG{+w}{ }\PYG{l+m+mi}{0}\PYG{p}{)}\PYG{p}{\PYGZob{}}
\PYG{+w}{          }\PYG{n}{putc}\PYG{p}{(}\PYG{n}{fd}\PYG{p}{,}\PYG{+w}{ }\PYG{o}{*}\PYG{n}{s}\PYG{p}{)}\PYG{p}{;}
\PYG{+w}{          }\PYG{n}{s}\PYG{o}{+}\PYG{o}{+}\PYG{p}{;}
\PYG{+w}{        }\PYG{p}{\PYGZcb{}}
\PYG{+w}{      }\PYG{p}{\PYGZcb{}}\PYG{+w}{ }\PYG{k}{else}\PYG{+w}{ }\PYG{k}{if}\PYG{p}{(}\PYG{n}{c}\PYG{+w}{ }\PYG{o}{=}\PYG{o}{=}\PYG{+w}{ }\PYG{l+s+sc}{\PYGZsq{}}\PYG{l+s+sc}{c}\PYG{l+s+sc}{\PYGZsq{}}\PYG{p}{)}\PYG{p}{\PYGZob{}}
\PYG{+w}{        }\PYG{n}{putc}\PYG{p}{(}\PYG{n}{fd}\PYG{p}{,}\PYG{+w}{ }\PYG{n}{va\PYGZus{}arg}\PYG{p}{(}\PYG{n}{ap}\PYG{p}{,}\PYG{+w}{ }\PYG{n}{uint}\PYG{p}{)}\PYG{p}{)}\PYG{p}{;}
\PYG{+w}{      }\PYG{p}{\PYGZcb{}}\PYG{+w}{ }\PYG{k}{else}\PYG{+w}{ }\PYG{k}{if}\PYG{p}{(}\PYG{n}{c}\PYG{+w}{ }\PYG{o}{=}\PYG{o}{=}\PYG{+w}{ }\PYG{l+s+sc}{\PYGZsq{}}\PYG{l+s+sc}{\PYGZpc{}}\PYG{l+s+sc}{\PYGZsq{}}\PYG{p}{)}\PYG{p}{\PYGZob{}}
\PYG{+w}{        }\PYG{n}{putc}\PYG{p}{(}\PYG{n}{fd}\PYG{p}{,}\PYG{+w}{ }\PYG{n}{c}\PYG{p}{)}\PYG{p}{;}
\PYG{+w}{      }\PYG{p}{\PYGZcb{}}\PYG{+w}{ }\PYG{k}{else}\PYG{+w}{ }\PYG{p}{\PYGZob{}}
\PYG{+w}{        }\PYG{c+c1}{// Unknown \PYGZpc{} sequence.  Print it to draw attention.}
\PYG{+w}{        }\PYG{n}{putc}\PYG{p}{(}\PYG{n}{fd}\PYG{p}{,}\PYG{+w}{ }\PYG{l+s+sc}{\PYGZsq{}}\PYG{l+s+sc}{\PYGZpc{}}\PYG{l+s+sc}{\PYGZsq{}}\PYG{p}{)}\PYG{p}{;}
\PYG{+w}{        }\PYG{n}{putc}\PYG{p}{(}\PYG{n}{fd}\PYG{p}{,}\PYG{+w}{ }\PYG{n}{c}\PYG{p}{)}\PYG{p}{;}
\PYG{+w}{      }\PYG{p}{\PYGZcb{}}
\PYG{+w}{      }\PYG{n}{state}\PYG{+w}{ }\PYG{o}{=}\PYG{+w}{ }\PYG{l+m+mi}{0}\PYG{p}{;}
\PYG{+w}{    }\PYG{p}{\PYGZcb{}}
\PYG{+w}{  }\PYG{p}{\PYGZcb{}}
\PYG{p}{\PYGZcb{}}
\end{sphinxVerbatim}

\sphinxAtStartPar
分为state==0和state==\%两种情况,当state==0时,直接输出,当state==\%时对下一个字符进行判断,分别是\%d \sphinxhyphen{} 十进制有符号整数,\%l \sphinxhyphen{} 十进制无符号长整数,\%x \sphinxhyphen{} 十六进制无符号整数,\%p \sphinxhyphen{} 指针地址,\%s \sphinxhyphen{} 字符串,\%c \sphinxhyphen{} 字符,\%\% \sphinxhyphen{} 百分号字符,va\_arg(ap,type),得到当前ap所指向的可变参数的值,并将ap指向下一个可变参数。


\paragraph{对于整数调用printint函数:}
\label{\detokenize{nju:printint}}
\begin{sphinxVerbatim}[commandchars=\\\{\}]
\PYG{k}{static}\PYG{+w}{ }\PYG{k+kt}{void}
\PYG{n+nf}{printint}\PYG{p}{(}\PYG{k+kt}{int}\PYG{+w}{ }\PYG{n}{fd}\PYG{p}{,}\PYG{+w}{ }\PYG{k+kt}{int}\PYG{+w}{ }\PYG{n}{xx}\PYG{p}{,}\PYG{+w}{ }\PYG{k+kt}{int}\PYG{+w}{ }\PYG{n}{base}\PYG{p}{,}\PYG{+w}{ }\PYG{k+kt}{int}\PYG{+w}{ }\PYG{n}{sgn}\PYG{p}{)}
\PYG{p}{\PYGZob{}}
\PYG{+w}{  }\PYG{k+kt}{char}\PYG{+w}{ }\PYG{n}{buf}\PYG{p}{[}\PYG{l+m+mi}{16}\PYG{p}{]}\PYG{p}{;}
\PYG{+w}{  }\PYG{k+kt}{int}\PYG{+w}{ }\PYG{n}{i}\PYG{p}{,}\PYG{+w}{ }\PYG{n}{neg}\PYG{p}{;}
\PYG{+w}{  }\PYG{n}{uint}\PYG{+w}{ }\PYG{n}{x}\PYG{p}{;}

\PYG{+w}{  }\PYG{n}{neg}\PYG{+w}{ }\PYG{o}{=}\PYG{+w}{ }\PYG{l+m+mi}{0}\PYG{p}{;}
\PYG{+w}{  }\PYG{k}{if}\PYG{p}{(}\PYG{n}{sgn}\PYG{+w}{ }\PYG{o}{\PYGZam{}}\PYG{o}{\PYGZam{}}\PYG{+w}{ }\PYG{n}{xx}\PYG{+w}{ }\PYG{o}{\PYGZlt{}}\PYG{+w}{ }\PYG{l+m+mi}{0}\PYG{p}{)}\PYG{p}{\PYGZob{}}
\PYG{+w}{    }\PYG{n}{neg}\PYG{+w}{ }\PYG{o}{=}\PYG{+w}{ }\PYG{l+m+mi}{1}\PYG{p}{;}
\PYG{+w}{    }\PYG{n}{x}\PYG{+w}{ }\PYG{o}{=}\PYG{+w}{ }\PYG{o}{\PYGZhy{}}\PYG{n}{xx}\PYG{p}{;}
\PYG{+w}{  }\PYG{p}{\PYGZcb{}}\PYG{+w}{ }\PYG{k}{else}\PYG{+w}{ }\PYG{p}{\PYGZob{}}
\PYG{+w}{    }\PYG{n}{x}\PYG{+w}{ }\PYG{o}{=}\PYG{+w}{ }\PYG{n}{xx}\PYG{p}{;}
\PYG{+w}{  }\PYG{p}{\PYGZcb{}}

\PYG{+w}{  }\PYG{n}{i}\PYG{+w}{ }\PYG{o}{=}\PYG{+w}{ }\PYG{l+m+mi}{0}\PYG{p}{;}
\PYG{+w}{  }\PYG{k}{do}\PYG{p}{\PYGZob{}}
\PYG{+w}{    }\PYG{n}{buf}\PYG{p}{[}\PYG{n}{i}\PYG{o}{+}\PYG{o}{+}\PYG{p}{]}\PYG{+w}{ }\PYG{o}{=}\PYG{+w}{ }\PYG{n}{digits}\PYG{p}{[}\PYG{n}{x}\PYG{+w}{ }\PYG{o}{\PYGZpc{}}\PYG{+w}{ }\PYG{n}{base}\PYG{p}{]}\PYG{p}{;}
\PYG{+w}{  }\PYG{p}{\PYGZcb{}}\PYG{k}{while}\PYG{p}{(}\PYG{p}{(}\PYG{n}{x}\PYG{+w}{ }\PYG{o}{/}\PYG{o}{=}\PYG{+w}{ }\PYG{n}{base}\PYG{p}{)}\PYG{+w}{ }\PYG{o}{!}\PYG{o}{=}\PYG{+w}{ }\PYG{l+m+mi}{0}\PYG{p}{)}\PYG{p}{;}
\PYG{+w}{  }\PYG{k}{if}\PYG{p}{(}\PYG{n}{neg}\PYG{p}{)}
\PYG{+w}{    }\PYG{n}{buf}\PYG{p}{[}\PYG{n}{i}\PYG{o}{+}\PYG{o}{+}\PYG{p}{]}\PYG{+w}{ }\PYG{o}{=}\PYG{+w}{ }\PYG{l+s+sc}{\PYGZsq{}}\PYG{l+s+sc}{\PYGZhy{}}\PYG{l+s+sc}{\PYGZsq{}}\PYG{p}{;}

\PYG{+w}{  }\PYG{k}{while}\PYG{p}{(}\PYG{o}{\PYGZhy{}}\PYG{o}{\PYGZhy{}}\PYG{n}{i}\PYG{+w}{ }\PYG{o}{\PYGZgt{}}\PYG{o}{=}\PYG{+w}{ }\PYG{l+m+mi}{0}\PYG{p}{)}
\PYG{+w}{    }\PYG{n}{putc}\PYG{p}{(}\PYG{n}{fd}\PYG{p}{,}\PYG{+w}{ }\PYG{n}{buf}\PYG{p}{[}\PYG{n}{i}\PYG{p}{]}\PYG{p}{)}\PYG{p}{;}
\PYG{p}{\PYGZcb{}}
\end{sphinxVerbatim}

\sphinxAtStartPar
base表示进制基数,sgn表示是否处理符号(1=有符号,0=无符号)、首先进行符号判断,如果是有符号,且数字<0,先按照正数处理,最后再加“\sphinxhyphen{}”。定义数组static char digits{[}{]} = “0123456789ABCDEF”;逆序将每个字符放入数组,再逆序输出。


\paragraph{对于指针地址调用printptr函数:}
\label{\detokenize{nju:printptr}}
\begin{sphinxVerbatim}[commandchars=\\\{\}]
\PYG{k}{static}\PYG{+w}{ }\PYG{k+kt}{void}
\PYG{n+nf}{printptr}\PYG{p}{(}\PYG{k+kt}{int}\PYG{+w}{ }\PYG{n}{fd}\PYG{p}{,}\PYG{+w}{ }\PYG{n}{uint64}\PYG{+w}{ }\PYG{n}{x}\PYG{p}{)}\PYG{+w}{ }\PYG{p}{\PYGZob{}}
\PYG{+w}{  }\PYG{k+kt}{int}\PYG{+w}{ }\PYG{n}{i}\PYG{p}{;}
\PYG{+w}{  }\PYG{n}{putc}\PYG{p}{(}\PYG{n}{fd}\PYG{p}{,}\PYG{+w}{ }\PYG{l+s+sc}{\PYGZsq{}}\PYG{l+s+sc}{0}\PYG{l+s+sc}{\PYGZsq{}}\PYG{p}{)}\PYG{p}{;}
\PYG{+w}{  }\PYG{n}{putc}\PYG{p}{(}\PYG{n}{fd}\PYG{p}{,}\PYG{+w}{ }\PYG{l+s+sc}{\PYGZsq{}}\PYG{l+s+sc}{x}\PYG{l+s+sc}{\PYGZsq{}}\PYG{p}{)}\PYG{p}{;}
\PYG{+w}{  }\PYG{k}{for}\PYG{+w}{ }\PYG{p}{(}\PYG{n}{i}\PYG{+w}{ }\PYG{o}{=}\PYG{+w}{ }\PYG{l+m+mi}{0}\PYG{p}{;}\PYG{+w}{ }\PYG{n}{i}\PYG{+w}{ }\PYG{o}{\PYGZlt{}}\PYG{+w}{ }\PYG{p}{(}\PYG{k}{sizeof}\PYG{p}{(}\PYG{n}{uint64}\PYG{p}{)}\PYG{+w}{ }\PYG{o}{*}\PYG{+w}{ }\PYG{l+m+mi}{2}\PYG{p}{)}\PYG{p}{;}\PYG{+w}{ }\PYG{n}{i}\PYG{o}{+}\PYG{o}{+}\PYG{p}{,}\PYG{+w}{ }\PYG{n}{x}\PYG{+w}{ }\PYG{o}{\PYGZlt{}}\PYG{o}{\PYGZlt{}}\PYG{o}{=}\PYG{+w}{ }\PYG{l+m+mi}{4}\PYG{p}{)}
\PYG{+w}{    }\PYG{n}{putc}\PYG{p}{(}\PYG{n}{fd}\PYG{p}{,}\PYG{+w}{ }\PYG{n}{digits}\PYG{p}{[}\PYG{n}{x}\PYG{+w}{ }\PYG{o}{\PYGZgt{}}\PYG{o}{\PYGZgt{}}\PYG{+w}{ }\PYG{p}{(}\PYG{k}{sizeof}\PYG{p}{(}\PYG{n}{uint64}\PYG{p}{)}\PYG{+w}{ }\PYG{o}{*}\PYG{+w}{ }\PYG{l+m+mi}{8}\PYG{+w}{ }\PYG{o}{\PYGZhy{}}\PYG{+w}{ }\PYG{l+m+mi}{4}\PYG{p}{)}\PYG{p}{]}\PYG{p}{)}\PYG{p}{;}
\PYG{p}{\PYGZcb{}}
\end{sphinxVerbatim}


\paragraph{对于字符或者字符串使用va\_arg(),然后putc输出。}
\label{\detokenize{nju:va-arg-putc}}
\sphinxAtStartPar
putc函数:

\begin{sphinxVerbatim}[commandchars=\\\{\}]
\PYG{k}{static}\PYG{+w}{ }\PYG{k+kt}{void}
\PYG{n+nf}{putc}\PYG{p}{(}\PYG{k+kt}{int}\PYG{+w}{ }\PYG{n}{fd}\PYG{p}{,}\PYG{+w}{ }\PYG{k+kt}{char}\PYG{+w}{ }\PYG{n}{c}\PYG{p}{)}
\PYG{p}{\PYGZob{}}
\PYG{+w}{  }\PYG{n}{write}\PYG{p}{(}\PYG{n}{fd}\PYG{p}{,}\PYG{+w}{ }\PYG{o}{\PYGZam{}}\PYG{n}{c}\PYG{p}{,}\PYG{+w}{ }\PYG{l+m+mi}{1}\PYG{p}{)}\PYG{p}{;}
\PYG{p}{\PYGZcb{}}
\end{sphinxVerbatim}

\sphinxAtStartPar
当调用write时

\begin{sphinxVerbatim}[commandchars=\\\{\}]
.global write
write:
li.d \PYGZdl{}a7, SYS\PYGZus{}write
syscall 0
jirl \PYGZdl{}zero, \PYGZdl{}ra, 0
\end{sphinxVerbatim}

\sphinxAtStartPar
向a7寄存器写入SYS\_write,之后硬件保存 PC+4 到 CSR\_ERA,保存当前状态到 CSR\_PRMD,切换到内核特权级(PLV=0),禁用中断,跳转到 CSR\_EENTRY 指定的地址(我们设置的是 w\_csr\_eentry((uint64)uservec))。进入usertrap(),进入syscall()函数。

\begin{sphinxVerbatim}[commandchars=\\\{\}]
\PYG{k+kt}{void}
\PYG{n+nf}{syscall}\PYG{p}{(}\PYG{k+kt}{void}\PYG{p}{)}
\PYG{p}{\PYGZob{}}
\PYG{+w}{  }\PYG{k+kt}{int}\PYG{+w}{ }\PYG{n}{num}\PYG{p}{;}
\PYG{+w}{  }\PYG{k}{struct}\PYG{+w}{ }\PYG{n+nc}{proc}\PYG{+w}{ }\PYG{o}{*}\PYG{n}{p}\PYG{+w}{ }\PYG{o}{=}\PYG{+w}{ }\PYG{n}{myproc}\PYG{p}{(}\PYG{p}{)}\PYG{p}{;}

\PYG{+w}{  }\PYG{n}{num}\PYG{+w}{ }\PYG{o}{=}\PYG{+w}{ }\PYG{n}{p}\PYG{o}{\PYGZhy{}}\PYG{o}{\PYGZgt{}}\PYG{n}{trapframe}\PYG{o}{\PYGZhy{}}\PYG{o}{\PYGZgt{}}\PYG{n}{a7}\PYG{p}{;}
\PYG{+w}{  }\PYG{k}{if}\PYG{p}{(}\PYG{n}{num}\PYG{+w}{ }\PYG{o}{\PYGZgt{}}\PYG{+w}{ }\PYG{l+m+mi}{0}\PYG{+w}{ }\PYG{o}{\PYGZam{}}\PYG{o}{\PYGZam{}}\PYG{+w}{ }\PYG{n}{num}\PYG{+w}{ }\PYG{o}{\PYGZlt{}}\PYG{+w}{ }\PYG{n}{NELEM}\PYG{p}{(}\PYG{n}{syscalls}\PYG{p}{)}\PYG{+w}{ }\PYG{o}{\PYGZam{}}\PYG{o}{\PYGZam{}}\PYG{+w}{ }\PYG{n}{syscalls}\PYG{p}{[}\PYG{n}{num}\PYG{p}{]}\PYG{p}{)}\PYG{+w}{ }\PYG{p}{\PYGZob{}}
\PYG{+w}{    }\PYG{n}{p}\PYG{o}{\PYGZhy{}}\PYG{o}{\PYGZgt{}}\PYG{n}{trapframe}\PYG{o}{\PYGZhy{}}\PYG{o}{\PYGZgt{}}\PYG{n}{a0}\PYG{+w}{ }\PYG{o}{=}\PYG{+w}{ }\PYG{n}{syscalls}\PYG{p}{[}\PYG{n}{num}\PYG{p}{]}\PYG{p}{(}\PYG{p}{)}\PYG{p}{;}
\PYG{+w}{  }\PYG{p}{\PYGZcb{}}\PYG{+w}{ }\PYG{k}{else}\PYG{+w}{ }\PYG{p}{\PYGZob{}}
\PYG{+w}{    }\PYG{n}{printf}\PYG{p}{(}\PYG{l+s}{\PYGZdq{}}\PYG{l+s}{\PYGZpc{}d \PYGZpc{}s: unknown sys call \PYGZpc{}d}\PYG{l+s+se}{\PYGZbs{}n}\PYG{l+s}{\PYGZdq{}}\PYG{p}{,}
\PYG{+w}{            }\PYG{n}{p}\PYG{o}{\PYGZhy{}}\PYG{o}{\PYGZgt{}}\PYG{n}{pid}\PYG{p}{,}\PYG{+w}{ }\PYG{n}{p}\PYG{o}{\PYGZhy{}}\PYG{o}{\PYGZgt{}}\PYG{n}{name}\PYG{p}{,}\PYG{+w}{ }\PYG{n}{num}\PYG{p}{)}\PYG{p}{;}
\PYG{+w}{    }\PYG{n}{p}\PYG{o}{\PYGZhy{}}\PYG{o}{\PYGZgt{}}\PYG{n}{trapframe}\PYG{o}{\PYGZhy{}}\PYG{o}{\PYGZgt{}}\PYG{n}{a0}\PYG{+w}{ }\PYG{o}{=}\PYG{+w}{ }\PYG{l+m+mi}{\PYGZhy{}1}\PYG{p}{;}
\PYG{+w}{  }\PYG{p}{\PYGZcb{}}
\PYG{p}{\PYGZcb{}}
\end{sphinxVerbatim}

\sphinxAtStartPar
通过p\sphinxhyphen{}>trapframe\sphinxhyphen{}>a0 = syscalls\DUrole{xref,myst}{num}进入sys\_write函数。

\begin{sphinxVerbatim}[commandchars=\\\{\}]
\PYG{n}{uint64}
\PYG{n+nf}{sys\PYGZus{}write}\PYG{p}{(}\PYG{k+kt}{void}\PYG{p}{)}
\PYG{p}{\PYGZob{}}
\PYG{+w}{  }\PYG{k}{struct}\PYG{+w}{ }\PYG{n+nc}{file}\PYG{+w}{ }\PYG{o}{*}\PYG{n}{f}\PYG{p}{;}
\PYG{+w}{  }\PYG{k+kt}{int}\PYG{+w}{ }\PYG{n}{n}\PYG{p}{;}
\PYG{+w}{  }\PYG{n}{uint64}\PYG{+w}{ }\PYG{n}{p}\PYG{p}{;}

\PYG{+w}{  }\PYG{k}{if}\PYG{p}{(}\PYG{n}{argfd}\PYG{p}{(}\PYG{l+m+mi}{0}\PYG{p}{,}\PYG{+w}{ }\PYG{l+m+mi}{0}\PYG{p}{,}\PYG{+w}{ }\PYG{o}{\PYGZam{}}\PYG{n}{f}\PYG{p}{)}\PYG{+w}{ }\PYG{o}{\PYGZlt{}}\PYG{+w}{ }\PYG{l+m+mi}{0}\PYG{+w}{ }\PYG{o}{|}\PYG{o}{|}\PYG{+w}{ }\PYG{n}{argint}\PYG{p}{(}\PYG{l+m+mi}{2}\PYG{p}{,}\PYG{+w}{ }\PYG{o}{\PYGZam{}}\PYG{n}{n}\PYG{p}{)}\PYG{+w}{ }\PYG{o}{\PYGZlt{}}\PYG{+w}{ }\PYG{l+m+mi}{0}\PYG{+w}{ }\PYG{o}{|}\PYG{o}{|}\PYG{+w}{ }\PYG{n}{argaddr}\PYG{p}{(}\PYG{l+m+mi}{1}\PYG{p}{,}\PYG{+w}{ }\PYG{o}{\PYGZam{}}\PYG{n}{p}\PYG{p}{)}\PYG{+w}{ }\PYG{o}{\PYGZlt{}}\PYG{+w}{ }\PYG{l+m+mi}{0}\PYG{p}{)}
\PYG{+w}{    }\PYG{k}{return}\PYG{+w}{ }\PYG{l+m+mi}{\PYGZhy{}1}\PYG{p}{;}
\PYG{+w}{  }\PYG{k}{return}\PYG{+w}{ }\PYG{n}{filewrite}\PYG{p}{(}\PYG{n}{f}\PYG{p}{,}\PYG{+w}{ }\PYG{n}{p}\PYG{p}{,}\PYG{+w}{ }\PYG{n}{n}\PYG{p}{)}\PYG{p}{;}
\PYG{p}{\PYGZcb{}}
\end{sphinxVerbatim}

\sphinxAtStartPar
通过if(argfd(0, 0, \&f) < 0 || argint(2, \&n) < 0 || argaddr(1, \&p) < 0)获取参数,包括fd、字节数、用户地址,传入 filewrite函数。因为打开的是设备文件,所以ret = devsw{[}f\sphinxhyphen{}>major{]}.write(1, addr, n);
在console.c中的consoleinit中devsw{[}CONSOLE{]}.write = consolewrite;所以进入consolewrite函数:

\begin{sphinxVerbatim}[commandchars=\\\{\}]
\PYG{k+kt}{int}
\PYG{n+nf}{consolewrite}\PYG{p}{(}\PYG{k+kt}{int}\PYG{+w}{ }\PYG{n}{user\PYGZus{}src}\PYG{p}{,}\PYG{+w}{ }\PYG{n}{uint64}\PYG{+w}{ }\PYG{n}{src}\PYG{p}{,}\PYG{+w}{ }\PYG{k+kt}{int}\PYG{+w}{ }\PYG{n}{n}\PYG{p}{)}
\PYG{p}{\PYGZob{}}
\PYG{+w}{  }\PYG{k+kt}{int}\PYG{+w}{ }\PYG{n}{i}\PYG{p}{;}

\PYG{+w}{  }\PYG{k}{for}\PYG{p}{(}\PYG{n}{i}\PYG{+w}{ }\PYG{o}{=}\PYG{+w}{ }\PYG{l+m+mi}{0}\PYG{p}{;}\PYG{+w}{ }\PYG{n}{i}\PYG{+w}{ }\PYG{o}{\PYGZlt{}}\PYG{+w}{ }\PYG{n}{n}\PYG{p}{;}\PYG{+w}{ }\PYG{n}{i}\PYG{o}{+}\PYG{o}{+}\PYG{p}{)}\PYG{p}{\PYGZob{}}
\PYG{+w}{    }\PYG{k+kt}{char}\PYG{+w}{ }\PYG{n}{c}\PYG{p}{;}
\PYG{+w}{    }\PYG{k}{if}\PYG{p}{(}\PYG{n}{either\PYGZus{}copyin}\PYG{p}{(}\PYG{o}{\PYGZam{}}\PYG{n}{c}\PYG{p}{,}\PYG{+w}{ }\PYG{n}{user\PYGZus{}src}\PYG{p}{,}\PYG{+w}{ }\PYG{n}{src}\PYG{o}{+}\PYG{n}{i}\PYG{p}{,}\PYG{+w}{ }\PYG{l+m+mi}{1}\PYG{p}{)}\PYG{+w}{ }\PYG{o}{=}\PYG{o}{=}\PYG{+w}{ }\PYG{l+m+mi}{\PYGZhy{}1}\PYG{p}{)}
\PYG{+w}{      }\PYG{k}{break}\PYG{p}{;}
\PYG{+w}{    }\PYG{n}{uartputc}\PYG{p}{(}\PYG{n}{c}\PYG{p}{)}\PYG{p}{;}
\PYG{+w}{  }\PYG{p}{\PYGZcb{}}

\PYG{+w}{  }\PYG{k}{return}\PYG{+w}{ }\PYG{n}{i}\PYG{p}{;}
\PYG{p}{\PYGZcb{}}
\end{sphinxVerbatim}

\sphinxAtStartPar
uartputc\sphinxhyphen{}>uartstart\sphinxhyphen{}>WriteReg,qemu通过读取寄存器将内容打印到控制台。
scanf调用的是read逻辑与write相同不做描述。


\subsubsection{main}
\label{\detokenize{nju:main}}
\begin{sphinxVerbatim}[commandchars=\\\{\}]
\PYG{k+kt}{int}
\PYG{n+nf}{main}\PYG{p}{(}\PYG{k+kt}{void}\PYG{p}{)}
\PYG{p}{\PYGZob{}}
\PYG{+w}{    }\PYG{k}{if}\PYG{p}{(}\PYG{n}{open}\PYG{p}{(}\PYG{l+s}{\PYGZdq{}}\PYG{l+s}{console}\PYG{l+s}{\PYGZdq{}}\PYG{p}{,}\PYG{+w}{ }\PYG{n}{O\PYGZus{}RDWR}\PYG{p}{)}\PYG{+w}{ }\PYG{o}{\PYGZlt{}}\PYG{+w}{ }\PYG{l+m+mi}{0}\PYG{p}{)}\PYG{p}{\PYGZob{}}
\PYG{+w}{        }\PYG{n}{mknod}\PYG{p}{(}\PYG{l+s}{\PYGZdq{}}\PYG{l+s}{console}\PYG{l+s}{\PYGZdq{}}\PYG{p}{,}\PYG{+w}{ }\PYG{n}{CONSOLE}\PYG{p}{,}\PYG{+w}{ }\PYG{l+m+mi}{0}\PYG{p}{)}\PYG{p}{;}
\PYG{+w}{        }\PYG{n}{open}\PYG{p}{(}\PYG{l+s}{\PYGZdq{}}\PYG{l+s}{console}\PYG{l+s}{\PYGZdq{}}\PYG{p}{,}\PYG{+w}{ }\PYG{n}{O\PYGZus{}RDWR}\PYG{p}{)}\PYG{p}{;}
\PYG{+w}{    }\PYG{p}{\PYGZcb{}}
\PYG{+w}{    }\PYG{n}{dup}\PYG{p}{(}\PYG{l+m+mi}{0}\PYG{p}{)}\PYG{p}{;}\PYG{+w}{  }\PYG{c+c1}{// stdout}
\PYG{+w}{    }\PYG{n}{dup}\PYG{p}{(}\PYG{l+m+mi}{0}\PYG{p}{)}\PYG{p}{;}\PYG{+w}{  }\PYG{c+c1}{// stderr}
\PYG{c+c1}{//lab2}
\PYG{+w}{    }\PYG{k+kt}{int}\PYG{+w}{ }\PYG{n}{num}\PYG{o}{=}\PYG{l+m+mi}{0}\PYG{p}{;}
\PYG{+w}{    }\PYG{n}{printf}\PYG{p}{(}\PYG{l+s}{\PYGZdq{}}\PYG{l+s}{write a num}\PYG{l+s+se}{\PYGZbs{}n}\PYG{l+s}{\PYGZdq{}}\PYG{p}{)}\PYG{p}{;}
\PYG{+w}{    }\PYG{n}{scanf}\PYG{p}{(}\PYG{l+s}{\PYGZdq{}}\PYG{l+s}{\PYGZpc{}d}\PYG{l+s}{\PYGZdq{}}\PYG{p}{,}\PYG{+w}{ }\PYG{o}{\PYGZam{}}\PYG{n}{num}\PYG{p}{)}\PYG{p}{;}\PYG{+w}{  }
\PYG{+w}{    }\PYG{n}{printf}\PYG{p}{(}\PYG{l+s}{\PYGZdq{}}\PYG{l+s}{what u write num is: \PYGZpc{}d}\PYG{l+s+se}{\PYGZbs{}n}\PYG{l+s}{\PYGZdq{}}\PYG{p}{,}\PYG{+w}{ }\PYG{n}{num}\PYG{p}{)}\PYG{p}{;}
\PYG{+w}{    }\PYG{n}{printf}\PYG{p}{(}\PYG{l+s}{\PYGZdq{}}\PYG{l+s}{lab2 success}\PYG{l+s+se}{\PYGZbs{}n}\PYG{l+s}{\PYGZdq{}}\PYG{p}{)}\PYG{p}{;}



\PYG{+w}{   }
\PYG{+w}{    }\PYG{k}{while}\PYG{p}{(}\PYG{l+m+mi}{1}\PYG{p}{)}\PYG{p}{\PYGZob{}}
\PYG{+w}{    }\PYG{p}{\PYGZcb{}}
\PYG{+w}{    }\PYG{k}{return}\PYG{+w}{ }\PYG{l+m+mi}{0}\PYG{p}{;}
\PYG{p}{\PYGZcb{}}
\end{sphinxVerbatim}

\sphinxAtStartPar
测试printf和scanf函数。


\section{实验结果}
\label{\detokenize{nju:id7}}
\sphinxAtStartPar
\sphinxincludegraphics{{img0}.png}


\chapter{lab2}
\label{\detokenize{nju:lab2}}

\section{实验目的}
\label{\detokenize{nju:id8}}\begin{enumerate}
\sphinxsetlistlabels{\arabic}{enumi}{enumii}{}{.}%
\item {} 
\sphinxAtStartPar
实现一个简单的任务调度。

\item {} 
\sphinxAtStartPar
介绍基于时间中断进行进程切换以及纯用户态的非抢占式的线程切换完成任务调度的全过程。

\end{enumerate}


\section{实验内容}
\label{\detokenize{nju:id9}}\begin{enumerate}
\sphinxsetlistlabels{\arabic}{enumi}{enumii}{}{.}%
\item {} 
\sphinxAtStartPar
内核:实现进程切换机制,并提供系统调用 fork、sleep、exit。

\item {} 
\sphinxAtStartPar
库:对上述系统调用进行封装;实现一个用户态的线程库,完成 pthread\_create、
pthread\_join、pthread\_yield、pthread\_exit 等接口。

\item {} 
\sphinxAtStartPar
用户:对上述库函数进行测试。

\end{enumerate}


\section{背景知识}
\label{\detokenize{nju:id10}}

\subsection{进程与线程}
\label{\detokenize{nju:id11}}
\sphinxAtStartPar
进程为操作系统资源分配的单位,每个进程都有独立的地址空间(代码段、数据段),独立的堆栈,独立的进程控制块;线程作为任务调度的基本单位,与进程的唯一区别在于其地址空间并非独立,而是与其他线程共享(创建这些线程的“进程”的地址空间)。以下为一个广义的进程(包括进程与线程)生命周期中的状态转换图。\\
\sphinxincludegraphics{{img2}.png}

\sphinxAtStartPar
(1) 进程由其父进程利用 FORK 系统调用创建,则该进程进入 RUNNABLE 状态。\\
(2) 时间中断到来,RUNNABLE 状态的进程被切换到,则该进程进入 RUNNING 状态。
(3) 时 间 中断 到来,RUNNING 状 态的进程 处理时间 片耗尽, 则该进程 进入
RUNNABLE 状态。\\
(4) RUNNING 状态的进程利用 SLEEP 系统调用主动阻塞;或利用系统调用等待硬
件 I/O,则该进程进入 BLOCKED 状态。\\
(5) 时间中断到来,BLOCKED 状态的进程的 SLEEP 时间片耗尽;或外部硬件中断
表明 I/O 完成,则该进程进入 RUNNABLE 状态。\\
(6) RUNNING 状态的进程利用 EXIT 系统调用主动销毁,则该进程进入 DEAD 状态。


\subsection{进程切换与堆栈切换}
\label{\detokenize{nju:id12}}

\begin{savenotes}\sphinxattablestart
\sphinxthistablewithglobalstyle
\centering
\sphinxcapstartof{table}
\sphinxthecaptionisattop
\sphinxcaption{时间中断与进程切换流程详解}\label{\detokenize{nju:id32}}
\sphinxaftertopcaption
\begin{tabular}[t]{|\X{25}{100}|\X{75}{100}|}
\sphinxtoprule
\sphinxstyletheadfamily 
\sphinxAtStartPar
\sphinxstylestrong{阶段}
&\sphinxstyletheadfamily 
\sphinxAtStartPar
\sphinxstylestrong{详细动作与状态}
\\
\sphinxmidrule
\sphinxtableatstartofbodyhook
\sphinxAtStartPar
时间中断发生前
&
\sphinxAtStartPar
P1 用户态执行
堆栈:P1 用户栈
时间片:还剩 5ms
\\
\sphinxhline
\sphinxAtStartPar
时间中断发生!
\sphinxstyleemphasis{(硬件自动完成)}
&\begin{enumerate}
\sphinxsetlistlabels{\arabic}{enumi}{enumii}{}{.}%
\item {} 
\sphinxAtStartPar
保存 PC 到 EPC/ERA 寄存器

\item {} 
\sphinxAtStartPar
切换特权级到内核态

\item {} 
\sphinxAtStartPar
跳转到异常向量地址

\end{enumerate}
\\
\sphinxhline
\sphinxAtStartPar
进入异常向量
\sphinxstyleemphasis{(汇编代码)}
&\begin{enumerate}
\sphinxsetlistlabels{\arabic}{enumi}{enumii}{}{.}%
\item {} 
\sphinxAtStartPar
立即设置堆栈为 P1 内核栈!

\item {} 
\sphinxAtStartPar
保存所有用户寄存器到内核栈

\item {} 
\sphinxAtStartPar
调用 C 语言中断处理程序

\end{enumerate}
\\
\sphinxhline
\sphinxAtStartPar
C 语言中断处理程序
&\begin{enumerate}
\sphinxsetlistlabels{\arabic}{enumi}{enumii}{}{.}%
\item {} 
\sphinxAtStartPar
更新时间统计

\item {} 
\sphinxAtStartPar
减少时间片计数:\sphinxcode{\sphinxupquote{P1.time\_slice\sphinxhyphen{}\sphinxhyphen{}}}

\item {} 
\sphinxAtStartPar
检查时间片是否耗尽
\sphinxcode{\sphinxupquote{if (P1.time\_slice <= 0) \{ 设置重新调度标志 \}}}

\item {} 
\sphinxAtStartPar
如果需要,触发调度

\end{enumerate}
\\
\sphinxhline
\sphinxAtStartPar
如果需要进程切换
&\begin{enumerate}
\sphinxsetlistlabels{\arabic}{enumi}{enumii}{}{.}%
\item {} 
\sphinxAtStartPar
保存 P1 内核上下文到 \sphinxcode{\sphinxupquote{task\_struct}}

\item {} 
\sphinxAtStartPar
选择 P2 为下一个进程

\item {} 
\sphinxAtStartPar
切换到 P2 的内核栈

\item {} 
\sphinxAtStartPar
从 P2 内核栈恢复 P2 的上下文

\end{enumerate}
\\
\sphinxhline
\sphinxAtStartPar
中断返回
&\begin{enumerate}
\sphinxsetlistlabels{\arabic}{enumi}{enumii}{}{.}%
\item {} 
\sphinxAtStartPar
从内核栈恢复用户寄存器

\item {} 
\sphinxAtStartPar
返回到 P2 用户态执行
堆栈:P2 用户栈

\end{enumerate}
\\
\sphinxbottomrule
\end{tabular}
\sphinxtableafterendhook\par
\sphinxattableend\end{savenotes}

\sphinxAtStartPar
(1) 进程 P1 在用户态执行,中断发生后,硬件自动操作:
\begin{enumerate}
\sphinxsetlistlabels{\arabic}{enumi}{enumii}{}{.}%
\item {} 
\sphinxAtStartPar
保存PC到CSR\_ERA。

\item {} 
\sphinxAtStartPar
设置CRMD.PLV=0 (进入内核态)。

\item {} 
\sphinxAtStartPar
跳转到异常向量地址。\\
在异常向量中,从 P1 的用户态堆栈切换至 P1 的内核堆栈,并将 P1 的现场信息压入内核堆栈中,跳转执行时间中断处理程序,等p1的时间片用完,调度器切换到p2。\\
(2) 进程 P1 的处理时间片耗尽,切换至就绪状态的进程 P2,并从当前 P1 的内核
堆栈切换至 P2 的内核堆栈。\\
(3) 从进程 P2 的内核堆栈中弹出 P2 的现场信息,切换至 P2 的用户态堆栈,从时
间中断处理程序返回执行 P2。

\end{enumerate}


\section{实验过程}
\label{\detokenize{nju:id13}}

\subsection{进程}
\label{\detokenize{nju:id14}}

\subsubsection{sys\_fork}
\label{\detokenize{nju:sys-fork}}
\begin{sphinxVerbatim}[commandchars=\\\{\}]
\PYG{k+kt}{int}\PYG{+w}{ }\PYG{n+nf}{fork}\PYG{p}{(}\PYG{k+kt}{void}\PYG{p}{)}
\PYG{p}{\PYGZob{}}
\PYG{+w}{  }\PYG{k+kt}{int}\PYG{+w}{ }\PYG{n}{i}\PYG{p}{,}\PYG{+w}{ }\PYG{n}{pid}\PYG{p}{;}
\PYG{+w}{  }\PYG{k}{struct}\PYG{+w}{ }\PYG{n+nc}{proc}\PYG{+w}{ }\PYG{o}{*}\PYG{n}{np}\PYG{p}{;}
\PYG{+w}{  }\PYG{k}{struct}\PYG{+w}{ }\PYG{n+nc}{proc}\PYG{+w}{ }\PYG{o}{*}\PYG{n}{p}\PYG{+w}{ }\PYG{o}{=}\PYG{+w}{ }\PYG{n}{myproc}\PYG{p}{(}\PYG{p}{)}\PYG{p}{;}

\PYG{+w}{  }\PYG{c+c1}{// Allocate process.}
\PYG{+w}{  }\PYG{k}{if}\PYG{+w}{ }\PYG{p}{(}\PYG{p}{(}\PYG{n}{np}\PYG{+w}{ }\PYG{o}{=}\PYG{+w}{ }\PYG{n}{allocproc}\PYG{p}{(}\PYG{p}{)}\PYG{p}{)}\PYG{+w}{ }\PYG{o}{=}\PYG{o}{=}\PYG{+w}{ }\PYG{l+m+mi}{0}\PYG{p}{)}
\PYG{+w}{  }\PYG{p}{\PYGZob{}}
\PYG{+w}{    }\PYG{k}{return}\PYG{+w}{ }\PYG{l+m+mi}{\PYGZhy{}1}\PYG{p}{;}
\PYG{+w}{  }\PYG{p}{\PYGZcb{}}

\PYG{+w}{  }\PYG{c+c1}{// Copy user memory from parent to child.}
\PYG{+w}{  }\PYG{k}{if}\PYG{+w}{ }\PYG{p}{(}\PYG{n}{uvmcopy}\PYG{p}{(}\PYG{n}{p}\PYG{o}{\PYGZhy{}}\PYG{o}{\PYGZgt{}}\PYG{n}{pagetable}\PYG{p}{,}\PYG{+w}{ }\PYG{n}{np}\PYG{o}{\PYGZhy{}}\PYG{o}{\PYGZgt{}}\PYG{n}{pagetable}\PYG{p}{,}\PYG{+w}{ }\PYG{n}{p}\PYG{o}{\PYGZhy{}}\PYG{o}{\PYGZgt{}}\PYG{n}{sz}\PYG{p}{)}\PYG{+w}{ }\PYG{o}{\PYGZlt{}}\PYG{+w}{ }\PYG{l+m+mi}{0}\PYG{p}{)}
\PYG{+w}{  }\PYG{p}{\PYGZob{}}
\PYG{+w}{    }\PYG{n}{freeproc}\PYG{p}{(}\PYG{n}{np}\PYG{p}{)}\PYG{p}{;}
\PYG{+w}{    }\PYG{n}{release}\PYG{p}{(}\PYG{o}{\PYGZam{}}\PYG{n}{np}\PYG{o}{\PYGZhy{}}\PYG{o}{\PYGZgt{}}\PYG{n}{lock}\PYG{p}{)}\PYG{p}{;}
\PYG{+w}{    }\PYG{k}{return}\PYG{+w}{ }\PYG{l+m+mi}{\PYGZhy{}1}\PYG{p}{;}
\PYG{+w}{  }\PYG{p}{\PYGZcb{}}
\PYG{+w}{  }\PYG{n}{np}\PYG{o}{\PYGZhy{}}\PYG{o}{\PYGZgt{}}\PYG{n}{sz}\PYG{+w}{ }\PYG{o}{=}\PYG{+w}{ }\PYG{n}{p}\PYG{o}{\PYGZhy{}}\PYG{o}{\PYGZgt{}}\PYG{n}{sz}\PYG{p}{;}

\PYG{+w}{  }\PYG{c+c1}{// copy saved user registers.}
\PYG{+w}{  }\PYG{o}{*}\PYG{p}{(}\PYG{n}{np}\PYG{o}{\PYGZhy{}}\PYG{o}{\PYGZgt{}}\PYG{n}{trapframe}\PYG{p}{)}\PYG{+w}{ }\PYG{o}{=}\PYG{+w}{ }\PYG{o}{*}\PYG{p}{(}\PYG{n}{p}\PYG{o}{\PYGZhy{}}\PYG{o}{\PYGZgt{}}\PYG{n}{trapframe}\PYG{p}{)}\PYG{p}{;}

\PYG{+w}{  }\PYG{c+c1}{// Cause fork to return 0 in the child.}
\PYG{+w}{  }\PYG{n}{np}\PYG{o}{\PYGZhy{}}\PYG{o}{\PYGZgt{}}\PYG{n}{trapframe}\PYG{o}{\PYGZhy{}}\PYG{o}{\PYGZgt{}}\PYG{n}{a0}\PYG{+w}{ }\PYG{o}{=}\PYG{+w}{ }\PYG{l+m+mi}{0}\PYG{p}{;}

\PYG{+w}{  }\PYG{c+c1}{// increment reference counts on open file descriptors.}
\PYG{+w}{  }\PYG{k}{for}\PYG{+w}{ }\PYG{p}{(}\PYG{n}{i}\PYG{+w}{ }\PYG{o}{=}\PYG{+w}{ }\PYG{l+m+mi}{0}\PYG{p}{;}\PYG{+w}{ }\PYG{n}{i}\PYG{+w}{ }\PYG{o}{\PYGZlt{}}\PYG{+w}{ }\PYG{n}{NOFILE}\PYG{p}{;}\PYG{+w}{ }\PYG{n}{i}\PYG{o}{+}\PYG{o}{+}\PYG{p}{)}
\PYG{+w}{    }\PYG{k}{if}\PYG{+w}{ }\PYG{p}{(}\PYG{n}{p}\PYG{o}{\PYGZhy{}}\PYG{o}{\PYGZgt{}}\PYG{n}{ofile}\PYG{p}{[}\PYG{n}{i}\PYG{p}{]}\PYG{p}{)}
\PYG{+w}{      }\PYG{n}{np}\PYG{o}{\PYGZhy{}}\PYG{o}{\PYGZgt{}}\PYG{n}{ofile}\PYG{p}{[}\PYG{n}{i}\PYG{p}{]}\PYG{+w}{ }\PYG{o}{=}\PYG{+w}{ }\PYG{n}{filedup}\PYG{p}{(}\PYG{n}{p}\PYG{o}{\PYGZhy{}}\PYG{o}{\PYGZgt{}}\PYG{n}{ofile}\PYG{p}{[}\PYG{n}{i}\PYG{p}{]}\PYG{p}{)}\PYG{p}{;}
\PYG{+w}{  }\PYG{n}{np}\PYG{o}{\PYGZhy{}}\PYG{o}{\PYGZgt{}}\PYG{n}{cwd}\PYG{+w}{ }\PYG{o}{=}\PYG{+w}{ }\PYG{n}{idup}\PYG{p}{(}\PYG{n}{p}\PYG{o}{\PYGZhy{}}\PYG{o}{\PYGZgt{}}\PYG{n}{cwd}\PYG{p}{)}\PYG{p}{;}

\PYG{+w}{  }\PYG{n}{safestrcpy}\PYG{p}{(}\PYG{n}{np}\PYG{o}{\PYGZhy{}}\PYG{o}{\PYGZgt{}}\PYG{n}{name}\PYG{p}{,}\PYG{+w}{ }\PYG{n}{p}\PYG{o}{\PYGZhy{}}\PYG{o}{\PYGZgt{}}\PYG{n}{name}\PYG{p}{,}\PYG{+w}{ }\PYG{k}{sizeof}\PYG{p}{(}\PYG{n}{p}\PYG{o}{\PYGZhy{}}\PYG{o}{\PYGZgt{}}\PYG{n}{name}\PYG{p}{)}\PYG{p}{)}\PYG{p}{;}

\PYG{+w}{  }\PYG{n}{pid}\PYG{+w}{ }\PYG{o}{=}\PYG{+w}{ }\PYG{n}{np}\PYG{o}{\PYGZhy{}}\PYG{o}{\PYGZgt{}}\PYG{n}{pid}\PYG{p}{;}

\PYG{+w}{  }\PYG{n}{release}\PYG{p}{(}\PYG{o}{\PYGZam{}}\PYG{n}{np}\PYG{o}{\PYGZhy{}}\PYG{o}{\PYGZgt{}}\PYG{n}{lock}\PYG{p}{)}\PYG{p}{;}

\PYG{+w}{  }\PYG{n}{acquire}\PYG{p}{(}\PYG{o}{\PYGZam{}}\PYG{n}{wait\PYGZus{}lock}\PYG{p}{)}\PYG{p}{;}
\PYG{+w}{  }\PYG{n}{np}\PYG{o}{\PYGZhy{}}\PYG{o}{\PYGZgt{}}\PYG{n}{parent}\PYG{+w}{ }\PYG{o}{=}\PYG{+w}{ }\PYG{n}{p}\PYG{p}{;}
\PYG{+w}{  }\PYG{n}{release}\PYG{p}{(}\PYG{o}{\PYGZam{}}\PYG{n}{wait\PYGZus{}lock}\PYG{p}{)}\PYG{p}{;}

\PYG{+w}{  }\PYG{n}{acquire}\PYG{p}{(}\PYG{o}{\PYGZam{}}\PYG{n}{np}\PYG{o}{\PYGZhy{}}\PYG{o}{\PYGZgt{}}\PYG{n}{lock}\PYG{p}{)}\PYG{p}{;}
\PYG{+w}{  }\PYG{n}{np}\PYG{o}{\PYGZhy{}}\PYG{o}{\PYGZgt{}}\PYG{n}{state}\PYG{+w}{ }\PYG{o}{=}\PYG{+w}{ }\PYG{n}{RUNNABLE}\PYG{p}{;}
\PYG{+w}{  }\PYG{n}{release}\PYG{p}{(}\PYG{o}{\PYGZam{}}\PYG{n}{np}\PYG{o}{\PYGZhy{}}\PYG{o}{\PYGZgt{}}\PYG{n}{lock}\PYG{p}{)}\PYG{p}{;}

\PYG{+w}{  }\PYG{k}{return}\PYG{+w}{ }\PYG{n}{pid}\PYG{p}{;}
\PYG{p}{\PYGZcb{}}
\end{sphinxVerbatim}

\sphinxAtStartPar
fork 系统调用用于创建子进程,内核需要为子进程分配一块独立的内存,将父进程的地址空间、用户态堆栈完全拷贝至子进程的内存中,并为子进程分配独立的进程控制块,完成对子进程的进程控制块的设置。若子进程创建成功,则对于父进程,该系统调用的返回值为子进程的 pid,对于子进程,其返回值为 0;若子进程创建失败,该系统调用的返回值为\sphinxhyphen{}1。将父进程的整个用户地址空间复制到子进程,使用写时复制(Copy\sphinxhyphen{}on\sphinxhyphen{}Write)技术:实际不复制物理内存,只是复制页表条目,并标记为只读,当任一进程尝试写入时,会发生页错误,内核才会真正复制该页,如果复制失败,清理资源并返回错误,复制陷阱帧和设置返回值,遍历父进程的所有文件描述符,对每个打开的文件,调用filedup()增加引用计数,子进程共享相同的文件对象,复制当前工作目录:idup()增加目录inode的引用计数,父子进程共享相同的工作目录。设置进程关系和状态包括id,状态,进程名等。


\subsubsection{sys\_sleep}
\label{\detokenize{nju:sys-sleep}}
\begin{sphinxVerbatim}[commandchars=\\\{\}]
\PYG{n}{uint64}
\PYG{n+nf}{sys\PYGZus{}sleep}\PYG{p}{(}\PYG{k+kt}{void}\PYG{p}{)}
\PYG{p}{\PYGZob{}}
\PYG{+w}{  }\PYG{k+kt}{int}\PYG{+w}{ }\PYG{n}{n}\PYG{p}{;}
\PYG{+w}{  }\PYG{n}{uint}\PYG{+w}{ }\PYG{n}{ticks0}\PYG{p}{;}

\PYG{+w}{  }\PYG{k}{if}\PYG{p}{(}\PYG{n}{argint}\PYG{p}{(}\PYG{l+m+mi}{0}\PYG{p}{,}\PYG{+w}{ }\PYG{o}{\PYGZam{}}\PYG{n}{n}\PYG{p}{)}\PYG{+w}{ }\PYG{o}{\PYGZlt{}}\PYG{+w}{ }\PYG{l+m+mi}{0}\PYG{p}{)}
\PYG{+w}{    }\PYG{k}{return}\PYG{+w}{ }\PYG{l+m+mi}{\PYGZhy{}1}\PYG{p}{;}
\PYG{+w}{  }\PYG{n}{acquire}\PYG{p}{(}\PYG{o}{\PYGZam{}}\PYG{n}{tickslock}\PYG{p}{)}\PYG{p}{;}
\PYG{+w}{  }\PYG{n}{ticks0}\PYG{+w}{ }\PYG{o}{=}\PYG{+w}{ }\PYG{n}{ticks}\PYG{p}{;}
\PYG{+w}{  }\PYG{k}{while}\PYG{p}{(}\PYG{n}{ticks}\PYG{+w}{ }\PYG{o}{\PYGZhy{}}\PYG{+w}{ }\PYG{n}{ticks0}\PYG{+w}{ }\PYG{o}{\PYGZlt{}}\PYG{+w}{ }\PYG{n}{n}\PYG{p}{)}\PYG{p}{\PYGZob{}}
\PYG{+w}{    }\PYG{k}{if}\PYG{p}{(}\PYG{n}{myproc}\PYG{p}{(}\PYG{p}{)}\PYG{o}{\PYGZhy{}}\PYG{o}{\PYGZgt{}}\PYG{n}{killed}\PYG{p}{)}\PYG{p}{\PYGZob{}}
\PYG{+w}{      }\PYG{n}{release}\PYG{p}{(}\PYG{o}{\PYGZam{}}\PYG{n}{tickslock}\PYG{p}{)}\PYG{p}{;}
\PYG{+w}{      }\PYG{k}{return}\PYG{+w}{ }\PYG{l+m+mi}{\PYGZhy{}1}\PYG{p}{;}
\PYG{+w}{    }\PYG{p}{\PYGZcb{}}
\PYG{+w}{    }\PYG{n}{sleep}\PYG{p}{(}\PYG{o}{\PYGZam{}}\PYG{n}{ticks}\PYG{p}{,}\PYG{+w}{ }\PYG{o}{\PYGZam{}}\PYG{n}{tickslock}\PYG{p}{)}\PYG{p}{;}
\PYG{+w}{  }\PYG{p}{\PYGZcb{}}
\PYG{+w}{  }\PYG{n}{release}\PYG{p}{(}\PYG{o}{\PYGZam{}}\PYG{n}{tickslock}\PYG{p}{)}\PYG{p}{;}
\PYG{+w}{  }\PYG{k}{return}\PYG{+w}{ }\PYG{l+m+mi}{0}\PYG{p}{;}
\PYG{p}{\PYGZcb{}}
\end{sphinxVerbatim}

\sphinxAtStartPar
获取用户态需要休眠的时间,获取当前的ticks赋值给ticks0,当ticks \sphinxhyphen{} ticks0 < n时,将当前状态设置为SLEEPING,调用sched,让出cpu。sleep 系统调用用于进程主动阻塞自身,内核需要将该进程由 RUNNING 状态转换为 SLEEPING 状态,设置该进程的 SLEEP 时间片,并切换运行其他 RUNNABLE 状态的进程。


\subsubsection{sys\_exit}
\label{\detokenize{nju:sys-exit}}
\begin{sphinxVerbatim}[commandchars=\\\{\}]
\PYG{n}{uint64}
\PYG{n+nf}{sys\PYGZus{}exit}\PYG{p}{(}\PYG{k+kt}{void}\PYG{p}{)}
\PYG{p}{\PYGZob{}}
\PYG{+w}{  }\PYG{k+kt}{int}\PYG{+w}{ }\PYG{n}{n}\PYG{p}{;}
\PYG{+w}{  }\PYG{k}{if}\PYG{p}{(}\PYG{n}{argint}\PYG{p}{(}\PYG{l+m+mi}{0}\PYG{p}{,}\PYG{+w}{ }\PYG{o}{\PYGZam{}}\PYG{n}{n}\PYG{p}{)}\PYG{+w}{ }\PYG{o}{\PYGZlt{}}\PYG{+w}{ }\PYG{l+m+mi}{0}\PYG{p}{)}
\PYG{+w}{    }\PYG{k}{return}\PYG{+w}{ }\PYG{l+m+mi}{\PYGZhy{}1}\PYG{p}{;}
\PYG{+w}{  }\PYG{n}{exit}\PYG{p}{(}\PYG{n}{n}\PYG{p}{)}\PYG{p}{;}
\PYG{+w}{  }\PYG{k}{return}\PYG{+w}{ }\PYG{l+m+mi}{0}\PYG{p}{;}\PYG{+w}{  }\PYG{c+c1}{// not reached}
\PYG{p}{\PYGZcb{}}
\end{sphinxVerbatim}

\sphinxAtStartPar
EXIT 系统调用用于进程主动销毁自身,内核需要将该进程由 RUNNING 状态转换为 DEAD 状态,回收分配给该进程的内存、进程控制块等资源,并切换运行其他RUNNABLE 状态的进程。


\subsection{线程}
\label{\detokenize{nju:id15}}

\subsubsection{pthread\_create}
\label{\detokenize{nju:pthread-create}}
\begin{sphinxVerbatim}[commandchars=\\\{\}]
\PYG{k+kt}{int}\PYG{+w}{ }\PYG{n+nf}{pthread\PYGZus{}create}\PYG{p}{(}\PYG{n}{uint32}\PYG{+w}{ }\PYG{o}{*}\PYG{k+kr}{thread}\PYG{p}{,}\PYG{+w}{ }\PYG{k}{const}\PYG{+w}{ }\PYG{k+kt}{int}\PYG{+w}{ }\PYG{o}{*}\PYG{n}{attr}\PYG{p}{,}
\PYG{+w}{                   }\PYG{k+kt}{void}\PYG{+w}{ }\PYG{o}{*}\PYG{p}{(}\PYG{o}{*}\PYG{n}{start\PYGZus{}routine}\PYG{p}{)}\PYG{p}{(}\PYG{k+kt}{void}\PYG{+w}{ }\PYG{o}{*}\PYG{p}{)}\PYG{p}{,}\PYG{+w}{ }\PYG{k+kt}{void}\PYG{+w}{ }\PYG{o}{*}\PYG{n}{arg}\PYG{p}{)}\PYG{+w}{ }\PYG{p}{\PYGZob{}}
\PYG{+w}{    }\PYG{k+kt}{int}\PYG{+w}{ }\PYG{n}{i}\PYG{p}{;}
\PYG{+w}{    }\PYG{k}{for}\PYG{+w}{ }\PYG{p}{(}\PYG{n}{i}\PYG{+w}{ }\PYG{o}{=}\PYG{+w}{ }\PYG{l+m+mi}{1}\PYG{p}{;}\PYG{+w}{ }\PYG{n}{i}\PYG{+w}{ }\PYG{o}{\PYGZlt{}}\PYG{+w}{ }\PYG{n}{MAX\PYGZus{}TCB\PYGZus{}NUM}\PYG{p}{;}\PYG{+w}{ }\PYG{o}{+}\PYG{o}{+}\PYG{n}{i}\PYG{p}{)}\PYG{+w}{ }\PYG{p}{\PYGZob{}}
\PYG{+w}{        }\PYG{k}{if}\PYG{+w}{ }\PYG{p}{(}\PYG{n}{tcb}\PYG{p}{[}\PYG{n}{i}\PYG{p}{]}\PYG{p}{.}\PYG{n}{state}\PYG{+w}{ }\PYG{o}{=}\PYG{o}{=}\PYG{+w}{ }\PYG{n}{STATE\PYGZus{}DEAD}\PYG{p}{)}\PYG{+w}{ }\PYG{p}{\PYGZob{}}
\PYG{+w}{            }\PYG{k}{break}\PYG{p}{;}
\PYG{+w}{        }\PYG{p}{\PYGZcb{}}
\PYG{+w}{    }\PYG{p}{\PYGZcb{}}
\PYG{+w}{    }\PYG{k}{if}\PYG{+w}{ }\PYG{p}{(}\PYG{n}{i}\PYG{+w}{ }\PYG{o}{=}\PYG{o}{=}\PYG{+w}{ }\PYG{n}{MAX\PYGZus{}TCB\PYGZus{}NUM}\PYG{p}{)}\PYG{+w}{ }\PYG{p}{\PYGZob{}}
\PYG{+w}{        }\PYG{k}{return}\PYG{+w}{ }\PYG{l+m+mi}{\PYGZhy{}1}\PYG{p}{;}\PYG{+w}{ }
\PYG{+w}{    }\PYG{p}{\PYGZcb{}}
\PYG{+w}{    }
\PYG{+w}{    }\PYG{o}{*}\PYG{k+kr}{thread}\PYG{+w}{ }\PYG{o}{=}\PYG{+w}{ }\PYG{n}{i}\PYG{p}{;}

\PYG{+w}{    }\PYG{n}{tcb}\PYG{p}{[}\PYG{n}{i}\PYG{p}{]}\PYG{p}{.}\PYG{n}{pthid}\PYG{+w}{ }\PYG{o}{=}\PYG{+w}{ }\PYG{n}{i}\PYG{p}{;}
\PYG{+w}{    }\PYG{n}{tcb}\PYG{p}{[}\PYG{n}{i}\PYG{p}{]}\PYG{p}{.}\PYG{n}{pthArg}\PYG{+w}{ }\PYG{o}{=}\PYG{+w}{ }\PYG{p}{(}\PYG{n}{uint64}\PYG{p}{)}\PYG{n}{arg}\PYG{p}{;}
\PYG{+w}{    }\PYG{n}{tcb}\PYG{p}{[}\PYG{n}{i}\PYG{p}{]}\PYG{p}{.}\PYG{n}{joinid}\PYG{+w}{ }\PYG{o}{=}\PYG{+w}{ }\PYG{l+m+mi}{\PYGZhy{}1}\PYG{p}{;}
\PYG{+w}{    }\PYG{n}{tcb}\PYG{p}{[}\PYG{n}{i}\PYG{p}{]}\PYG{p}{.}\PYG{n}{retval}\PYG{+w}{ }\PYG{o}{=}\PYG{+w}{ }\PYG{l+m+mi}{0}\PYG{p}{;}

\PYG{+w}{    }\PYG{n}{uint64}\PYG{+w}{ }\PYG{n}{stack\PYGZus{}top}\PYG{+w}{ }\PYG{o}{=}\PYG{+w}{ }\PYG{p}{(}\PYG{n}{uint64}\PYG{p}{)}\PYG{o}{\PYGZam{}}\PYG{n}{tcb}\PYG{p}{[}\PYG{n}{i}\PYG{p}{]}\PYG{p}{.}\PYG{n}{stack}\PYG{p}{[}\PYG{n}{MAX\PYGZus{}STACK\PYGZus{}SIZE}\PYG{p}{]}\PYG{p}{;}
\PYG{+w}{    }
\PYG{+w}{    }\PYG{n}{stack\PYGZus{}top}\PYG{+w}{ }\PYG{o}{=}\PYG{+w}{ }\PYG{p}{(}\PYG{n}{stack\PYGZus{}top}\PYG{+w}{ }\PYG{o}{\PYGZhy{}}\PYG{+w}{ }\PYG{l+m+mi}{128}\PYG{p}{)}\PYG{+w}{ }\PYG{o}{\PYGZam{}}\PYG{+w}{ }\PYG{o}{\PYGZti{}}\PYG{l+m+mh}{0xF}\PYG{p}{;}

\PYG{+w}{    }\PYG{n}{tcb}\PYG{p}{[}\PYG{n}{i}\PYG{p}{]}\PYG{p}{.}\PYG{n}{context}\PYG{p}{.}\PYG{n}{ra}\PYG{+w}{ }\PYG{o}{=}\PYG{+w}{ }\PYG{p}{(}\PYG{n}{uint64}\PYG{p}{)}\PYG{n}{thread\PYGZus{}wrapper}\PYG{p}{;}\PYG{+w}{   }
\PYG{+w}{    }\PYG{n}{tcb}\PYG{p}{[}\PYG{n}{i}\PYG{p}{]}\PYG{p}{.}\PYG{n}{context}\PYG{p}{.}\PYG{n}{sp}\PYG{+w}{ }\PYG{o}{=}\PYG{+w}{ }\PYG{n}{stack\PYGZus{}top}\PYG{p}{;}\PYG{+w}{                }
\PYG{+w}{    }\PYG{n}{tcb}\PYG{p}{[}\PYG{n}{i}\PYG{p}{]}\PYG{p}{.}\PYG{n}{context}\PYG{p}{.}\PYG{n}{fp}\PYG{+w}{ }\PYG{o}{=}\PYG{+w}{ }\PYG{n}{stack\PYGZus{}top}\PYG{p}{;}\PYG{+w}{               }
\PYG{+w}{    }
\PYG{+w}{  }
\PYG{+w}{    }\PYG{n}{tcb}\PYG{p}{[}\PYG{n}{i}\PYG{p}{]}\PYG{p}{.}\PYG{n}{context}\PYG{p}{.}\PYG{n}{s0}\PYG{+w}{ }\PYG{o}{=}\PYG{+w}{ }\PYG{p}{(}\PYG{n}{uint64}\PYG{p}{)}\PYG{n}{start\PYGZus{}routine}\PYG{p}{;}\PYG{+w}{    }
\PYG{+w}{    }\PYG{n}{tcb}\PYG{p}{[}\PYG{n}{i}\PYG{p}{]}\PYG{p}{.}\PYG{n}{context}\PYG{p}{.}\PYG{n}{s1}\PYG{+w}{ }\PYG{o}{=}\PYG{+w}{ }\PYG{p}{(}\PYG{n}{uint64}\PYG{p}{)}\PYG{n}{arg}\PYG{p}{;}\PYG{+w}{              }
\PYG{+w}{    }

\PYG{+w}{    }\PYG{n}{tcb}\PYG{p}{[}\PYG{n}{i}\PYG{p}{]}\PYG{p}{.}\PYG{n}{context}\PYG{p}{.}\PYG{n}{s2}\PYG{+w}{ }\PYG{o}{=}\PYG{+w}{ }\PYG{l+m+mi}{0}\PYG{p}{;}
\PYG{+w}{    }\PYG{n}{tcb}\PYG{p}{[}\PYG{n}{i}\PYG{p}{]}\PYG{p}{.}\PYG{n}{context}\PYG{p}{.}\PYG{n}{s3}\PYG{+w}{ }\PYG{o}{=}\PYG{+w}{ }\PYG{l+m+mi}{0}\PYG{p}{;}
\PYG{+w}{    }\PYG{n}{tcb}\PYG{p}{[}\PYG{n}{i}\PYG{p}{]}\PYG{p}{.}\PYG{n}{context}\PYG{p}{.}\PYG{n}{s4}\PYG{+w}{ }\PYG{o}{=}\PYG{+w}{ }\PYG{l+m+mi}{0}\PYG{p}{;}
\PYG{+w}{    }\PYG{n}{tcb}\PYG{p}{[}\PYG{n}{i}\PYG{p}{]}\PYG{p}{.}\PYG{n}{context}\PYG{p}{.}\PYG{n}{s5}\PYG{+w}{ }\PYG{o}{=}\PYG{+w}{ }\PYG{l+m+mi}{0}\PYG{p}{;}
\PYG{+w}{    }\PYG{n}{tcb}\PYG{p}{[}\PYG{n}{i}\PYG{p}{]}\PYG{p}{.}\PYG{n}{context}\PYG{p}{.}\PYG{n}{s6}\PYG{+w}{ }\PYG{o}{=}\PYG{+w}{ }\PYG{l+m+mi}{0}\PYG{p}{;}
\PYG{+w}{    }\PYG{n}{tcb}\PYG{p}{[}\PYG{n}{i}\PYG{p}{]}\PYG{p}{.}\PYG{n}{context}\PYG{p}{.}\PYG{n}{s7}\PYG{+w}{ }\PYG{o}{=}\PYG{+w}{ }\PYG{l+m+mi}{0}\PYG{p}{;}
\PYG{+w}{    }\PYG{n}{tcb}\PYG{p}{[}\PYG{n}{i}\PYG{p}{]}\PYG{p}{.}\PYG{n}{context}\PYG{p}{.}\PYG{n}{s8}\PYG{+w}{ }\PYG{o}{=}\PYG{+w}{ }\PYG{l+m+mi}{0}\PYG{p}{;}
\PYG{+w}{    }
\PYG{+w}{    }\PYG{n}{tcb}\PYG{p}{[}\PYG{n}{i}\PYG{p}{]}\PYG{p}{.}\PYG{n}{state}\PYG{+w}{ }\PYG{o}{=}\PYG{+w}{ }\PYG{n}{STATE\PYGZus{}RUNNABLE}\PYG{p}{;}
\PYG{+w}{    }
\PYG{+w}{    }\PYG{k}{return}\PYG{+w}{ }\PYG{l+m+mi}{0}\PYG{p}{;}
\PYG{p}{\PYGZcb{}}
\end{sphinxVerbatim}

\sphinxAtStartPar
找一个空闲的TCB槽位,没有空闲槽位返回\sphinxhyphen{}1。初始化线程控制块,设置线程栈(向下增长),这里需要注意我们要给栈预留空间,栈需要16字节对齐,并且预留一些空间给函数调用,在栈上设置返回地址当线程函数返回时调用pthread\_exit,最后使用包装函数,可以省略此步骤。初始化上下文,设置返回地址指向包装函数,设置栈、帧指针。保存函数指针和参数到寄存器,并初始化其他寄存器(为0)。设置状态为STATE\_RUNNABLE。


\subsubsection{pthread\_yield}
\label{\detokenize{nju:pthread-yield}}
\begin{sphinxVerbatim}[commandchars=\\\{\}]
\PYG{k+kt}{int}\PYG{+w}{ }\PYG{n+nf}{pthread\PYGZus{}yield}\PYG{p}{(}\PYG{k+kt}{void}\PYG{p}{)}\PYG{+w}{ }\PYG{p}{\PYGZob{}}
\PYG{+w}{    }\PYG{n}{\PYGZus{}\PYGZus{}asm\PYGZus{}\PYGZus{}}\PYG{+w}{ }\PYG{n}{\PYGZus{}\PYGZus{}volatile\PYGZus{}\PYGZus{}}\PYG{p}{(}
\PYG{+w}{        }\PYG{l+s}{\PYGZdq{}}\PYG{l+s}{st.d \PYGZdl{}ra, \PYGZpc{}0, 0}\PYG{l+s+se}{\PYGZbs{}n}\PYG{l+s+se}{\PYGZbs{}t}\PYG{l+s}{\PYGZdq{}}
\PYG{+w}{        }\PYG{l+s}{\PYGZdq{}}\PYG{l+s}{st.d \PYGZdl{}sp, \PYGZpc{}0, 8}\PYG{l+s+se}{\PYGZbs{}n}\PYG{l+s+se}{\PYGZbs{}t}\PYG{l+s}{\PYGZdq{}}
\PYG{+w}{        }\PYG{l+s}{\PYGZdq{}}\PYG{l+s}{st.d \PYGZdl{}s0, \PYGZpc{}0, 16}\PYG{l+s+se}{\PYGZbs{}n}\PYG{l+s+se}{\PYGZbs{}t}\PYG{l+s}{\PYGZdq{}}
\PYG{+w}{        }\PYG{l+s}{\PYGZdq{}}\PYG{l+s}{st.d \PYGZdl{}s1, \PYGZpc{}0, 24}\PYG{l+s+se}{\PYGZbs{}n}\PYG{l+s+se}{\PYGZbs{}t}\PYG{l+s}{\PYGZdq{}}
\PYG{+w}{        }\PYG{l+s}{\PYGZdq{}}\PYG{l+s}{st.d \PYGZdl{}s2, \PYGZpc{}0, 32}\PYG{l+s+se}{\PYGZbs{}n}\PYG{l+s+se}{\PYGZbs{}t}\PYG{l+s}{\PYGZdq{}}
\PYG{+w}{        }\PYG{l+s}{\PYGZdq{}}\PYG{l+s}{st.d \PYGZdl{}s3, \PYGZpc{}0, 40}\PYG{l+s+se}{\PYGZbs{}n}\PYG{l+s+se}{\PYGZbs{}t}\PYG{l+s}{\PYGZdq{}}
\PYG{+w}{        }\PYG{l+s}{\PYGZdq{}}\PYG{l+s}{st.d \PYGZdl{}s4, \PYGZpc{}0, 48}\PYG{l+s+se}{\PYGZbs{}n}\PYG{l+s+se}{\PYGZbs{}t}\PYG{l+s}{\PYGZdq{}}
\PYG{+w}{        }\PYG{l+s}{\PYGZdq{}}\PYG{l+s}{st.d \PYGZdl{}s5, \PYGZpc{}0, 56}\PYG{l+s+se}{\PYGZbs{}n}\PYG{l+s+se}{\PYGZbs{}t}\PYG{l+s}{\PYGZdq{}}
\PYG{+w}{        }\PYG{l+s}{\PYGZdq{}}\PYG{l+s}{st.d \PYGZdl{}s6, \PYGZpc{}0, 64}\PYG{l+s+se}{\PYGZbs{}n}\PYG{l+s+se}{\PYGZbs{}t}\PYG{l+s}{\PYGZdq{}}
\PYG{+w}{        }\PYG{l+s}{\PYGZdq{}}\PYG{l+s}{st.d \PYGZdl{}s7, \PYGZpc{}0, 72}\PYG{l+s+se}{\PYGZbs{}n}\PYG{l+s+se}{\PYGZbs{}t}\PYG{l+s}{\PYGZdq{}}
\PYG{+w}{        }\PYG{l+s}{\PYGZdq{}}\PYG{l+s}{st.d \PYGZdl{}s8, \PYGZpc{}0, 80}\PYG{l+s+se}{\PYGZbs{}n}\PYG{l+s+se}{\PYGZbs{}t}\PYG{l+s}{\PYGZdq{}}
\PYG{+w}{        }\PYG{l+s}{\PYGZdq{}}\PYG{l+s}{st.d \PYGZdl{}fp, \PYGZpc{}0, 88}\PYG{l+s+se}{\PYGZbs{}n}\PYG{l+s+se}{\PYGZbs{}t}\PYG{l+s}{\PYGZdq{}}
\PYG{+w}{        }\PYG{o}{:}\PYG{+w}{ }
\PYG{+w}{        }\PYG{o}{:}\PYG{+w}{ }\PYG{l+s}{\PYGZdq{}}\PYG{l+s}{r}\PYG{l+s}{\PYGZdq{}}\PYG{p}{(}\PYG{o}{\PYGZam{}}\PYG{n}{tcb}\PYG{p}{[}\PYG{n}{current}\PYG{p}{]}\PYG{p}{.}\PYG{n}{context}\PYG{p}{)}
\PYG{+w}{        }\PYG{o}{:}\PYG{+w}{ }\PYG{l+s}{\PYGZdq{}}\PYG{l+s}{memory}\PYG{l+s}{\PYGZdq{}}
\PYG{+w}{    }\PYG{p}{)}\PYG{p}{;}
\PYG{+w}{    }
\PYG{+w}{    }\PYG{n}{tcb}\PYG{p}{[}\PYG{n}{current}\PYG{p}{]}\PYG{p}{.}\PYG{n}{state}\PYG{+w}{ }\PYG{o}{=}\PYG{+w}{ }\PYG{n}{STATE\PYGZus{}RUNNABLE}\PYG{p}{;}
\PYG{+w}{    }
\PYG{+w}{    }\PYG{k+kt}{int}\PYG{+w}{ }\PYG{n}{next}\PYG{+w}{ }\PYG{o}{=}\PYG{+w}{ }\PYG{l+m+mi}{\PYGZhy{}1}\PYG{p}{;}
\PYG{+w}{    }
\PYG{+w}{    }\PYG{k}{for}\PYG{+w}{ }\PYG{p}{(}\PYG{k+kt}{int}\PYG{+w}{ }\PYG{n}{i}\PYG{+w}{ }\PYG{o}{=}\PYG{+w}{ }\PYG{l+m+mi}{1}\PYG{p}{;}\PYG{+w}{ }\PYG{n}{i}\PYG{+w}{ }\PYG{o}{\PYGZlt{}}\PYG{o}{=}\PYG{+w}{ }\PYG{n}{MAX\PYGZus{}TCB\PYGZus{}NUM}\PYG{p}{;}\PYG{+w}{ }\PYG{n}{i}\PYG{o}{+}\PYG{o}{+}\PYG{p}{)}\PYG{+w}{ }\PYG{p}{\PYGZob{}}
\PYG{+w}{        }\PYG{k+kt}{int}\PYG{+w}{ }\PYG{n}{idx}\PYG{+w}{ }\PYG{o}{=}\PYG{+w}{ }\PYG{p}{(}\PYG{n}{current}\PYG{+w}{ }\PYG{o}{+}\PYG{+w}{ }\PYG{n}{i}\PYG{p}{)}\PYG{+w}{ }\PYG{o}{\PYGZpc{}}\PYG{+w}{ }\PYG{n}{MAX\PYGZus{}TCB\PYGZus{}NUM}\PYG{p}{;}
\PYG{+w}{        }\PYG{k}{if}\PYG{+w}{ }\PYG{p}{(}\PYG{n}{tcb}\PYG{p}{[}\PYG{n}{idx}\PYG{p}{]}\PYG{p}{.}\PYG{n}{state}\PYG{+w}{ }\PYG{o}{=}\PYG{o}{=}\PYG{+w}{ }\PYG{n}{STATE\PYGZus{}RUNNABLE}\PYG{p}{)}\PYG{+w}{ }\PYG{p}{\PYGZob{}}
\PYG{+w}{            }\PYG{n}{next}\PYG{+w}{ }\PYG{o}{=}\PYG{+w}{ }\PYG{n}{idx}\PYG{p}{;}
\PYG{+w}{            }\PYG{k}{break}\PYG{p}{;}
\PYG{+w}{        }\PYG{p}{\PYGZcb{}}
\PYG{+w}{    }\PYG{p}{\PYGZcb{}}
\PYG{+w}{    }
\PYG{+w}{    }\PYG{k}{if}\PYG{+w}{ }\PYG{p}{(}\PYG{n}{next}\PYG{+w}{ }\PYG{o}{=}\PYG{o}{=}\PYG{+w}{ }\PYG{l+m+mi}{\PYGZhy{}1}\PYG{p}{)}\PYG{+w}{ }\PYG{p}{\PYGZob{}}
\PYG{+w}{        }\PYG{n}{tcb}\PYG{p}{[}\PYG{n}{current}\PYG{p}{]}\PYG{p}{.}\PYG{n}{state}\PYG{+w}{ }\PYG{o}{=}\PYG{+w}{ }\PYG{n}{STATE\PYGZus{}RUNNING}\PYG{p}{;}
\PYG{+w}{        }\PYG{k}{return}\PYG{+w}{ }\PYG{l+m+mi}{0}\PYG{p}{;}
\PYG{+w}{    }\PYG{p}{\PYGZcb{}}
\PYG{+w}{    }\PYG{k+kt}{int}\PYG{+w}{ }\PYG{n}{prev}\PYG{+w}{ }\PYG{o}{=}\PYG{+w}{ }\PYG{n}{current}\PYG{p}{;}
\PYG{+w}{    }\PYG{n}{current}\PYG{+w}{ }\PYG{o}{=}\PYG{+w}{ }\PYG{n}{next}\PYG{p}{;}
\PYG{+w}{    }\PYG{n}{tcb}\PYG{p}{[}\PYG{n}{current}\PYG{p}{]}\PYG{p}{.}\PYG{n}{state}\PYG{+w}{ }\PYG{o}{=}\PYG{+w}{ }\PYG{n}{STATE\PYGZus{}RUNNING}\PYG{p}{;}
\PYG{+w}{    }
\PYG{+w}{    }\PYG{n}{user\PYGZus{}swtch}\PYG{p}{(}\PYG{o}{\PYGZam{}}\PYG{n}{tcb}\PYG{p}{[}\PYG{n}{prev}\PYG{p}{]}\PYG{p}{.}\PYG{n}{context}\PYG{p}{,}\PYG{+w}{ }\PYG{o}{\PYGZam{}}\PYG{n}{tcb}\PYG{p}{[}\PYG{n}{current}\PYG{p}{]}\PYG{p}{.}\PYG{n}{context}\PYG{p}{)}\PYG{p}{;}
\PYG{+w}{    }
\PYG{+w}{    }\PYG{k}{return}\PYG{+w}{ }\PYG{l+m+mi}{0}\PYG{p}{;}
\PYG{p}{\PYGZcb{}}
\end{sphinxVerbatim}

\sphinxAtStartPar
pthread\_yield 函数会使得调用此函数的线程让出 CPU。实验要求 pthread\_yield 调用成功返回 0;出错返回\sphinxhyphen{}1。实际测试只考虑调用成功的情况。第一步是保存现场信息,保存当前线程的完整上下文。第二步是查找一个处于 RUNNABLE 状态,把当前线程设置成 RUNNABLE,被选中进程设置为 RUNNING,然后 current 赋值为被选中线程。第三步是选择下一个可运行的线程,从当前的下一个开始查找,如果没有其他可运行线程,继续当前的线程,否则切换下一个线程。


\subsubsection{pthread\_join}
\label{\detokenize{nju:pthread-join}}
\begin{sphinxVerbatim}[commandchars=\\\{\}]
\PYG{k+kt}{int}\PYG{+w}{ }\PYG{n+nf}{pthread\PYGZus{}join}\PYG{p}{(}\PYG{n}{uint32}\PYG{+w}{ }\PYG{k+kr}{thread}\PYG{p}{,}\PYG{+w}{ }\PYG{k+kt}{void}\PYG{+w}{ }\PYG{o}{*}\PYG{o}{*}\PYG{n}{retval}\PYG{p}{)}\PYG{+w}{ }\PYG{p}{\PYGZob{}}

\PYG{+w}{    }\PYG{k}{if}\PYG{+w}{ }\PYG{p}{(}\PYG{k+kr}{thread}\PYG{+w}{ }\PYG{o}{\PYGZgt{}}\PYG{o}{=}\PYG{+w}{ }\PYG{n}{MAX\PYGZus{}TCB\PYGZus{}NUM}\PYG{+w}{ }\PYG{o}{|}\PYG{o}{|}\PYG{+w}{ }\PYG{k+kr}{thread}\PYG{+w}{ }\PYG{o}{=}\PYG{o}{=}\PYG{+w}{ }\PYG{l+m+mi}{0}\PYG{p}{)}\PYG{+w}{ }\PYG{p}{\PYGZob{}}\PYG{+w}{  }
\PYG{+w}{        }\PYG{k}{return}\PYG{+w}{ }\PYG{l+m+mi}{\PYGZhy{}1}\PYG{p}{;}
\PYG{+w}{    }\PYG{p}{\PYGZcb{}}
\PYG{+w}{    }
\PYG{+w}{    }\PYG{k}{if}\PYG{+w}{ }\PYG{p}{(}\PYG{n}{tcb}\PYG{p}{[}\PYG{k+kr}{thread}\PYG{p}{]}\PYG{p}{.}\PYG{n}{state}\PYG{+w}{ }\PYG{o}{=}\PYG{o}{=}\PYG{+w}{ }\PYG{n}{STATE\PYGZus{}DEAD}\PYG{p}{)}\PYG{+w}{ }\PYG{p}{\PYGZob{}}
\PYG{+w}{        }\PYG{k}{if}\PYG{+w}{ }\PYG{p}{(}\PYG{n}{retval}\PYG{+w}{ }\PYG{o}{!}\PYG{o}{=}\PYG{+w}{ }\PYG{n+nb}{NULL}\PYG{p}{)}\PYG{+w}{ }\PYG{p}{\PYGZob{}}
\PYG{+w}{            }\PYG{o}{*}\PYG{n}{retval}\PYG{+w}{ }\PYG{o}{=}\PYG{+w}{ }\PYG{p}{(}\PYG{k+kt}{void}\PYG{+w}{ }\PYG{o}{*}\PYG{p}{)}\PYG{p}{(}\PYG{n}{uint64}\PYG{p}{)}\PYG{n}{tcb}\PYG{p}{[}\PYG{k+kr}{thread}\PYG{p}{]}\PYG{p}{.}\PYG{n}{retval}\PYG{p}{;}
\PYG{+w}{        }\PYG{p}{\PYGZcb{}}
\PYG{+w}{        }\PYG{k}{return}\PYG{+w}{ }\PYG{l+m+mi}{0}\PYG{p}{;}
\PYG{+w}{    }\PYG{p}{\PYGZcb{}}
\PYG{+w}{    }
\PYG{+w}{    }\PYG{n}{tcb}\PYG{p}{[}\PYG{k+kr}{thread}\PYG{p}{]}\PYG{p}{.}\PYG{n}{joinid}\PYG{+w}{ }\PYG{o}{=}\PYG{+w}{ }\PYG{n}{current}\PYG{p}{;}

\PYG{+w}{    }\PYG{k}{while}\PYG{+w}{ }\PYG{p}{(}\PYG{n}{tcb}\PYG{p}{[}\PYG{k+kr}{thread}\PYG{p}{]}\PYG{p}{.}\PYG{n}{state}\PYG{+w}{ }\PYG{o}{!}\PYG{o}{=}\PYG{+w}{ }\PYG{n}{STATE\PYGZus{}DEAD}\PYG{p}{)}\PYG{+w}{ }\PYG{p}{\PYGZob{}}
\PYG{+w}{        }\PYG{n}{pthread\PYGZus{}yield}\PYG{p}{(}\PYG{p}{)}\PYG{p}{;}
\PYG{+w}{    }\PYG{p}{\PYGZcb{}}
\PYG{+w}{    }
\PYG{+w}{    }\PYG{k}{if}\PYG{+w}{ }\PYG{p}{(}\PYG{n}{retval}\PYG{+w}{ }\PYG{o}{!}\PYG{o}{=}\PYG{+w}{ }\PYG{n+nb}{NULL}\PYG{p}{)}\PYG{+w}{ }\PYG{p}{\PYGZob{}}
\PYG{+w}{        }\PYG{o}{*}\PYG{n}{retval}\PYG{+w}{ }\PYG{o}{=}\PYG{+w}{ }\PYG{p}{(}\PYG{k+kt}{void}\PYG{+w}{ }\PYG{o}{*}\PYG{p}{)}\PYG{p}{(}\PYG{n}{uint64}\PYG{p}{)}\PYG{n}{tcb}\PYG{p}{[}\PYG{k+kr}{thread}\PYG{p}{]}\PYG{p}{.}\PYG{n}{retval}\PYG{p}{;}
\PYG{+w}{    }\PYG{p}{\PYGZcb{}}
\PYG{+w}{    }
\PYG{+w}{    }\PYG{k}{return}\PYG{+w}{ }\PYG{l+m+mi}{0}\PYG{p}{;}
\PYG{p}{\PYGZcb{}}

\end{sphinxVerbatim}

\sphinxAtStartPar
pthread\_join 函数会等待 thread 指向的线程结束。首先检查参数有效性,如果线程已经结束,直接返回。否则记录当前线程正在等待这个线程,等待线程结束,获取返回值。


\subsubsection{pthread\_exit}
\label{\detokenize{nju:pthread-exit}}
\begin{sphinxVerbatim}[commandchars=\\\{\}]
\PYG{k+kt}{void}\PYG{+w}{ }\PYG{n+nf}{pthread\PYGZus{}exit}\PYG{p}{(}\PYG{k+kt}{void}\PYG{+w}{ }\PYG{o}{*}\PYG{n}{retval}\PYG{p}{)}\PYG{+w}{ }\PYG{p}{\PYGZob{}}
\PYG{+w}{    }\PYG{n}{tcb}\PYG{p}{[}\PYG{n}{current}\PYG{p}{]}\PYG{p}{.}\PYG{n}{retval}\PYG{+w}{ }\PYG{o}{=}\PYG{+w}{ }\PYG{p}{(}\PYG{n}{uint64}\PYG{p}{)}\PYG{n}{retval}\PYG{p}{;}
\PYG{+w}{    }
\PYG{+w}{    }\PYG{k+kt}{int}\PYG{+w}{ }\PYG{n}{joiner}\PYG{+w}{ }\PYG{o}{=}\PYG{+w}{ }\PYG{n}{tcb}\PYG{p}{[}\PYG{n}{current}\PYG{p}{]}\PYG{p}{.}\PYG{n}{joinid}\PYG{p}{;}
\PYG{+w}{    }
\PYG{+w}{    }\PYG{n}{tcb}\PYG{p}{[}\PYG{n}{current}\PYG{p}{]}\PYG{p}{.}\PYG{n}{state}\PYG{+w}{ }\PYG{o}{=}\PYG{+w}{ }\PYG{n}{STATE\PYGZus{}DEAD}\PYG{p}{;}

\PYG{+w}{    }\PYG{k+kt}{int}\PYG{+w}{ }\PYG{n}{next}\PYG{+w}{ }\PYG{o}{=}\PYG{+w}{ }\PYG{l+m+mi}{\PYGZhy{}1}\PYG{p}{;}
\PYG{+w}{    }
\PYG{+w}{    }\PYG{k}{if}\PYG{+w}{ }\PYG{p}{(}\PYG{n}{tcb}\PYG{p}{[}\PYG{l+m+mi}{0}\PYG{p}{]}\PYG{p}{.}\PYG{n}{state}\PYG{+w}{ }\PYG{o}{=}\PYG{o}{=}\PYG{+w}{ }\PYG{n}{STATE\PYGZus{}RUNNABLE}\PYG{+w}{ }\PYG{o}{|}\PYG{o}{|}\PYG{+w}{ }\PYG{n}{tcb}\PYG{p}{[}\PYG{l+m+mi}{0}\PYG{p}{]}\PYG{p}{.}\PYG{n}{state}\PYG{+w}{ }\PYG{o}{=}\PYG{o}{=}\PYG{+w}{ }\PYG{n}{STATE\PYGZus{}RUNNING}\PYG{p}{)}\PYG{+w}{ }\PYG{p}{\PYGZob{}}
\PYG{+w}{        }\PYG{n}{next}\PYG{+w}{ }\PYG{o}{=}\PYG{+w}{ }\PYG{l+m+mi}{0}\PYG{p}{;}
\PYG{+w}{    }\PYG{p}{\PYGZcb{}}\PYG{+w}{ }\PYG{k}{else}\PYG{+w}{ }\PYG{p}{\PYGZob{}}
\PYG{+w}{        }\PYG{k}{for}\PYG{+w}{ }\PYG{p}{(}\PYG{k+kt}{int}\PYG{+w}{ }\PYG{n}{i}\PYG{+w}{ }\PYG{o}{=}\PYG{+w}{ }\PYG{l+m+mi}{1}\PYG{p}{;}\PYG{+w}{ }\PYG{n}{i}\PYG{+w}{ }\PYG{o}{\PYGZlt{}}\PYG{+w}{ }\PYG{n}{MAX\PYGZus{}TCB\PYGZus{}NUM}\PYG{p}{;}\PYG{+w}{ }\PYG{n}{i}\PYG{o}{+}\PYG{o}{+}\PYG{p}{)}\PYG{+w}{ }\PYG{p}{\PYGZob{}}
\PYG{+w}{            }\PYG{k}{if}\PYG{+w}{ }\PYG{p}{(}\PYG{n}{i}\PYG{+w}{ }\PYG{o}{!}\PYG{o}{=}\PYG{+w}{ }\PYG{n}{current}\PYG{+w}{ }\PYG{o}{\PYGZam{}}\PYG{o}{\PYGZam{}}\PYG{+w}{ }\PYG{n}{tcb}\PYG{p}{[}\PYG{n}{i}\PYG{p}{]}\PYG{p}{.}\PYG{n}{state}\PYG{+w}{ }\PYG{o}{=}\PYG{o}{=}\PYG{+w}{ }\PYG{n}{STATE\PYGZus{}RUNNABLE}\PYG{p}{)}\PYG{+w}{ }\PYG{p}{\PYGZob{}}
\PYG{+w}{                }\PYG{n}{next}\PYG{+w}{ }\PYG{o}{=}\PYG{+w}{ }\PYG{n}{i}\PYG{p}{;}
\PYG{+w}{                }\PYG{k}{break}\PYG{p}{;}
\PYG{+w}{            }\PYG{p}{\PYGZcb{}}
\PYG{+w}{        }\PYG{p}{\PYGZcb{}}
\PYG{+w}{    }\PYG{p}{\PYGZcb{}}

\PYG{+w}{    }\PYG{k}{if}\PYG{+w}{ }\PYG{p}{(}\PYG{n}{joiner}\PYG{+w}{ }\PYG{o}{!}\PYG{o}{=}\PYG{+w}{ }\PYG{l+m+mi}{\PYGZhy{}1}\PYG{+w}{ }\PYG{o}{\PYGZam{}}\PYG{o}{\PYGZam{}}\PYG{+w}{ }\PYG{n}{tcb}\PYG{p}{[}\PYG{n}{joiner}\PYG{p}{]}\PYG{p}{.}\PYG{n}{state}\PYG{+w}{ }\PYG{o}{=}\PYG{o}{=}\PYG{+w}{ }\PYG{n}{STATE\PYGZus{}RUNNABLE}\PYG{p}{)}\PYG{+w}{ }\PYG{p}{\PYGZob{}}
\PYG{+w}{        }\PYG{n}{next}\PYG{+w}{ }\PYG{o}{=}\PYG{+w}{ }\PYG{n}{joiner}\PYG{p}{;}
\PYG{+w}{    }\PYG{p}{\PYGZcb{}}
\PYG{+w}{    }
\PYG{+w}{    }\PYG{k}{if}\PYG{+w}{ }\PYG{p}{(}\PYG{n}{next}\PYG{+w}{ }\PYG{o}{=}\PYG{o}{=}\PYG{+w}{ }\PYG{l+m+mi}{\PYGZhy{}1}\PYG{p}{)}\PYG{+w}{ }\PYG{p}{\PYGZob{}}
\PYG{+w}{        }\PYG{n}{exit}\PYG{p}{(}\PYG{l+m+mi}{0}\PYG{p}{)}\PYG{p}{;}
\PYG{+w}{    }\PYG{p}{\PYGZcb{}}
\PYG{+w}{    }
\PYG{+w}{    }\PYG{k+kt}{int}\PYG{+w}{ }\PYG{n}{prev}\PYG{+w}{ }\PYG{o}{=}\PYG{+w}{ }\PYG{n}{current}\PYG{p}{;}
\PYG{+w}{    }\PYG{n}{current}\PYG{+w}{ }\PYG{o}{=}\PYG{+w}{ }\PYG{n}{next}\PYG{p}{;}
\PYG{+w}{    }\PYG{n}{tcb}\PYG{p}{[}\PYG{n}{current}\PYG{p}{]}\PYG{p}{.}\PYG{n}{state}\PYG{+w}{ }\PYG{o}{=}\PYG{+w}{ }\PYG{n}{STATE\PYGZus{}RUNNING}\PYG{p}{;}
\PYG{+w}{    }
\PYG{+w}{    }\PYG{n}{user\PYGZus{}swtch}\PYG{p}{(}\PYG{o}{\PYGZam{}}\PYG{n}{tcb}\PYG{p}{[}\PYG{n}{prev}\PYG{p}{]}\PYG{p}{.}\PYG{n}{context}\PYG{p}{,}\PYG{+w}{ }\PYG{o}{\PYGZam{}}\PYG{n}{tcb}\PYG{p}{[}\PYG{n}{current}\PYG{p}{]}\PYG{p}{.}\PYG{n}{context}\PYG{p}{)}\PYG{p}{;}
\PYG{+w}{    }
\PYG{p}{\PYGZcb{}}
\end{sphinxVerbatim}

\sphinxAtStartPar
pthread\_exit 函数会结束当前线程。保存返回值,如果有线程等待join这个线程,需要唤醒它,标记线程为结束状态,找到下一个可运行线程,如果有线程在等待join当前线程,优先运行它,如果找不到任何可运行线程,退出进程否则切换到下一个线程。join 使用 while 等待 thread 指向的线程结束,在循环体内调度运行 thread 指向的线程,而 exit 则将当前线程置为 DEAD状态,然后调度运行一个 RUNNABLE 状态的线程。


\subsubsection{main}
\label{\detokenize{nju:id16}}
\begin{sphinxVerbatim}[commandchars=\\\{\}]
\PYG{k+kt}{int}
\PYG{n+nf}{main}\PYG{p}{(}\PYG{k+kt}{void}\PYG{p}{)}
\PYG{p}{\PYGZob{}}
\PYG{+w}{    }\PYG{k}{if}\PYG{p}{(}\PYG{n}{open}\PYG{p}{(}\PYG{l+s}{\PYGZdq{}}\PYG{l+s}{console}\PYG{l+s}{\PYGZdq{}}\PYG{p}{,}\PYG{+w}{ }\PYG{n}{O\PYGZus{}RDWR}\PYG{p}{)}\PYG{+w}{ }\PYG{o}{\PYGZlt{}}\PYG{+w}{ }\PYG{l+m+mi}{0}\PYG{p}{)}\PYG{p}{\PYGZob{}}
\PYG{+w}{        }\PYG{n}{mknod}\PYG{p}{(}\PYG{l+s}{\PYGZdq{}}\PYG{l+s}{console}\PYG{l+s}{\PYGZdq{}}\PYG{p}{,}\PYG{+w}{ }\PYG{n}{CONSOLE}\PYG{p}{,}\PYG{+w}{ }\PYG{l+m+mi}{0}\PYG{p}{)}\PYG{p}{;}
\PYG{+w}{        }\PYG{n}{open}\PYG{p}{(}\PYG{l+s}{\PYGZdq{}}\PYG{l+s}{console}\PYG{l+s}{\PYGZdq{}}\PYG{p}{,}\PYG{+w}{ }\PYG{n}{O\PYGZus{}RDWR}\PYG{p}{)}\PYG{p}{;}
\PYG{+w}{    }\PYG{p}{\PYGZcb{}}
\PYG{+w}{    }\PYG{n}{dup}\PYG{p}{(}\PYG{l+m+mi}{0}\PYG{p}{)}\PYG{p}{;}\PYG{+w}{  }\PYG{c+c1}{// stdout}
\PYG{+w}{    }\PYG{n}{dup}\PYG{p}{(}\PYG{l+m+mi}{0}\PYG{p}{)}\PYG{p}{;}\PYG{+w}{  }\PYG{c+c1}{// stderr}

\PYG{c+c1}{//lab3}
\PYG{+w}{    }\PYG{k+kt}{int}\PYG{+w}{ }\PYG{n}{i}\PYG{o}{=}\PYG{l+m+mi}{8}\PYG{p}{;}
\PYG{+w}{    }\PYG{k+kt}{int}\PYG{+w}{ }\PYG{n}{data}\PYG{+w}{ }\PYG{o}{=}\PYG{+w}{ }\PYG{l+m+mi}{0}\PYG{p}{;}
\PYG{+w}{    }\PYG{k+kt}{int}\PYG{+w}{ }\PYG{n}{ret}\PYG{o}{=}\PYG{l+m+mi}{0}\PYG{p}{;}
\PYG{+w}{    }\PYG{n}{printf}\PYG{p}{(}\PYG{l+s}{\PYGZdq{}}\PYG{l+s}{ret:\PYGZpc{}d}\PYG{l+s+se}{\PYGZbs{}n}\PYG{l+s}{\PYGZdq{}}\PYG{p}{,}\PYG{n}{ret}\PYG{p}{)}\PYG{p}{;}
\PYG{+w}{    }\PYG{n}{ret}\PYG{+w}{ }\PYG{o}{=}\PYG{+w}{ }\PYG{n}{fork}\PYG{p}{(}\PYG{p}{)}\PYG{p}{;}
\PYG{+w}{    }\PYG{n}{printf}\PYG{p}{(}\PYG{l+s}{\PYGZdq{}}\PYG{l+s}{ret:\PYGZpc{}d}\PYG{l+s+se}{\PYGZbs{}n}\PYG{l+s}{\PYGZdq{}}\PYG{p}{,}\PYG{n}{ret}\PYG{p}{)}\PYG{p}{;}
\PYG{+w}{ }\PYG{k}{if}\PYG{+w}{ }\PYG{p}{(}\PYG{n}{ret}\PYG{+w}{ }\PYG{o}{=}\PYG{o}{=}\PYG{+w}{ }\PYG{l+m+mi}{0}\PYG{p}{)}\PYG{p}{\PYGZob{}}
\PYG{+w}{        }\PYG{n}{data}\PYG{+w}{ }\PYG{o}{=}\PYG{+w}{ }\PYG{l+m+mi}{2}\PYG{p}{;}
\PYG{+w}{        }\PYG{k}{while}\PYG{p}{(}\PYG{n}{i}\PYG{+w}{ }\PYG{o}{!}\PYG{o}{=}\PYG{+w}{ }\PYG{l+m+mi}{0}\PYG{p}{)}\PYG{p}{\PYGZob{}}
\PYG{+w}{            }\PYG{n}{i}\PYG{+w}{ }\PYG{o}{\PYGZhy{}}\PYG{o}{\PYGZhy{}}\PYG{p}{;}
\PYG{+w}{            }\PYG{n}{printf}\PYG{p}{(}\PYG{l+s}{\PYGZdq{}}\PYG{l+s}{child Process: \PYGZpc{}d, \PYGZpc{}d;}\PYG{l+s+se}{\PYGZbs{}n}\PYG{l+s}{\PYGZdq{}}\PYG{p}{,}\PYG{+w}{ }\PYG{n}{data}\PYG{p}{,}\PYG{+w}{ }\PYG{n}{i}\PYG{p}{)}\PYG{p}{;}
\PYG{+w}{            }\PYG{n}{sleep}\PYG{p}{(}\PYG{l+m+mi}{4}\PYG{p}{)}\PYG{p}{;}
\PYG{+w}{        }\PYG{p}{\PYGZcb{}}
\PYG{+w}{        }\PYG{n}{exit}\PYG{p}{(}\PYG{l+m+mi}{1}\PYG{p}{)}\PYG{p}{;}
\PYG{+w}{    }\PYG{p}{\PYGZcb{}}
\PYG{+w}{   }\PYG{k}{else}\PYG{+w}{ }\PYG{k}{if}\PYG{p}{(}\PYG{n}{ret}\PYG{+w}{ }\PYG{o}{!}\PYG{o}{=}\PYG{+w}{ }\PYG{l+m+mi}{\PYGZhy{}1}\PYG{p}{)}\PYG{p}{\PYGZob{}}
\PYG{+w}{     }\PYG{n}{pthread\PYGZus{}initial}\PYG{p}{(}\PYG{p}{)}\PYG{p}{;}
\PYG{+w}{     }\PYG{n}{test}\PYG{p}{(}\PYG{p}{)}\PYG{p}{;}
\PYG{+w}{     }\PYG{n}{printf}\PYG{p}{(}\PYG{l+s}{\PYGZdq{}}\PYG{l+s}{lab3 success}\PYG{l+s+se}{\PYGZbs{}n}\PYG{l+s}{\PYGZdq{}}\PYG{p}{)}\PYG{p}{;}
\PYG{+w}{    }\PYG{p}{\PYGZcb{}}
\PYG{+w}{    }
\PYG{+w}{    }\PYG{k}{while}\PYG{p}{(}\PYG{l+m+mi}{1}\PYG{p}{)}\PYG{p}{\PYGZob{}}
\PYG{+w}{    }\PYG{p}{\PYGZcb{}}
\PYG{+w}{    }\PYG{k}{return}\PYG{+w}{ }\PYG{l+m+mi}{0}\PYG{p}{;}
\PYG{p}{\PYGZcb{}}
\end{sphinxVerbatim}


\subsubsection{pthreadtest}
\label{\detokenize{nju:pthreadtest}}
\begin{sphinxVerbatim}[commandchars=\\\{\}]
\PYG{c+cp}{\PYGZsh{}}\PYG{c+cp}{include}\PYG{+w}{ }\PYG{c+cpf}{\PYGZdq{}kernel/param.h\PYGZdq{}}
\PYG{c+cp}{\PYGZsh{}}\PYG{c+cp}{include}\PYG{+w}{ }\PYG{c+cpf}{\PYGZdq{}kernel/types.h\PYGZdq{}}
\PYG{c+cp}{\PYGZsh{}}\PYG{c+cp}{include}\PYG{+w}{ }\PYG{c+cpf}{\PYGZdq{}kernel/stat.h\PYGZdq{}}
\PYG{c+cp}{\PYGZsh{}}\PYG{c+cp}{include}\PYG{+w}{ }\PYG{c+cpf}{\PYGZdq{}user/user.h\PYGZdq{}}
\PYG{c+cp}{\PYGZsh{}}\PYG{c+cp}{include}\PYG{+w}{ }\PYG{c+cpf}{\PYGZdq{}kernel/fs.h\PYGZdq{}}
\PYG{c+cp}{\PYGZsh{}}\PYG{c+cp}{include}\PYG{+w}{ }\PYG{c+cpf}{\PYGZdq{}kernel/fcntl.h\PYGZdq{}}
\PYG{c+cp}{\PYGZsh{}}\PYG{c+cp}{include}\PYG{+w}{ }\PYG{c+cpf}{\PYGZdq{}kernel/syscall.h\PYGZdq{}}
\PYG{c+cp}{\PYGZsh{}}\PYG{c+cp}{include}\PYG{+w}{ }\PYG{c+cpf}{\PYGZdq{}kernel/memlayout.h\PYGZdq{}}
\PYG{c+cp}{\PYGZsh{}}\PYG{c+cp}{include}\PYG{+w}{ }\PYG{c+cpf}{\PYGZdq{}kernel/loongarch.h\PYGZdq{}}
\PYG{c+cp}{\PYGZsh{}}\PYG{c+cp}{include}\PYG{+w}{ }\PYG{c+cpf}{\PYGZdq{}pthread.h\PYGZdq{}}
\PYG{c+cp}{\PYGZsh{}}\PYG{c+cp}{include}\PYG{+w}{ }\PYG{c+cpf}{\PYGZdq{}pthreadtest.h\PYGZdq{}}
\PYG{c+cp}{\PYGZsh{}}\PYG{c+cp}{include}\PYG{+w}{ }\PYG{c+cpf}{\PYGZlt{}stddef.h\PYGZgt{}}
\PYG{k+kt}{int}\PYG{+w}{ }\PYG{n}{gi}\PYG{+w}{ }\PYG{o}{=}\PYG{+w}{ }\PYG{l+m+mi}{0}\PYG{p}{;}

\PYG{k+kt}{void}\PYG{+w}{ }\PYG{o}{*}\PYG{+w}{ }\PYG{n+nf}{ping\PYGZus{}thread\PYGZus{}function}\PYG{p}{(}\PYG{k+kt}{void}\PYG{+w}{ }\PYG{o}{*}\PYG{n}{arg}\PYG{p}{)}\PYG{p}{\PYGZob{}}
\PYG{+w}{    }\PYG{k+kt}{int}\PYG{+w}{ }\PYG{n}{i}\PYG{p}{;}
\PYG{+w}{    }\PYG{k}{for}\PYG{+w}{ }\PYG{p}{(}\PYG{n}{i}\PYG{+w}{ }\PYG{o}{=}\PYG{+w}{ }\PYG{l+m+mi}{0}\PYG{p}{;}\PYG{+w}{ }\PYG{n}{i}\PYG{+w}{ }\PYG{o}{\PYGZlt{}}\PYG{+w}{ }\PYG{l+m+mi}{5}\PYG{p}{;}\PYG{+w}{ }\PYG{n}{i}\PYG{o}{+}\PYG{o}{+}\PYG{p}{)}\PYG{p}{\PYGZob{}}
\PYG{+w}{        }\PYG{n}{gi}\PYG{o}{+}\PYG{o}{+}\PYG{p}{;}
\PYG{+w}{        }\PYG{n}{printf}\PYG{p}{(}\PYG{l+s}{\PYGZdq{}}\PYG{l+s}{Ping@\PYGZpc{}d\PYGZhy{}\PYGZpc{}d}\PYG{l+s+se}{\PYGZbs{}n}\PYG{l+s}{\PYGZdq{}}\PYG{p}{,}\PYG{+w}{ }\PYG{n}{gi}\PYG{p}{,}\PYG{+w}{ }\PYG{o}{*}\PYG{p}{(}\PYG{k+kt}{int}\PYG{+w}{ }\PYG{o}{*}\PYG{p}{)}\PYG{n}{arg}\PYG{p}{)}\PYG{p}{;}
\PYG{+w}{        }\PYG{n}{sleep}\PYG{p}{(}\PYG{l+m+mi}{10}\PYG{p}{)}\PYG{p}{;}
\PYG{+w}{        }\PYG{n}{pthread\PYGZus{}yield}\PYG{p}{(}\PYG{p}{)}\PYG{p}{;}
\PYG{+w}{    }\PYG{p}{\PYGZcb{}}
\PYG{+w}{    }\PYG{n}{pthread\PYGZus{}exit}\PYG{p}{(}\PYG{n+nb}{NULL}\PYG{p}{)}\PYG{p}{;}
\PYG{+w}{    }\PYG{k}{return}\PYG{+w}{ }\PYG{n+nb}{NULL}\PYG{p}{;}
\PYG{p}{\PYGZcb{}}

\PYG{k+kt}{void}\PYG{+w}{ }\PYG{o}{*}\PYG{+w}{ }\PYG{n+nf}{pong\PYGZus{}thread\PYGZus{}function}\PYG{p}{(}\PYG{k+kt}{void}\PYG{+w}{ }\PYG{o}{*}\PYG{n}{arg}\PYG{p}{)}\PYG{p}{\PYGZob{}}
\PYG{+w}{    }\PYG{k+kt}{int}\PYG{+w}{ }\PYG{n}{i}\PYG{p}{;}
\PYG{+w}{    }\PYG{k}{for}\PYG{+w}{ }\PYG{p}{(}\PYG{n}{i}\PYG{+w}{ }\PYG{o}{=}\PYG{+w}{ }\PYG{l+m+mi}{0}\PYG{p}{;}\PYG{+w}{ }\PYG{n}{i}\PYG{+w}{ }\PYG{o}{\PYGZlt{}}\PYG{+w}{ }\PYG{l+m+mi}{5}\PYG{p}{;}\PYG{+w}{ }\PYG{n}{i}\PYG{o}{+}\PYG{o}{+}\PYG{p}{)}\PYG{p}{\PYGZob{}}
\PYG{+w}{        }\PYG{n}{gi}\PYG{o}{+}\PYG{o}{+}\PYG{p}{;}
\PYG{+w}{        }\PYG{n}{printf}\PYG{p}{(}\PYG{l+s}{\PYGZdq{}}\PYG{l+s}{Pong@\PYGZpc{}d}\PYG{l+s+se}{\PYGZbs{}n}\PYG{l+s}{\PYGZdq{}}\PYG{p}{,}\PYG{+w}{ }\PYG{n}{gi}\PYG{p}{)}\PYG{p}{;}
\PYG{+w}{        }\PYG{n}{sleep}\PYG{p}{(}\PYG{l+m+mi}{10}\PYG{p}{)}\PYG{p}{;}
\PYG{+w}{        }\PYG{n}{pthread\PYGZus{}yield}\PYG{p}{(}\PYG{p}{)}\PYG{p}{;}
\PYG{+w}{    }\PYG{p}{\PYGZcb{}}
\PYG{+w}{    }\PYG{n}{pthread\PYGZus{}exit}\PYG{p}{(}\PYG{n+nb}{NULL}\PYG{p}{)}\PYG{p}{;}
\PYG{+w}{    }\PYG{k}{return}\PYG{+w}{ }\PYG{n+nb}{NULL}\PYG{p}{;}
\PYG{p}{\PYGZcb{}}

\PYG{k+kt}{int}\PYG{+w}{ }\PYG{n+nf}{test}\PYG{p}{(}\PYG{k+kt}{void}\PYG{p}{)}\PYG{+w}{ }\PYG{p}{\PYGZob{}}
\PYG{+w}{    }\PYG{k}{static}\PYG{+w}{ }\PYG{k+kt}{int}\PYG{+w}{ }\PYG{n}{a}\PYG{+w}{ }\PYG{o}{=}\PYG{+w}{ }\PYG{l+m+mi}{1}\PYG{p}{;}
\PYG{+w}{    }\PYG{k}{static}\PYG{+w}{ }\PYG{k+kt}{int}\PYG{+w}{ }\PYG{n}{b}\PYG{+w}{ }\PYG{o}{=}\PYG{+w}{ }\PYG{l+m+mi}{2}\PYG{p}{;}
\PYG{+w}{    }\PYG{n}{uint32}\PYG{+w}{ }\PYG{n}{pi1\PYGZus{}thread\PYGZus{}ID}\PYG{p}{,}\PYG{+w}{ }\PYG{n}{pi2\PYGZus{}thread\PYGZus{}ID}\PYG{p}{,}\PYG{+w}{ }\PYG{n}{po\PYGZus{}thread\PYGZus{}ID}\PYG{p}{;}
\PYG{+w}{    }\PYG{n}{pthread\PYGZus{}create}\PYG{p}{(}\PYG{o}{\PYGZam{}}\PYG{n}{pi1\PYGZus{}thread\PYGZus{}ID}\PYG{p}{,}\PYG{+w}{ }\PYG{n+nb}{NULL}\PYG{p}{,}\PYG{+w}{ }\PYG{n}{ping\PYGZus{}thread\PYGZus{}function}\PYG{p}{,}\PYG{+w}{ }\PYG{o}{\PYGZam{}}\PYG{n}{a}\PYG{p}{)}\PYG{p}{;}
\PYG{+w}{    }\PYG{n}{pthread\PYGZus{}create}\PYG{p}{(}\PYG{o}{\PYGZam{}}\PYG{n}{pi2\PYGZus{}thread\PYGZus{}ID}\PYG{p}{,}\PYG{+w}{ }\PYG{n+nb}{NULL}\PYG{p}{,}\PYG{+w}{ }\PYG{n}{ping\PYGZus{}thread\PYGZus{}function}\PYG{p}{,}\PYG{+w}{ }\PYG{o}{\PYGZam{}}\PYG{n}{b}\PYG{p}{)}\PYG{p}{;}
\PYG{+w}{    }\PYG{n}{pthread\PYGZus{}create}\PYG{p}{(}\PYG{o}{\PYGZam{}}\PYG{n}{po\PYGZus{}thread\PYGZus{}ID}\PYG{p}{,}\PYG{+w}{ }\PYG{n+nb}{NULL}\PYG{p}{,}\PYG{+w}{ }\PYG{n}{pong\PYGZus{}thread\PYGZus{}function}\PYG{p}{,}\PYG{+w}{ }\PYG{n+nb}{NULL}\PYG{p}{)}\PYG{p}{;}
\PYG{+w}{    }\PYG{n}{pthread\PYGZus{}join}\PYG{p}{(}\PYG{n}{pi1\PYGZus{}thread\PYGZus{}ID}\PYG{p}{,}\PYG{+w}{ }\PYG{n+nb}{NULL}\PYG{p}{)}\PYG{p}{;}
\PYG{+w}{    }\PYG{n}{pthread\PYGZus{}join}\PYG{p}{(}\PYG{n}{pi2\PYGZus{}thread\PYGZus{}ID}\PYG{p}{,}\PYG{+w}{ }\PYG{n+nb}{NULL}\PYG{p}{)}\PYG{p}{;}
\PYG{+w}{    }\PYG{n}{pthread\PYGZus{}join}\PYG{p}{(}\PYG{n}{po\PYGZus{}thread\PYGZus{}ID}\PYG{p}{,}\PYG{+w}{ }\PYG{n+nb}{NULL}\PYG{p}{)}\PYG{p}{;}
\PYG{+w}{    }
\PYG{+w}{    }\PYG{c+c1}{//while(1)\PYGZob{}\PYGZcb{}}
\PYG{+w}{    }\PYG{k}{return}\PYG{+w}{ }\PYG{l+m+mi}{0}\PYG{p}{;}
\PYG{p}{\PYGZcb{}}
\end{sphinxVerbatim}

\sphinxAtStartPar
该测试是一个多线程同步测试程序,实现了经典的”Ping\sphinxhyphen{}Pong”线程交互模式,创建了3个线程来竞争访问和修改一个共享的全局变量 gi。\\
在main函数则是子进程执行循环计数,循环8次,每次睡眠4ticks,父进程初始化线程库并进行pthreadtest测试。

\sphinxAtStartPar
这里需要修改Makefile文件,将pthreadtest.c链接到init.c中。

\begin{sphinxVerbatim}[commandchars=\\\{\}]
ULIB = \PYGZdl{}U/ulib.o \PYGZdl{}U/usys.o \PYGZdl{}U/printf.o \PYGZdl{}U/umalloc.o 
\PYGZdl{}U/\PYGZus{}init: \PYGZdl{}U/init.o \PYGZdl{}U/pthreadtest.o \PYGZdl{}U/pthread.o \PYGZdl{}(ULIB)
\end{sphinxVerbatim}


\section{实验结果}
\label{\detokenize{nju:id17}}
\sphinxAtStartPar
\sphinxincludegraphics{{img4}.png}


\chapter{lab3}
\label{\detokenize{nju:lab3}}

\section{实验目的}
\label{\detokenize{nju:id18}}\begin{enumerate}
\sphinxsetlistlabels{\arabic}{enumi}{enumii}{}{.}%
\item {} 
\sphinxAtStartPar
实现一个简单的生产者消费者程序。

\item {} 
\sphinxAtStartPar
介绍基于信号量的进程同步机制。

\end{enumerate}


\section{实验内容}
\label{\detokenize{nju:id19}}\begin{enumerate}
\sphinxsetlistlabels{\arabic}{enumi}{enumii}{}{.}%
\item {} 
\sphinxAtStartPar
内核:提供基于信号量的进程同步机制,并提供系统调用 sem\_init、sem\_post、
sem\_wait、sem\_destroy。

\item {} 
\sphinxAtStartPar
库:对上述系统调用进行封装。

\item {} 
\sphinxAtStartPar
用户:对上述库函数进行测试。

\end{enumerate}


\section{实验过程}
\label{\detokenize{nju:id20}}

\subsection{内核态}
\label{\detokenize{nju:id21}}
\begin{sphinxVerbatim}[commandchars=\\\{\}]
\PYG{k+kt}{void}
\PYG{n+nf}{semsinit}\PYG{p}{(}\PYG{k+kt}{void}\PYG{p}{)}
\PYG{p}{\PYGZob{}}
\PYG{+w}{    }\PYG{n}{initlock}\PYG{p}{(}\PYG{o}{\PYGZam{}}\PYG{n}{semslock}\PYG{p}{,}\PYG{+w}{ }\PYG{l+s}{\PYGZdq{}}\PYG{l+s}{sems}\PYG{l+s}{\PYGZdq{}}\PYG{p}{)}\PYG{p}{;}
\PYG{+w}{    }\PYG{k}{for}\PYG{p}{(}\PYG{k+kt}{int}\PYG{+w}{ }\PYG{n}{i}\PYG{+w}{ }\PYG{o}{=}\PYG{+w}{ }\PYG{l+m+mi}{0}\PYG{p}{;}\PYG{+w}{ }\PYG{n}{i}\PYG{+w}{ }\PYG{o}{\PYGZlt{}}\PYG{+w}{ }\PYG{n}{MAX\PYGZus{}SEM\PYGZus{}NUM}\PYG{p}{;}\PYG{+w}{ }\PYG{n}{i}\PYG{o}{+}\PYG{o}{+}\PYG{p}{)}\PYG{+w}{ }\PYG{p}{\PYGZob{}}
\PYG{+w}{        }\PYG{n}{initlock}\PYG{p}{(}\PYG{o}{\PYGZam{}}\PYG{n}{sem}\PYG{p}{[}\PYG{n}{i}\PYG{p}{]}\PYG{p}{.}\PYG{n}{lock}\PYG{p}{,}\PYG{+w}{ }\PYG{l+s}{\PYGZdq{}}\PYG{l+s}{sem}\PYG{l+s}{\PYGZdq{}}\PYG{p}{)}\PYG{p}{;}
\PYG{+w}{        }\PYG{n}{sem}\PYG{p}{[}\PYG{n}{i}\PYG{p}{]}\PYG{p}{.}\PYG{n}{valid}\PYG{+w}{ }\PYG{o}{=}\PYG{+w}{ }\PYG{l+m+mi}{0}\PYG{p}{;}\PYG{+w}{      }\PYG{c+c1}{// 标记为无效}
\PYG{+w}{        }\PYG{n}{sem}\PYG{p}{[}\PYG{n}{i}\PYG{p}{]}\PYG{p}{.}\PYG{n}{value}\PYG{+w}{ }\PYG{o}{=}\PYG{+w}{ }\PYG{l+m+mi}{0}\PYG{p}{;}
\PYG{+w}{        }\PYG{n}{sem}\PYG{p}{[}\PYG{n}{i}\PYG{p}{]}\PYG{p}{.}\PYG{n}{waiting}\PYG{+w}{ }\PYG{o}{=}\PYG{+w}{ }\PYG{l+m+mi}{0}\PYG{p}{;}
\PYG{+w}{    }\PYG{p}{\PYGZcb{}}
\PYG{p}{\PYGZcb{}}
\end{sphinxVerbatim}

\sphinxAtStartPar
新增seminit函数用来初始化信号量。


\subsection{用户态}
\label{\detokenize{nju:id22}}

\subsubsection{新增系统调用步骤}
\label{\detokenize{nju:id23}}
\sphinxAtStartPar
以sys\_seminit系统调用为例:


\paragraph{增加用户入口}
\label{\detokenize{nju:id24}}
\sphinxAtStartPar
在user/usys.S中添加:

\begin{sphinxVerbatim}[commandchars=\\\{\}]
.global seminit
seminit:
 li.d \PYGZdl{}a7, SYS\PYGZus{}seminit
 syscall 0
 jirl \PYGZdl{}zero, \PYGZdl{}ra, 0
\end{sphinxVerbatim}

\sphinxAtStartPar
修改user.h
在//system calls注释下添加函数原型声明:

\begin{sphinxVerbatim}[commandchars=\\\{\}]
\PYG{n+nb}{int} \PYG{n}{seminit}\PYG{p}{(}\PYG{n+nb}{int} \PYG{n}{value}\PYG{p}{)}\PYG{p}{;}
\end{sphinxVerbatim}


\paragraph{增加系统调用号}
\label{\detokenize{nju:id25}}
\sphinxAtStartPar
在 kernel/syscall.h 中。我们可以在 SYS\_close 的后面,新加入一行:

\begin{sphinxVerbatim}[commandchars=\\\{\}]
\PYG{c+c1}{\PYGZsh{}define SYS\PYGZus{}seminit    23}
\end{sphinxVerbatim}


\paragraph{修改 syscall.c 中的跳转表}
\label{\detokenize{nju:syscall-c}}
\sphinxAtStartPar
在 kernel/syscall.c中。在{[}SYS\_close{]}   sys\_close后添加一行:

\begin{sphinxVerbatim}[commandchars=\\\{\}]
\PYG{p}{[}\PYG{n}{SYS\PYGZus{}seminit}\PYG{p}{]}    \PYG{n}{sys\PYGZus{}seminit}\PYG{p}{,}
\end{sphinxVerbatim}


\paragraph{实现 sys\_seminit()}
\label{\detokenize{nju:sys-seminit}}
\sphinxAtStartPar
见之后的小结。


\subsubsection{sys\_seminit 系统调用}
\label{\detokenize{nju:id26}}
\begin{sphinxVerbatim}[commandchars=\\\{\}]
\PYG{n}{uint64}
\PYG{n+nf}{sys\PYGZus{}seminit}\PYG{p}{(}\PYG{k+kt}{void}\PYG{p}{)}
\PYG{p}{\PYGZob{}}
\PYG{+w}{    }\PYG{k+kt}{int}\PYG{+w}{ }\PYG{n}{value}\PYG{p}{;}
\PYG{+w}{    }\PYG{k+kt}{int}\PYG{+w}{ }\PYG{n}{semid}\PYG{p}{;}
\PYG{+w}{    }
\PYG{+w}{    }\PYG{k}{if}\PYG{p}{(}\PYG{n}{argint}\PYG{p}{(}\PYG{l+m+mi}{0}\PYG{p}{,}\PYG{+w}{ }\PYG{o}{\PYGZam{}}\PYG{n}{value}\PYG{p}{)}\PYG{+w}{ }\PYG{o}{\PYGZlt{}}\PYG{+w}{ }\PYG{l+m+mi}{0}\PYG{p}{)}
\PYG{+w}{        }\PYG{k}{return}\PYG{+w}{ }\PYG{l+m+mi}{\PYGZhy{}1}\PYG{p}{;}
\PYG{+w}{    }
\PYG{+w}{    }\PYG{k}{if}\PYG{p}{(}\PYG{n}{value}\PYG{+w}{ }\PYG{o}{\PYGZlt{}}\PYG{+w}{ }\PYG{l+m+mi}{0}\PYG{+w}{ }\PYG{o}{|}\PYG{o}{|}\PYG{+w}{ }\PYG{n}{value}\PYG{+w}{ }\PYG{o}{\PYGZgt{}}\PYG{+w}{ }\PYG{n}{SEM\PYGZus{}VALUE\PYGZus{}MAX}\PYG{p}{)}
\PYG{+w}{        }\PYG{k}{return}\PYG{+w}{ }\PYG{l+m+mi}{\PYGZhy{}1}\PYG{p}{;}
\PYG{+w}{    }
\PYG{+w}{    }\PYG{n}{semid}\PYG{+w}{ }\PYG{o}{=}\PYG{+w}{ }\PYG{n}{semalloc}\PYG{p}{(}\PYG{n}{value}\PYG{p}{)}\PYG{p}{;}
\PYG{+w}{    }\PYG{k}{return}\PYG{+w}{ }\PYG{n}{semid}\PYG{p}{;}
\PYG{p}{\PYGZcb{}}
\end{sphinxVerbatim}

\begin{sphinxVerbatim}[commandchars=\\\{\}]
\PYG{k+kt}{int}
\PYG{n+nf}{semalloc}\PYG{p}{(}\PYG{k+kt}{int}\PYG{+w}{ }\PYG{n}{value}\PYG{p}{)}
\PYG{p}{\PYGZob{}}
\PYG{+w}{    }\PYG{k+kt}{int}\PYG{+w}{ }\PYG{n}{i}\PYG{o}{=}\PYG{l+m+mi}{\PYGZhy{}1}\PYG{p}{;}
\PYG{+w}{    }
\PYG{+w}{    }\PYG{n}{acquire}\PYG{p}{(}\PYG{o}{\PYGZam{}}\PYG{n}{semslock}\PYG{p}{)}\PYG{p}{;}
\PYG{+w}{    }\PYG{k}{for}\PYG{p}{(}\PYG{n}{i}\PYG{+w}{ }\PYG{o}{=}\PYG{+w}{ }\PYG{l+m+mi}{0}\PYG{p}{;}\PYG{+w}{ }\PYG{n}{i}\PYG{+w}{ }\PYG{o}{\PYGZlt{}}\PYG{+w}{ }\PYG{n}{MAX\PYGZus{}SEM\PYGZus{}NUM}\PYG{p}{;}\PYG{+w}{ }\PYG{n}{i}\PYG{o}{+}\PYG{o}{+}\PYG{p}{)}\PYG{+w}{ }\PYG{p}{\PYGZob{}}
\PYG{+w}{        }\PYG{k}{if}\PYG{p}{(}\PYG{n}{sem}\PYG{p}{[}\PYG{n}{i}\PYG{p}{]}\PYG{p}{.}\PYG{n}{valid}\PYG{+w}{ }\PYG{o}{=}\PYG{o}{=}\PYG{+w}{ }\PYG{l+m+mi}{0}\PYG{p}{)}\PYG{+w}{ }\PYG{p}{\PYGZob{}}
\PYG{+w}{            }\PYG{n}{sem}\PYG{p}{[}\PYG{n}{i}\PYG{p}{]}\PYG{p}{.}\PYG{n}{valid}\PYG{+w}{ }\PYG{o}{=}\PYG{+w}{ }\PYG{l+m+mi}{1}\PYG{p}{;}
\PYG{+w}{            }\PYG{n}{sem}\PYG{p}{[}\PYG{n}{i}\PYG{p}{]}\PYG{p}{.}\PYG{n}{value}\PYG{+w}{ }\PYG{o}{=}\PYG{+w}{ }\PYG{n}{value}\PYG{p}{;}
\PYG{+w}{            }\PYG{n}{sem}\PYG{p}{[}\PYG{n}{i}\PYG{p}{]}\PYG{p}{.}\PYG{n}{waiting}\PYG{+w}{ }\PYG{o}{=}\PYG{+w}{ }\PYG{l+m+mi}{0}\PYG{p}{;}
\PYG{+w}{            }\PYG{n}{release}\PYG{p}{(}\PYG{o}{\PYGZam{}}\PYG{n}{semslock}\PYG{p}{)}\PYG{p}{;}
\PYG{+w}{            }\PYG{k}{return}\PYG{+w}{ }\PYG{n}{i}\PYG{p}{;}
\PYG{+w}{        }\PYG{p}{\PYGZcb{}}
\PYG{+w}{    }\PYG{p}{\PYGZcb{}}
\PYG{+w}{    }\PYG{n}{release}\PYG{p}{(}\PYG{o}{\PYGZam{}}\PYG{n}{semslock}\PYG{p}{)}\PYG{p}{;}
\PYG{+w}{    }\PYG{k}{return}\PYG{+w}{ }\PYG{l+m+mi}{\PYGZhy{}1}\PYG{p}{;}
\PYG{p}{\PYGZcb{}}
\end{sphinxVerbatim}

\sphinxAtStartPar
sys\_seminit 系统调用用于分配一个信号量,将valid设为1代表已使用,通过argint得到需要赋予的value值,初始化等待队列计数器。


\subsubsection{sys\_sempost 系统调用}
\label{\detokenize{nju:sys-sempost}}
\begin{sphinxVerbatim}[commandchars=\\\{\}]
\PYG{n}{uint64}
\PYG{n+nf}{sys\PYGZus{}sempost}\PYG{p}{(}\PYG{k+kt}{void}\PYG{p}{)}
\PYG{p}{\PYGZob{}}
\PYG{+w}{    }\PYG{k+kt}{int}\PYG{+w}{ }\PYG{n}{semid}\PYG{p}{;}
\PYG{+w}{    }
\PYG{+w}{    }\PYG{k}{if}\PYG{p}{(}\PYG{n}{argint}\PYG{p}{(}\PYG{l+m+mi}{0}\PYG{p}{,}\PYG{+w}{ }\PYG{o}{\PYGZam{}}\PYG{n}{semid}\PYG{p}{)}\PYG{+w}{ }\PYG{o}{\PYGZlt{}}\PYG{+w}{ }\PYG{l+m+mi}{0}\PYG{p}{)}
\PYG{+w}{        }\PYG{k}{return}\PYG{+w}{ }\PYG{l+m+mi}{\PYGZhy{}1}\PYG{p}{;}
\PYG{+w}{    }
\PYG{+w}{    }\PYG{k}{if}\PYG{p}{(}\PYG{n}{semid}\PYG{+w}{ }\PYG{o}{\PYGZlt{}}\PYG{+w}{ }\PYG{l+m+mi}{0}\PYG{+w}{ }\PYG{o}{|}\PYG{o}{|}\PYG{+w}{ }\PYG{n}{semid}\PYG{+w}{ }\PYG{o}{\PYGZgt{}}\PYG{o}{=}\PYG{+w}{ }\PYG{n}{MAX\PYGZus{}SEM\PYGZus{}NUM}\PYG{p}{)}
\PYG{+w}{        }\PYG{k}{return}\PYG{+w}{ }\PYG{l+m+mi}{\PYGZhy{}1}\PYG{p}{;}
\PYG{+w}{    }
\PYG{+w}{    }\PYG{n}{sempost}\PYG{p}{(}\PYG{n}{semid}\PYG{p}{)}\PYG{p}{;}
\PYG{+w}{    }\PYG{k}{return}\PYG{+w}{ }\PYG{l+m+mi}{0}\PYG{p}{;}
\PYG{p}{\PYGZcb{}}
\end{sphinxVerbatim}

\begin{sphinxVerbatim}[commandchars=\\\{\}]
\PYG{k+kt}{void}
\PYG{n+nf}{sempost}\PYG{p}{(}\PYG{k+kt}{int}\PYG{+w}{ }\PYG{n}{semid}\PYG{p}{)}
\PYG{p}{\PYGZob{}}
\PYG{+w}{    }\PYG{k}{struct}\PYG{+w}{ }\PYG{n+nc}{semaphore}\PYG{+w}{ }\PYG{o}{*}\PYG{n}{s}\PYG{p}{;}
\PYG{+w}{    }\PYG{k}{struct}\PYG{+w}{ }\PYG{n+nc}{proc}\PYG{+w}{ }\PYG{o}{*}\PYG{n}{p}\PYG{p}{;}
\PYG{+w}{    }
\PYG{+w}{    }\PYG{k}{if}\PYG{p}{(}\PYG{n}{semid}\PYG{+w}{ }\PYG{o}{\PYGZlt{}}\PYG{+w}{ }\PYG{l+m+mi}{0}\PYG{+w}{ }\PYG{o}{|}\PYG{o}{|}\PYG{+w}{ }\PYG{n}{semid}\PYG{+w}{ }\PYG{o}{\PYGZgt{}}\PYG{o}{=}\PYG{+w}{ }\PYG{n}{MAX\PYGZus{}SEM\PYGZus{}NUM}\PYG{p}{)}
\PYG{+w}{        }\PYG{k}{return}\PYG{p}{;}
\PYG{+w}{    }
\PYG{+w}{    }\PYG{n}{s}\PYG{+w}{ }\PYG{o}{=}\PYG{+w}{ }\PYG{o}{\PYGZam{}}\PYG{n}{sem}\PYG{p}{[}\PYG{n}{semid}\PYG{p}{]}\PYG{p}{;}
\PYG{+w}{    }\PYG{n}{acquire}\PYG{p}{(}\PYG{o}{\PYGZam{}}\PYG{n}{s}\PYG{o}{\PYGZhy{}}\PYG{o}{\PYGZgt{}}\PYG{n}{lock}\PYG{p}{)}\PYG{p}{;}
\PYG{+w}{    }
\PYG{+w}{    }\PYG{n}{s}\PYG{o}{\PYGZhy{}}\PYG{o}{\PYGZgt{}}\PYG{n}{value}\PYG{o}{+}\PYG{o}{+}\PYG{p}{;}
\PYG{+w}{    }\PYG{k}{if}\PYG{p}{(}\PYG{n}{s}\PYG{o}{\PYGZhy{}}\PYG{o}{\PYGZgt{}}\PYG{n}{value}\PYG{+w}{ }\PYG{o}{\PYGZlt{}}\PYG{o}{=}\PYG{+w}{ }\PYG{l+m+mi}{0}\PYG{p}{)}\PYG{+w}{ }\PYG{p}{\PYGZob{}}
\PYG{+w}{        }\PYG{k}{if}\PYG{p}{(}\PYG{n}{s}\PYG{o}{\PYGZhy{}}\PYG{o}{\PYGZgt{}}\PYG{n}{waiting}\PYG{+w}{ }\PYG{o}{!}\PYG{o}{=}\PYG{+w}{ }\PYG{l+m+mi}{0}\PYG{p}{)}\PYG{+w}{ }\PYG{p}{\PYGZob{}}
\PYG{+w}{            }\PYG{n}{p}\PYG{+w}{ }\PYG{o}{=}\PYG{+w}{ }\PYG{n}{s}\PYG{o}{\PYGZhy{}}\PYG{o}{\PYGZgt{}}\PYG{n}{waiting}\PYG{p}{;}
\PYG{+w}{            }\PYG{n}{s}\PYG{o}{\PYGZhy{}}\PYG{o}{\PYGZgt{}}\PYG{n}{waiting}\PYG{+w}{ }\PYG{o}{=}\PYG{+w}{ }\PYG{n}{p}\PYG{o}{\PYGZhy{}}\PYG{o}{\PYGZgt{}}\PYG{n}{next}\PYG{p}{;}
\PYG{+w}{            }\PYG{n}{p}\PYG{o}{\PYGZhy{}}\PYG{o}{\PYGZgt{}}\PYG{n}{next}\PYG{+w}{ }\PYG{o}{=}\PYG{+w}{ }\PYG{l+m+mi}{0}\PYG{p}{;}\PYG{+w}{  }
\PYG{+w}{            }\PYG{n}{wakeup}\PYG{p}{(}\PYG{n}{s}\PYG{p}{)}\PYG{p}{;}
\PYG{+w}{        }\PYG{p}{\PYGZcb{}}
\PYG{+w}{    }\PYG{p}{\PYGZcb{}}
\PYG{+w}{    }
\PYG{+w}{    }\PYG{n}{release}\PYG{p}{(}\PYG{o}{\PYGZam{}}\PYG{n}{s}\PYG{o}{\PYGZhy{}}\PYG{o}{\PYGZgt{}}\PYG{n}{lock}\PYG{p}{)}\PYG{p}{;}
\PYG{p}{\PYGZcb{}}
\end{sphinxVerbatim}

\sphinxAtStartPar
sem\_post 系统调用对应信号量的 V 操作(释放资源/增加信号量),其使得 sem 指向的信号量的 value 加一,若 value 取值不大于 0,则释放一个阻塞在该信号量上进程(即将该进程设置为就绪态),若操作成功则返回 0,否则返回\sphinxhyphen{}1。
sempost 首先检查参数对应的信号量,否则将value 加一,若 value<=0,唤醒等待队列中的一个进程:waiting != 0,移除队列头部,清空next指针,唤醒进程。


\subsubsection{sys\_semwait 系统调用}
\label{\detokenize{nju:sys-semwait}}
\begin{sphinxVerbatim}[commandchars=\\\{\}]
\PYG{n}{uint64}
\PYG{n+nf}{sys\PYGZus{}semwait}\PYG{p}{(}\PYG{k+kt}{void}\PYG{p}{)}
\PYG{p}{\PYGZob{}}
\PYG{+w}{    }\PYG{k+kt}{int}\PYG{+w}{ }\PYG{n}{semid}\PYG{p}{;}
\PYG{+w}{    }
\PYG{+w}{    }\PYG{k}{if}\PYG{p}{(}\PYG{n}{argint}\PYG{p}{(}\PYG{l+m+mi}{0}\PYG{p}{,}\PYG{+w}{ }\PYG{o}{\PYGZam{}}\PYG{n}{semid}\PYG{p}{)}\PYG{+w}{ }\PYG{o}{\PYGZlt{}}\PYG{+w}{ }\PYG{l+m+mi}{0}\PYG{p}{)}
\PYG{+w}{        }\PYG{k}{return}\PYG{+w}{ }\PYG{l+m+mi}{\PYGZhy{}1}\PYG{p}{;}
\PYG{+w}{    }
\PYG{+w}{    }\PYG{k}{if}\PYG{p}{(}\PYG{n}{semid}\PYG{+w}{ }\PYG{o}{\PYGZlt{}}\PYG{+w}{ }\PYG{l+m+mi}{0}\PYG{+w}{ }\PYG{o}{|}\PYG{o}{|}\PYG{+w}{ }\PYG{n}{semid}\PYG{+w}{ }\PYG{o}{\PYGZgt{}}\PYG{o}{=}\PYG{+w}{ }\PYG{n}{MAX\PYGZus{}SEM\PYGZus{}NUM}\PYG{p}{)}
\PYG{+w}{        }\PYG{k}{return}\PYG{+w}{ }\PYG{l+m+mi}{\PYGZhy{}1}\PYG{p}{;}
\PYG{+w}{    }
\PYG{+w}{    }\PYG{n}{semwait}\PYG{p}{(}\PYG{n}{semid}\PYG{p}{)}\PYG{p}{;}
\PYG{+w}{    }\PYG{k}{return}\PYG{+w}{ }\PYG{l+m+mi}{0}\PYG{p}{;}
\PYG{p}{\PYGZcb{}}
\end{sphinxVerbatim}

\begin{sphinxVerbatim}[commandchars=\\\{\}]
\PYG{k+kt}{void}
\PYG{n+nf}{semwait}\PYG{p}{(}\PYG{k+kt}{int}\PYG{+w}{ }\PYG{n}{semid}\PYG{p}{)}
\PYG{p}{\PYGZob{}}
\PYG{+w}{    }\PYG{k}{struct}\PYG{+w}{ }\PYG{n+nc}{semaphore}\PYG{+w}{ }\PYG{o}{*}\PYG{n}{s}\PYG{p}{;}
\PYG{+w}{    }\PYG{k}{struct}\PYG{+w}{ }\PYG{n+nc}{proc}\PYG{+w}{ }\PYG{o}{*}\PYG{n}{p}\PYG{+w}{ }\PYG{o}{=}\PYG{+w}{ }\PYG{n}{myproc}\PYG{p}{(}\PYG{p}{)}\PYG{p}{;}
\PYG{+w}{    }
\PYG{+w}{    }\PYG{k}{if}\PYG{p}{(}\PYG{n}{semid}\PYG{+w}{ }\PYG{o}{\PYGZlt{}}\PYG{+w}{ }\PYG{l+m+mi}{0}\PYG{+w}{ }\PYG{o}{|}\PYG{o}{|}\PYG{+w}{ }\PYG{n}{semid}\PYG{+w}{ }\PYG{o}{\PYGZgt{}}\PYG{o}{=}\PYG{+w}{ }\PYG{n}{MAX\PYGZus{}SEM\PYGZus{}NUM}\PYG{p}{)}
\PYG{+w}{        }\PYG{k}{return}\PYG{p}{;}
\PYG{+w}{    }
\PYG{+w}{    }\PYG{n}{s}\PYG{+w}{ }\PYG{o}{=}\PYG{+w}{ }\PYG{o}{\PYGZam{}}\PYG{n}{sem}\PYG{p}{[}\PYG{n}{semid}\PYG{p}{]}\PYG{p}{;}
\PYG{+w}{    }\PYG{n}{acquire}\PYG{p}{(}\PYG{o}{\PYGZam{}}\PYG{n}{s}\PYG{o}{\PYGZhy{}}\PYG{o}{\PYGZgt{}}\PYG{n}{lock}\PYG{p}{)}\PYG{p}{;}
\PYG{+w}{    }
\PYG{+w}{    }\PYG{n}{s}\PYG{o}{\PYGZhy{}}\PYG{o}{\PYGZgt{}}\PYG{n}{value}\PYG{o}{\PYGZhy{}}\PYG{o}{\PYGZhy{}}\PYG{p}{;}
\PYG{+w}{    }\PYG{k}{if}\PYG{p}{(}\PYG{n}{s}\PYG{o}{\PYGZhy{}}\PYG{o}{\PYGZgt{}}\PYG{n}{value}\PYG{+w}{ }\PYG{o}{\PYGZlt{}}\PYG{+w}{ }\PYG{l+m+mi}{0}\PYG{p}{)}\PYG{+w}{ }\PYG{p}{\PYGZob{}}
\PYG{+w}{        }\PYG{n}{p}\PYG{o}{\PYGZhy{}}\PYG{o}{\PYGZgt{}}\PYG{n}{chan}\PYG{+w}{ }\PYG{o}{=}\PYG{+w}{ }\PYG{n}{s}\PYG{p}{;}\PYG{+w}{  }
\PYG{+w}{        }\PYG{n}{p}\PYG{o}{\PYGZhy{}}\PYG{o}{\PYGZgt{}}\PYG{n}{next}\PYG{+w}{ }\PYG{o}{=}\PYG{+w}{ }\PYG{n}{s}\PYG{o}{\PYGZhy{}}\PYG{o}{\PYGZgt{}}\PYG{n}{waiting}\PYG{p}{;}
\PYG{+w}{        }\PYG{n}{s}\PYG{o}{\PYGZhy{}}\PYG{o}{\PYGZgt{}}\PYG{n}{waiting}\PYG{+w}{ }\PYG{o}{=}\PYG{+w}{ }\PYG{n}{p}\PYG{p}{;}
\PYG{+w}{        }\PYG{n}{sleep}\PYG{p}{(}\PYG{n}{s}\PYG{p}{,}\PYG{+w}{ }\PYG{o}{\PYGZam{}}\PYG{n}{s}\PYG{o}{\PYGZhy{}}\PYG{o}{\PYGZgt{}}\PYG{n}{lock}\PYG{p}{)}\PYG{p}{;}
\PYG{+w}{    }\PYG{p}{\PYGZcb{}}
\PYG{+w}{    }
\PYG{+w}{    }\PYG{n}{release}\PYG{p}{(}\PYG{o}{\PYGZam{}}\PYG{n}{s}\PYG{o}{\PYGZhy{}}\PYG{o}{\PYGZgt{}}\PYG{n}{lock}\PYG{p}{)}\PYG{p}{;}
\PYG{p}{\PYGZcb{}}

\end{sphinxVerbatim}

\sphinxAtStartPar
sem\_wait 系统调用对应信号量的 P 操作(获取资源/减少信号量),其使得 sem 指向的信号量的 value 减一,若 value 取值小于 0,则阻塞自身,否则进程继续执行,若操作成功则返回0,否则返回\sphinxhyphen{}1。
sem\_wait 同样检查 state,然后将 value 减一,若 value<0,说明进程应该被阻塞,
设置睡眠通道,睡眠,被唤醒后,从等待队列中移除。(注意:这里不需要手动移除,由sempost负责)


\subsubsection{sys\_semdestroy 系统调用}
\label{\detokenize{nju:sys-semdestroy}}
\begin{sphinxVerbatim}[commandchars=\\\{\}]
\PYG{n}{uint64}
\PYG{n+nf}{sys\PYGZus{}semdestroy}\PYG{p}{(}\PYG{k+kt}{void}\PYG{p}{)}
\PYG{p}{\PYGZob{}}
\PYG{+w}{    }\PYG{k+kt}{int}\PYG{+w}{ }\PYG{n}{semid}\PYG{p}{;}
\PYG{+w}{    }
\PYG{+w}{    }\PYG{k}{if}\PYG{p}{(}\PYG{n}{argint}\PYG{p}{(}\PYG{l+m+mi}{0}\PYG{p}{,}\PYG{+w}{ }\PYG{o}{\PYGZam{}}\PYG{n}{semid}\PYG{p}{)}\PYG{+w}{ }\PYG{o}{\PYGZlt{}}\PYG{+w}{ }\PYG{l+m+mi}{0}\PYG{p}{)}
\PYG{+w}{        }\PYG{k}{return}\PYG{+w}{ }\PYG{l+m+mi}{\PYGZhy{}1}\PYG{p}{;}
\PYG{+w}{    }
\PYG{+w}{    }\PYG{k}{return}\PYG{+w}{ }\PYG{n}{semfree}\PYG{p}{(}\PYG{n}{semid}\PYG{p}{)}\PYG{p}{;}
\PYG{p}{\PYGZcb{}}
\end{sphinxVerbatim}

\begin{sphinxVerbatim}[commandchars=\\\{\}]
\PYG{k+kt}{int}
\PYG{n+nf}{semfree}\PYG{p}{(}\PYG{k+kt}{int}\PYG{+w}{ }\PYG{n}{semid}\PYG{p}{)}
\PYG{p}{\PYGZob{}}
\PYG{+w}{    }\PYG{k}{if}\PYG{p}{(}\PYG{n}{semid}\PYG{+w}{ }\PYG{o}{\PYGZlt{}}\PYG{+w}{ }\PYG{l+m+mi}{0}\PYG{+w}{ }\PYG{o}{|}\PYG{o}{|}\PYG{+w}{ }\PYG{n}{semid}\PYG{+w}{ }\PYG{o}{\PYGZgt{}}\PYG{o}{=}\PYG{+w}{ }\PYG{n}{MAX\PYGZus{}SEM\PYGZus{}NUM}\PYG{p}{)}
\PYG{+w}{        }\PYG{k}{return}\PYG{+w}{ }\PYG{l+m+mi}{\PYGZhy{}1}\PYG{p}{;}
\PYG{+w}{    }
\PYG{+w}{    }\PYG{n}{acquire}\PYG{p}{(}\PYG{o}{\PYGZam{}}\PYG{n}{semslock}\PYG{p}{)}\PYG{p}{;}
\PYG{+w}{    }\PYG{k}{if}\PYG{p}{(}\PYG{n}{sem}\PYG{p}{[}\PYG{n}{semid}\PYG{p}{]}\PYG{p}{.}\PYG{n}{valid}\PYG{+w}{ }\PYG{o}{=}\PYG{o}{=}\PYG{+w}{ }\PYG{l+m+mi}{0}\PYG{p}{)}\PYG{+w}{ }\PYG{p}{\PYGZob{}}
\PYG{+w}{        }\PYG{n}{release}\PYG{p}{(}\PYG{o}{\PYGZam{}}\PYG{n}{semslock}\PYG{p}{)}\PYG{p}{;}
\PYG{+w}{        }\PYG{k}{return}\PYG{+w}{ }\PYG{l+m+mi}{\PYGZhy{}1}\PYG{p}{;}
\PYG{+w}{    }\PYG{p}{\PYGZcb{}}
\PYG{+w}{    }\PYG{n}{sem}\PYG{p}{[}\PYG{n}{semid}\PYG{p}{]}\PYG{p}{.}\PYG{n}{valid}\PYG{+w}{ }\PYG{o}{=}\PYG{+w}{ }\PYG{l+m+mi}{0}\PYG{p}{;}
\PYG{+w}{    }\PYG{n}{release}\PYG{p}{(}\PYG{o}{\PYGZam{}}\PYG{n}{semslock}\PYG{p}{)}\PYG{p}{;}
\PYG{+w}{    }\PYG{k}{return}\PYG{+w}{ }\PYG{l+m+mi}{0}\PYG{p}{;}
\PYG{p}{\PYGZcb{}}
\end{sphinxVerbatim}

\sphinxAtStartPar
sem\_destroy 系统调用用于销毁 sem 指向的信号量,销毁成功则返回 0,否则返
回\sphinxhyphen{}1,若尚有进程阻塞在该信号量上,可带来未知错误。
sem\_destroy 直接将信号量的 state 清零,并触发时钟中断重新调度即可。


\subsection{信号量解决生产者消费者问题}
\label{\detokenize{nju:id27}}

\subsubsection{main 函数}
\label{\detokenize{nju:id28}}
\begin{sphinxVerbatim}[commandchars=\\\{\}]
\PYG{k+kt}{int}
\PYG{n+nf}{main}\PYG{p}{(}\PYG{k+kt}{void}\PYG{p}{)}
\PYG{p}{\PYGZob{}}
\PYG{+w}{    }\PYG{k}{if}\PYG{p}{(}\PYG{n}{open}\PYG{p}{(}\PYG{l+s}{\PYGZdq{}}\PYG{l+s}{console}\PYG{l+s}{\PYGZdq{}}\PYG{p}{,}\PYG{+w}{ }\PYG{n}{O\PYGZus{}RDWR}\PYG{p}{)}\PYG{+w}{ }\PYG{o}{\PYGZlt{}}\PYG{+w}{ }\PYG{l+m+mi}{0}\PYG{p}{)}\PYG{p}{\PYGZob{}}
\PYG{+w}{        }\PYG{n}{mknod}\PYG{p}{(}\PYG{l+s}{\PYGZdq{}}\PYG{l+s}{console}\PYG{l+s}{\PYGZdq{}}\PYG{p}{,}\PYG{+w}{ }\PYG{n}{CONSOLE}\PYG{p}{,}\PYG{+w}{ }\PYG{l+m+mi}{0}\PYG{p}{)}\PYG{p}{;}
\PYG{+w}{        }\PYG{n}{open}\PYG{p}{(}\PYG{l+s}{\PYGZdq{}}\PYG{l+s}{console}\PYG{l+s}{\PYGZdq{}}\PYG{p}{,}\PYG{+w}{ }\PYG{n}{O\PYGZus{}RDWR}\PYG{p}{)}\PYG{p}{;}
\PYG{+w}{    }\PYG{p}{\PYGZcb{}}
\PYG{+w}{    }\PYG{n}{dup}\PYG{p}{(}\PYG{l+m+mi}{0}\PYG{p}{)}\PYG{p}{;}\PYG{+w}{  }\PYG{c+c1}{// stdout}
\PYG{+w}{    }\PYG{n}{dup}\PYG{p}{(}\PYG{l+m+mi}{0}\PYG{p}{)}\PYG{p}{;}\PYG{+w}{  }\PYG{c+c1}{// stderr}
\PYG{+w}{   }
\PYG{+w}{    }\PYG{k+kt}{int}\PYG{+w}{ }\PYG{n}{i}\PYG{+w}{ }\PYG{o}{=}\PYG{+w}{ }\PYG{l+m+mi}{4}\PYG{p}{;}
\PYG{+w}{    }\PYG{k+kt}{int}\PYG{+w}{ }\PYG{n}{semid}\PYG{+w}{ }\PYG{o}{=}\PYG{+w}{ }\PYG{l+m+mi}{0}\PYG{p}{;}

\PYG{+w}{    }\PYG{k+kt}{int}\PYG{+w}{ }\PYG{n}{pid}\PYG{o}{=}\PYG{l+m+mi}{0}\PYG{p}{;}
\PYG{+w}{    }\PYG{c+c1}{//sem\PYGZus{}t sem;}
\PYG{+w}{    }\PYG{n}{printf}\PYG{p}{(}\PYG{l+s}{\PYGZdq{}}\PYG{l+s}{Father Process: Semaphore Initializing.}\PYG{l+s+se}{\PYGZbs{}n}\PYG{l+s}{\PYGZdq{}}\PYG{p}{)}\PYG{p}{;}
\PYG{+w}{    }\PYG{n}{semid}\PYG{+w}{ }\PYG{o}{=}\PYG{+w}{ }\PYG{n}{seminit}\PYG{p}{(}\PYG{l+m+mi}{2}\PYG{p}{)}\PYG{p}{;}
\PYG{+w}{    }\PYG{c+c1}{//printf(\PYGZdq{}\PYGZpc{}d\PYGZbs{}n\PYGZdq{},semid);}
\PYG{+w}{    }\PYG{n}{pid}\PYG{+w}{ }\PYG{o}{=}\PYG{+w}{ }\PYG{n}{fork}\PYG{p}{(}\PYG{p}{)}\PYG{p}{;}
\PYG{+w}{    }\PYG{k}{if}\PYG{+w}{ }\PYG{p}{(}\PYG{n}{pid}\PYG{+w}{ }\PYG{o}{=}\PYG{o}{=}\PYG{+w}{ }\PYG{l+m+mi}{0}\PYG{p}{)}\PYG{+w}{ }\PYG{p}{\PYGZob{}}
\PYG{+w}{        }\PYG{k}{while}\PYG{p}{(}\PYG{+w}{ }\PYG{n}{i}\PYG{+w}{ }\PYG{o}{!}\PYG{o}{=}\PYG{+w}{ }\PYG{l+m+mi}{0}\PYG{p}{)}\PYG{+w}{ }\PYG{p}{\PYGZob{}}
\PYG{+w}{            }\PYG{n}{i}\PYG{+w}{ }\PYG{o}{\PYGZhy{}}\PYG{o}{\PYGZhy{}}\PYG{p}{;}
\PYG{+w}{            }\PYG{n}{printf}\PYG{p}{(}\PYG{l+s}{\PYGZdq{}}\PYG{l+s}{Child Process: Semaphore Waiting.}\PYG{l+s+se}{\PYGZbs{}n}\PYG{l+s}{\PYGZdq{}}\PYG{p}{)}\PYG{p}{;}
\PYG{+w}{            }\PYG{n}{semwait}\PYG{p}{(}\PYG{n}{semid}\PYG{p}{)}\PYG{p}{;}
\PYG{+w}{            }\PYG{n}{printf}\PYG{p}{(}\PYG{l+s}{\PYGZdq{}}\PYG{l+s}{Child Process: In Critical Area.}\PYG{l+s+se}{\PYGZbs{}n}\PYG{l+s}{\PYGZdq{}}\PYG{p}{)}\PYG{p}{;}
\PYG{+w}{            }
\PYG{+w}{        }\PYG{p}{\PYGZcb{}}
\PYG{+w}{        }\PYG{n}{printf}\PYG{p}{(}\PYG{l+s}{\PYGZdq{}}\PYG{l+s}{Child Process: Semaphore Destroying.}\PYG{l+s+se}{\PYGZbs{}n}\PYG{l+s}{\PYGZdq{}}\PYG{p}{)}\PYG{p}{;}
\PYG{+w}{        }\PYG{n}{semdestroy}\PYG{p}{(}\PYG{n}{semid}\PYG{p}{)}\PYG{p}{;}
\PYG{+w}{        }\PYG{n}{exit}\PYG{p}{(}\PYG{l+m+mi}{1}\PYG{p}{)}\PYG{p}{;}
\PYG{+w}{    }\PYG{p}{\PYGZcb{}}
\PYG{k}{else}\PYG{+w}{ }\PYG{k}{if}\PYG{+w}{ }\PYG{p}{(}\PYG{n}{pid}\PYG{+w}{ }\PYG{o}{!}\PYG{o}{=}\PYG{+w}{ }\PYG{l+m+mi}{\PYGZhy{}1}\PYG{p}{)}\PYG{+w}{ }\PYG{p}{\PYGZob{}}
\PYG{+w}{        }\PYG{k}{while}\PYG{p}{(}\PYG{+w}{ }\PYG{n}{i}\PYG{+w}{ }\PYG{o}{!}\PYG{o}{=}\PYG{+w}{ }\PYG{l+m+mi}{0}\PYG{p}{)}\PYG{+w}{ }\PYG{p}{\PYGZob{}}
\PYG{+w}{            }\PYG{n}{i}\PYG{+w}{ }\PYG{o}{\PYGZhy{}}\PYG{o}{\PYGZhy{}}\PYG{p}{;}
\PYG{+w}{            }\PYG{n}{printf}\PYG{p}{(}\PYG{l+s}{\PYGZdq{}}\PYG{l+s}{Father Process: Sleeping.}\PYG{l+s+se}{\PYGZbs{}n}\PYG{l+s}{\PYGZdq{}}\PYG{p}{)}\PYG{p}{;}
\PYG{+w}{            }\PYG{n}{sleep}\PYG{p}{(}\PYG{l+m+mi}{128}\PYG{p}{)}\PYG{p}{;}
\PYG{+w}{            }\PYG{n}{printf}\PYG{p}{(}\PYG{l+s}{\PYGZdq{}}\PYG{l+s}{Father Process: Semaphore Posting.}\PYG{l+s+se}{\PYGZbs{}n}\PYG{l+s}{\PYGZdq{}}\PYG{p}{)}\PYG{p}{;}
\PYG{+w}{            }\PYG{n}{sempost}\PYG{p}{(}\PYG{n}{semid}\PYG{p}{)}\PYG{p}{;}
\PYG{+w}{        }\PYG{p}{\PYGZcb{}}
\PYG{+w}{        }\PYG{n}{printf}\PYG{p}{(}\PYG{l+s}{\PYGZdq{}}\PYG{l+s}{Father Process: Semaphore Destroying.}\PYG{l+s+se}{\PYGZbs{}n}\PYG{l+s}{\PYGZdq{}}\PYG{p}{)}\PYG{p}{;}
\PYG{+w}{        }\PYG{n}{semdestroy}\PYG{p}{(}\PYG{n}{semid}\PYG{p}{)}\PYG{p}{;}
\PYG{+w}{        }\PYG{c+c1}{//exit();}
\PYG{+w}{    }\PYG{p}{\PYGZcb{}}
\end{sphinxVerbatim}

\begin{sphinxVerbatim}[commandchars=\\\{\}]
\PYG{k+kt}{int}\PYG{+w}{ }\PYG{n}{ret}\PYG{o}{=}\PYG{l+m+mi}{\PYGZhy{}1}\PYG{p}{;}
\PYG{n}{mutex}\PYG{o}{=}\PYG{n}{seminit}\PYG{p}{(}\PYG{l+m+mi}{1}\PYG{p}{)}\PYG{p}{;}
\PYG{n}{buffer}\PYG{o}{=}\PYG{n}{seminit}\PYG{p}{(}\PYG{l+m+mi}{0}\PYG{p}{)}\PYG{p}{;}
\PYG{+w}{   }\PYG{k}{for}\PYG{+w}{ }\PYG{p}{(}\PYG{k+kt}{int}\PYG{+w}{ }\PYG{n}{i}\PYG{+w}{ }\PYG{o}{=}\PYG{+w}{ }\PYG{l+m+mi}{0}\PYG{p}{;}\PYG{+w}{ }\PYG{n}{i}\PYG{+w}{ }\PYG{o}{\PYGZlt{}}\PYG{+w}{ }\PYG{l+m+mi}{6}\PYG{p}{;}\PYG{+w}{ }\PYG{o}{+}\PYG{o}{+}\PYG{n}{i}\PYG{p}{)}\PYG{+w}{ }\PYG{p}{\PYGZob{}}
\PYG{+w}{  }\PYG{c+c1}{// printf(\PYGZdq{}in re\PYGZbs{}n\PYGZdq{});}
\PYG{+w}{      }\PYG{n}{ret}\PYG{o}{=}\PYG{n}{fork}\PYG{p}{(}\PYG{p}{)}\PYG{p}{;}
\PYG{+w}{      }\PYG{c+c1}{//test1();     }
\PYG{+w}{     }\PYG{c+c1}{// printf(\PYGZdq{}i=\PYGZpc{}d\PYGZbs{}n\PYGZdq{},i);}
\PYG{+w}{     }
\PYG{+w}{        }\PYG{k}{if}\PYG{+w}{ }\PYG{p}{(}\PYG{n}{ret}\PYG{+w}{ }\PYG{o}{=}\PYG{o}{=}\PYG{+w}{ }\PYG{l+m+mi}{0}\PYG{p}{)}\PYG{+w}{ }\PYG{p}{\PYGZob{}}
\PYG{+w}{            }\PYG{k}{if}\PYG{+w}{ }\PYG{p}{(}\PYG{n}{i}\PYG{+w}{ }\PYG{o}{\PYGZlt{}}\PYG{+w}{ }\PYG{l+m+mi}{2}\PYG{p}{)}\PYG{p}{\PYGZob{}}
\PYG{+w}{                }\PYG{n}{producer}\PYG{p}{(}\PYG{n}{i}\PYG{+w}{ }\PYG{o}{+}\PYG{+w}{ }\PYG{l+m+mi}{1}\PYG{p}{)}\PYG{p}{;}
\PYG{+w}{                }\PYG{p}{\PYGZcb{}}
\PYG{+w}{            }\PYG{k}{else}\PYG{p}{\PYGZob{}}
\PYG{+w}{                }\PYG{n}{consumer}\PYG{p}{(}\PYG{n}{i}\PYG{+w}{ }\PYG{o}{\PYGZhy{}}\PYG{+w}{ }\PYG{l+m+mi}{1}\PYG{p}{)}\PYG{p}{;}
\PYG{+w}{                }\PYG{p}{\PYGZcb{}}
\PYG{+w}{                }
\PYG{+w}{            }\PYG{c+c1}{//exit(1);}
\PYG{+w}{            }\PYG{k}{break}\PYG{p}{;}
\PYG{+w}{            }\PYG{c+c1}{//while(1);}
\PYG{+w}{        }\PYG{p}{\PYGZcb{}}
\PYG{+w}{    }\PYG{p}{\PYGZcb{}}
\PYG{+w}{        }\PYG{k}{while}\PYG{+w}{ }\PYG{p}{(}\PYG{l+m+mi}{1}\PYG{p}{)}\PYG{p}{;}
\PYG{+w}{    }\PYG{n}{semdestroy}\PYG{p}{(}\PYG{n}{mutex}\PYG{p}{)}\PYG{p}{;}
\PYG{+w}{    }\PYG{n}{semdestroy}\PYG{p}{(}\PYG{n}{buffer}\PYG{p}{)}\PYG{p}{;}



\PYG{+w}{   }
\PYG{+w}{    }\PYG{k}{while}\PYG{p}{(}\PYG{l+m+mi}{1}\PYG{p}{)}\PYG{p}{\PYGZob{}}
\PYG{+w}{    }\PYG{p}{\PYGZcb{}}
\PYG{+w}{    }\PYG{k}{return}\PYG{+w}{ }\PYG{l+m+mi}{0}\PYG{p}{;}
\PYG{p}{\PYGZcb{}}
\end{sphinxVerbatim}

\sphinxAtStartPar
main 函数首先对系统调用和库函数的测试,其次,需要准备两个信号量 mutex(用于互斥)和 buffer(用于生产者消费者同步),value 分别为 1 和 0,然后循环 fork 出 6 个子进程,根据 fork 的返回值,若为 0 说明是子进程,调用相应的生产者消费者函数,并跳出循环,否则是父进程,继续循环。


\subsubsection{生产者进程}
\label{\detokenize{nju:id29}}
\begin{sphinxVerbatim}[commandchars=\\\{\}]
\PYG{k+kt}{void}\PYG{+w}{ }\PYG{n+nf}{producer}\PYG{p}{(}\PYG{k+kt}{int}\PYG{+w}{ }\PYG{n}{arg}\PYG{p}{)}\PYG{+w}{ }\PYG{p}{\PYGZob{}}

\PYG{+w}{    }\PYG{k+kt}{int}\PYG{+w}{ }\PYG{n}{pid}\PYG{+w}{ }\PYG{o}{=}\PYG{+w}{ }\PYG{n}{getpid}\PYG{p}{(}\PYG{p}{)}\PYG{p}{;}
\PYG{+w}{    }\PYG{k}{for}\PYG{+w}{ }\PYG{p}{(}\PYG{k+kt}{int}\PYG{+w}{ }\PYG{n}{k}\PYG{+w}{ }\PYG{o}{=}\PYG{+w}{ }\PYG{l+m+mi}{1}\PYG{p}{;}\PYG{+w}{ }\PYG{n}{k}\PYG{+w}{ }\PYG{o}{\PYGZlt{}}\PYG{o}{=}\PYG{+w}{ }\PYG{l+m+mi}{8}\PYG{p}{;}\PYG{+w}{ }\PYG{o}{+}\PYG{o}{+}\PYG{n}{k}\PYG{p}{)}\PYG{+w}{ }\PYG{p}{\PYGZob{}}
\PYG{+w}{        }\PYG{n}{sleep}\PYG{p}{(}\PYG{l+m+mi}{64}\PYG{p}{)}\PYG{p}{;}
\PYG{+w}{        }\PYG{n}{printf}\PYG{p}{(}\PYG{l+s}{\PYGZdq{}}\PYG{l+s}{pid \PYGZpc{}d, producer \PYGZpc{}d, produce, product \PYGZpc{}d}\PYG{l+s+se}{\PYGZbs{}n}\PYG{l+s}{\PYGZdq{}}\PYG{p}{,}\PYG{+w}{ }\PYG{n}{pid}\PYG{p}{,}\PYG{+w}{ }\PYG{n}{arg}\PYG{p}{,}\PYG{+w}{ }\PYG{n}{k}\PYG{p}{)}\PYG{p}{;}
\PYG{+w}{        }\PYG{n}{printf}\PYG{p}{(}\PYG{l+s}{\PYGZdq{}}\PYG{l+s}{pid \PYGZpc{}d, producer \PYGZpc{}d, try lock, product \PYGZpc{}d}\PYG{l+s+se}{\PYGZbs{}n}\PYG{l+s}{\PYGZdq{}}\PYG{p}{,}\PYG{+w}{ }\PYG{n}{pid}\PYG{p}{,}\PYG{+w}{ }\PYG{n}{arg}\PYG{p}{,}\PYG{+w}{ }\PYG{n}{k}\PYG{p}{)}\PYG{p}{;}
\PYG{+w}{        }\PYG{n}{semwait}\PYG{p}{(}\PYG{n}{mutex}\PYG{p}{)}\PYG{p}{;}
\PYG{+w}{        }\PYG{n}{printf}\PYG{p}{(}\PYG{l+s}{\PYGZdq{}}\PYG{l+s}{pid \PYGZpc{}d, producer \PYGZpc{}d, locked}\PYG{l+s+se}{\PYGZbs{}n}\PYG{l+s}{\PYGZdq{}}\PYG{p}{,}\PYG{+w}{ }\PYG{n}{pid}\PYG{p}{,}\PYG{+w}{ }\PYG{n}{arg}\PYG{p}{)}\PYG{p}{;}
\PYG{+w}{        }\PYG{n}{sempost}\PYG{p}{(}\PYG{n}{mutex}\PYG{p}{)}\PYG{p}{;}
\PYG{+w}{        }\PYG{n}{printf}\PYG{p}{(}\PYG{l+s}{\PYGZdq{}}\PYG{l+s}{pid \PYGZpc{}d, producer \PYGZpc{}d, unlock}\PYG{l+s+se}{\PYGZbs{}n}\PYG{l+s}{\PYGZdq{}}\PYG{p}{,}\PYG{+w}{ }\PYG{n}{pid}\PYG{p}{,}\PYG{+w}{ }\PYG{n}{arg}\PYG{p}{)}\PYG{p}{;}
\PYG{+w}{       }\PYG{n}{sempost}\PYG{p}{(}\PYG{n}{buffer}\PYG{p}{)}\PYG{p}{;}
\PYG{+w}{    }\PYG{p}{\PYGZcb{}}
\PYG{p}{\PYGZcb{}}

\end{sphinxVerbatim}

\sphinxAtStartPar
生产者循环生产,生产过程用 sleep(64)模拟,生产完成打印 produce 信息,接着
尝试获取 mutex 实现互斥访问,最后对 buffer 执行 V 操作表示 buffer 内已经有
产品了。


\subsubsection{消费者进程}
\label{\detokenize{nju:id30}}
\begin{sphinxVerbatim}[commandchars=\\\{\}]
\PYG{k+kt}{void}\PYG{+w}{ }\PYG{n+nf}{consumer}\PYG{p}{(}\PYG{k+kt}{int}\PYG{+w}{ }\PYG{n}{arg}\PYG{p}{)}\PYG{+w}{ }\PYG{p}{\PYGZob{}}

\PYG{+w}{    }\PYG{k+kt}{int}\PYG{+w}{ }\PYG{n}{pid}\PYG{+w}{ }\PYG{o}{=}\PYG{+w}{ }\PYG{n}{getpid}\PYG{p}{(}\PYG{p}{)}\PYG{p}{;}
\PYG{+w}{    }\PYG{k}{for}\PYG{+w}{ }\PYG{p}{(}\PYG{k+kt}{int}\PYG{+w}{ }\PYG{n}{k}\PYG{+w}{ }\PYG{o}{=}\PYG{+w}{ }\PYG{l+m+mi}{1}\PYG{p}{;}\PYG{+w}{ }\PYG{n}{k}\PYG{+w}{ }\PYG{o}{\PYGZlt{}}\PYG{o}{=}\PYG{+w}{ }\PYG{l+m+mi}{4}\PYG{p}{;}\PYG{+w}{ }\PYG{o}{+}\PYG{o}{+}\PYG{n}{k}\PYG{p}{)}\PYG{+w}{ }\PYG{p}{\PYGZob{}}
\PYG{+w}{        }\PYG{n}{printf}\PYG{p}{(}\PYG{l+s}{\PYGZdq{}}\PYG{l+s}{pid \PYGZpc{}d, consumer \PYGZpc{}d, try consume, product \PYGZpc{}d}\PYG{l+s+se}{\PYGZbs{}n}\PYG{l+s}{\PYGZdq{}}\PYG{p}{,}\PYG{+w}{ }\PYG{n}{pid}\PYG{p}{,}\PYG{+w}{ }\PYG{n}{arg}\PYG{p}{,}\PYG{+w}{ }\PYG{n}{k}\PYG{p}{)}\PYG{p}{;}
\PYG{+w}{        }\PYG{n}{semwait}\PYG{p}{(}\PYG{n}{buffer}\PYG{p}{)}\PYG{p}{;}
\PYG{+w}{        }\PYG{n}{printf}\PYG{p}{(}\PYG{l+s}{\PYGZdq{}}\PYG{l+s}{pid \PYGZpc{}d, consumer \PYGZpc{}d, try lock, product \PYGZpc{}d}\PYG{l+s+se}{\PYGZbs{}n}\PYG{l+s}{\PYGZdq{}}\PYG{p}{,}\PYG{+w}{ }\PYG{n}{pid}\PYG{p}{,}\PYG{+w}{ }\PYG{n}{arg}\PYG{p}{,}\PYG{+w}{ }\PYG{n}{k}\PYG{p}{)}\PYG{p}{;}
\PYG{+w}{        }\PYG{n}{semwait}\PYG{p}{(}\PYG{n}{mutex}\PYG{p}{)}\PYG{p}{;}
\PYG{+w}{        }
\PYG{+w}{        }\PYG{n}{printf}\PYG{p}{(}\PYG{l+s}{\PYGZdq{}}\PYG{l+s}{pid \PYGZpc{}d, consumer \PYGZpc{}d, locked}\PYG{l+s+se}{\PYGZbs{}n}\PYG{l+s}{\PYGZdq{}}\PYG{p}{,}\PYG{+w}{ }\PYG{n}{pid}\PYG{p}{,}\PYG{+w}{ }\PYG{n}{arg}\PYG{p}{)}\PYG{p}{;}
\PYG{+w}{        }\PYG{n}{sempost}\PYG{p}{(}\PYG{n}{mutex}\PYG{p}{)}\PYG{p}{;}
\PYG{+w}{        }\PYG{n}{printf}\PYG{p}{(}\PYG{l+s}{\PYGZdq{}}\PYG{l+s}{pid \PYGZpc{}d, consumer \PYGZpc{}d, unlock}\PYG{l+s+se}{\PYGZbs{}n}\PYG{l+s}{\PYGZdq{}}\PYG{p}{,}\PYG{+w}{ }\PYG{n}{pid}\PYG{p}{,}\PYG{+w}{ }\PYG{n}{arg}\PYG{p}{)}\PYG{p}{;}
\PYG{+w}{        }\PYG{n}{sleep}\PYG{p}{(}\PYG{l+m+mi}{64}\PYG{p}{)}\PYG{p}{;}
\PYG{+w}{        }\PYG{n}{printf}\PYG{p}{(}\PYG{l+s}{\PYGZdq{}}\PYG{l+s}{pid \PYGZpc{}d, consumer \PYGZpc{}d, consumed, product \PYGZpc{}d}\PYG{l+s+se}{\PYGZbs{}n}\PYG{l+s}{\PYGZdq{}}\PYG{p}{,}\PYG{+w}{ }\PYG{n}{pid}\PYG{p}{,}\PYG{+w}{ }\PYG{n}{arg}\PYG{p}{,}\PYG{+w}{ }\PYG{n}{k}\PYG{p}{)}\PYG{p}{;}
\PYG{+w}{    }\PYG{p}{\PYGZcb{}}

\PYG{p}{\PYGZcb{}}
\end{sphinxVerbatim}

\sphinxAtStartPar
消费者循环消费,打印一条 try consume 信息,并对 buffer 执行 P 操作,此时若
buffer 为空则被阻塞,接着获取 mutex 访问临界区,最后用 sleep(64)模拟消费过
程,在完成消费后打印 consumed 信息。


\section{实验结果}
\label{\detokenize{nju:id31}}
\sphinxAtStartPar
(1) 用户程序测试
由框架代码给出的对系统调用和库函数的测试部分的实验结果如下
\sphinxincludegraphics{{img5}.png}\\
对执行过程的分析如下
\begin{enumerate}
\sphinxsetlistlabels{\arabic}{enumi}{enumii}{}{.}%
\item {} 
\sphinxAtStartPar
父进程初始化信号量,fork 出子进程,打印 sleeping 后便去睡眠。

\item {} 
\sphinxAtStartPar
由于信号量的值为 2,子进程可以进入两次关键区,在第三次时被阻塞。

\item {} 
\sphinxAtStartPar
父进程苏醒,释放信号量,又去睡眠

\item {} 
\sphinxAtStartPar
被阻塞的子进程释放,再次进入关键区,又被阻塞。

\item {} 
\sphinxAtStartPar
同 4。

\item {} 
\sphinxAtStartPar
被阻塞的子进程释放,销毁信号量,退出。

\item {} 
\sphinxAtStartPar
只有父进程运行,最终销毁信号量。

\end{enumerate}

\sphinxAtStartPar
(2) 生产者消费者
生产者消费者进程的执行结果较长,截取部分\\
\sphinxincludegraphics{{img6}.png}\\
对执行结果分析如下
\begin{enumerate}
\sphinxsetlistlabels{\arabic}{enumi}{enumii}{}{.}%
\item {} 
\sphinxAtStartPar
生产者 1、2 均执行 sleep 模拟生产过程,消费者 1、2、3、4 执行 try consume
但由于 buffer 为空被阻塞。

\item {} 
\sphinxAtStartPar
生产者 1、2 各完成生产一个产品。

\item {} 
\sphinxAtStartPar
消费者 1、2 各开始消费一个产品。

\item {} 
\sphinxAtStartPar
生产者 1、2 各完成生产一个产品。

\item {} 
\sphinxAtStartPar
消费者 1、2 各完成消费第一个产品,尝试消费第二个产品被阻塞。

\item {} 
\sphinxAtStartPar
消费者 3、4 各开始消费一个产品。

\end{enumerate}



\renewcommand{\indexname}{索引}
\printindex
\end{document}